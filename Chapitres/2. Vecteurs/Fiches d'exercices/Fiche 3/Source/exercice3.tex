\documentclass{book}
\usepackage[utf8]{inputenc}
%\usepackage[latin1]{inputenc}
\usepackage[T1]{fontenc}
\usepackage{amsmath,amssymb}
\usepackage{array,color,multirow,slashbox,multicol}
\usepackage{mathrsfs}
\usepackage{eurosym}
\usepackage{stmaryrd} %Pour le symbole parallele \\sslash
\usepackage{yhmath}   %pour dessiner les arcs \wideparen
%\usepackage{multicol}
\usepackage[portrait,nofootskip]{chingatome}








\begin{document}

\fontsize{10}{12}\fontfamily{cmr}\selectfont\titre[18pt]{Vecteurs et équations de droites - Fiche d'exercices 3}

\fontsize{10}{12}\fontfamily{cmr}\selectfont\begin{multicols*}{2}


%%%%%%%%%%%%%%%%%
\leavevmode\exercice


Dans le plan muni d'un rep\`ere $\Se\big(O\,;\,I\,;\,J\big)$ orthonorm\'e, on consid\`ere les deux points suivants :\newline
\hglue\leftmargini$\se A\coord{-4}{-2}$%
\quad\string;\quad%
$B\coord{-1}{2}$

\leavevmode\hfil\hbox{\Image{6481_graph-1.pdf}{1}{93}{81}}

\vskip\parskip
\begin{enumerate}
\item Placer les points $A$ et $B$.
\end{enumerate}

Le graphique sera compl\'et\'e au fur et \`a mesure des questions l'exercice.

\vskip\parskip
\begin{enumerate}
\setcounter{enumi}1
\item On note $K$ le milieu du segment $[AB]$. Montrer que le point $K$ a pour coordonn\'ees :%
\quad$K\coord{-2,5}{0}$.

\item On consid\`ere le point $C$ de coordonn\'ees $\se\coord{-2,5}{-2,5}$.

\vskip\parskip
\begin{enumerate}
\item D\'eterminer les longueurs $AB$ et $KC$.

\item Que repr\'esente le segment $[KC]$ pour le triangle $ABC$?

\item En d\'eduire que le triangle $ABC$ est rectangle en $C$.
\end{enumerate}
\end{enumerate}

\leavevmode\exercice


On consid\`ere les quatres points suivants caract\'eris\'es par leurs coordonn\'ees dans un rep\`ere $\Se(O\,;\,I\,;\,J)$ orthonorm\'e :

\hglue\leftmargini$\Se A\coord{-4}{-1}$%
\quad\string;\quad%
$\Se B\coord{-3}{-4}$%
\quad\string;\quad%
$\Se C\coord{3}{-2}$%
\quad\string;\quad%
$\Se D\coord{2}{1}$

Montrer que le quadrilat\`ere $ABCD$ est un rectangle.


\leavevmode\exercice


On consid\`ere le plan muni d'un rep\`ere $\Se(O\,;\,I\,;\,J)$ et le cercle $\mathscr{C}$ de centre $\Se K\coord{2}{-3}$ et de rayon $5$.

\vskip\parskip
\begin{enumerate}
\item Justifier que le point $ \Se A\coord{6}{-6}$ est un point du cercle $\mathscr{C}$

\item Consid\'erons le point $B$ diam\'etralement oppos\'e au point $A$ dans le cercle $\mathscr{C}$. D\'eterminer les coordonn\'ees du point $B$.

\item Soit $C$ le point du plan de coordonn\'es $\Se\coord{-\dfrac{14}5}{-\dfrac85}$.\newline
Justifier que le triangle $ABC$ est rectangle en $C$.
\end{enumerate}

\leavevmode\exercice


\begin{enumerate}
\item Pour chacun des quadrans ci-dessous : 

\vskip\parskip
\begin{enumerate}
\item Placer le point $B$ translat\'e du point $A$ par la translation de vecteur $\Vec{u}$.

\item Tracer le point $C$ translat\'e du point $B$ par la translation de vecteur $\Vec{v}$.
\end{enumerate}

Dans chaque cadran, le point $C$ obtenu s'appelle le translat\'e du point $A$ par le vecteur $\se\Vec{u}+\Vec{v}$.

\item Dans le premier quadran :

\vskip\parskip
\begin{enumerate}
\item Placer le point $B'$ translat\'e du point $A$ par le vecteur $\Vec{v}$.

\item Placer le point $C'$ translat\'e du point $B'$ par le vecteur $\Vec{u}$.

\item Que pouvez-vous dire de la translation compos\'e des translations de vecteurs $\Vec{u}$ puis celle de $\Vec{v}$ et de la translation compos\'ee des translations de vecteurs $\Vec{v}$ et $\vec{u}$?
\end{enumerate}
\end{enumerate}

\leavevmode\hfil\hbox{\Image{6485_quadrillage-1.pdf}{1}{92}{71}}

\leavevmode\exercice


Dans le rep\`ere orthonorm\'e $\Se(O\,;\,I\,;\,J)$ ci-dessous, sont repr\'esent\'es quatres vecteurs :

\leavevmode\hfil\hbox{\Image{6486_graph-1.pdf}{1}{91}{55}}

Graphiquement, d\'eterminer les coordonn\'ees de ces quatres vecteurs.

\leavevmode\exercice


Dans un rep\`ere orthonorm\'e $\Se(O\,;\,I\,;\,J)$, on consid\`ere les quatres points suivants caract\'eris\'es par leurs coordonn\'ees :

\leavevmode\vtop{\openup6pt
\def\x{\hskip0.3cm}
\halign{$#$\hfill&&\x\string;\x$#$\hfil\cr
A\coord{\dfrac53}{\dfrac74}&
\Se B\coord{\dfrac{11}{3}}{-\dfrac{5}{4}}&
C\coord{\dfrac{16}{7}}{\dfrac{12}{5}}&
D\coord{\dfrac{2}7}{\dfrac{27}5}\cr}}

Justifier que le quadrilat\`ere $ABCD$ est un parall\'elogramme.

\leavevmode\exercice


Sur une droite gradu\'ee, on place les points $A$, $B$, $C$, $D$, $E$ :

\leavevmode\hfil\hbox{\Image{5287_droiteGraduee-1.pdf}{1}{86}{6}}

Pour chaque question, d\'eterminer la valeur du nombre $k$ v\'erifiant l'\'egalit\'e :

\questioni{a&\Vec{BC}=k\cdot\Vec{AC}&
b&\Vec{ED}=k\cdot\Vec{AC}\cr
c&\Vec{AC}=k\cdot\Vec{CA}&
d&\Vec{ED}=k\cdot\Vec{CA}\cr
e&\Vec{EA}=k\cdot\Vec{AB}&
f&\Vec{AC}=k\cdot\Vec{BA}\cr}

\leavevmode\exercice


Dans le cas o\`u les vecteurs $\Vec{u}$ et $\Vec{v}$ sont colin\'eaires, donner le coefficient de colin\'earit\'e du vecteur $\Vec{u}$ par rapport au vecteur $\Vec{v}$ :

\def\x #1#2#3#4{\Se\Vec{u}\coord{#1}{#2}\hbox{ et }\Vec{v}\coord{#3}{#4}}
\hglue-\leftmargini\questioni[0.25cm]{a&\x{-2}{-10}{4}{20}&
b&\x{-6}{9}{\dfrac14}{-\dfrac12}\cr
c&\x{0}{5}{-5}{0}&
d&\x{-\dfrac43}{4}{3}{-9}\cr
e&\x{\dfrac13}{\dfrac25}{5}{6}&
f&\x{6}{-5}{\dfrac{14}{5}}{-2}\cr}

\leavevmode\exercice


On munit le plan d'un rep\`ere $\Se\big(O\,;\,\Vec i\,;\,\Vec j\big)$ :

\vskip\parskip
\begin{enumerate}
\item Montrer que les points suivants sont align\'es :\newline
\hglue\leftmarginii$\Se A\coord{0}{-1}$%
\quad\string;\quad%
$B\coord{2}{0}$%
\quad\string;\quad%
$\Se C\coord{-2}{-2}$

\item D\'eterminer si les points suivants sont align\'es :\newline
\hglue\leftmarginii$K\coord{3}{-4}$%
\quad\string;\quad%
$L\coord{2}{-2}$%
\quad\string;\quad%
$M\coord{-1}{3}$

\item On consid\`ere les points ci-dessous :

\def\x{\hskip0.5cm minus1cm}
\hglue\leftmarginii$O\coord{3}{2}$%
\x\string;\x%
$P\coord{4}{5}$%
\x\string;\x%
$\Se Q\coord{1}{-202}$%
\x\string;\x%
$R\coord{101}{98}$

D\'eterminer si les droites $(OP)$  et $(QR)$ sont parall\`eles.
\end{enumerate}

\leavevmode\exercice


Dans un un repere $\Se (O;\Vec{i};\Vec{j})$, on consid\`ere les points :

\hglue0pt$\Se A\coord{3}{-5}$%
\hfill\string;\hfill%
$\Se B\coord{-2}{0}$%
\hfill\string;\hfill%
$\Se C\coord{147}{-13}$%
\hfill\string;\hfill%
$\Se D\coord{-53}{187}$%

Etablir que les droites $(AB)$ et $(CD)$ sont parall\`eles.

\leavevmode\exercice


On consid\`ere le plan muni du rep\`ere $\Se\big(O\,;\,\Vec{i}\,;\,\Vec{j}\big)$ repr\'esent\'e ci-dessous :


On consid\`ere les quatres vecteurs ci-dessous :\newline
\hglue\leftmargini$\Se\Vec{u}\coord{\dfrac{9}{4}}{{-}\dfrac{3}{4}}$%
\quad\string;\quad%
$\Se\Vec{v}\coord{\dfrac{7}{2}}{{-}\dfrac{3}{2}}$%
\quad\string;\quad%
$\Se\Vec{w}\coord{-\dfrac{15}{4}}{\dfrac{5}{4}}$

\vskip\parskip
\begin{enumerate}
\item Repr\'esenter les trois vecteurs $\Vec{u}$, $\Vec{v}$ et $\Vec{w}$ avec pour origine le point $O$.

\item \begin{enumerate}
\item Graphiquement, \'emettre une conjecture sur la colin\'earit\'e de couples de vecteurs parmi $\Vec{u}$, $\Vec{v}$ et $\Vec{w}$.

\item Etablir votre conjecture.
\end{enumerate}
\end{enumerate}



\leavevmode\exercice


Soit $A$, $B$, $C$ et $D$ quatre points du plan. Dans chaque cas, d\'emontrer que les vecteurs $\Vec{AB}$ et $\Vec{CD}$, v\'erifiant la relation impos\'ee, sont colin\'eaires :

\hglue-\leftmarginii\questioni[0.5cm]{a&\Vec{AB}+\Vec{AD}=\Vec{AC}&
b&5{\cdot}\Vec{AD}=2{\cdot}\Vec{AC}+3{\cdot}\Vec{BD}\cr
c&\Vec{AD}+\Vec{BD}+2{\cdot}\Vec{CB}=\Vec{0}&
d&3{\cdot}\Vec{AD}+4{\cdot}\Vec{BC}=7{\cdot}\Vec{AC}\cr}

\leavevmode\exercice


On consid\`ere le plan muni d'un rep\`ere $\Se\big(O\,;\,I\,;\,J\big)$.\newline
Soit $A$, $B$, $C$ et $D$ quatre points du plan de coordonn\'ees :

\hglue\leftmargini$A\coord{-5}{1}$%
\quad\string;\quad%
$B\coord{2}{4}$%
\quad\string;\quad%
$\Se C\coord{-1}{-2}$%
\quad\string;\quad%
$D\coord{3}{y_D}$

D\'eterminer les coordonn\'ees du point $D$ tel que les droites $(AB)$ et $(CD)$ soient parall\`eles et que le point $D$ ait $3$ pour abscisse.

\leavevmode\exercice


Dans le plan muni d'un rep\`ere $\Se\big(O\,;\,I\,;\,J\big)$ orthonorm\'e, on consid\`ere les trois points suivants :\newline
\hglue\leftmargini$A\coord{-1}{1}$%
\quad\string;\quad%
$\Se B\coord{-3}{-1}$%
\quad\string;\quad%
$C\coord{2}{3}$

\vskip\parskip
\begin{enumerate}
\item Les points $A$, $B$ et $C$ sont-ils align\'es? Justifier votre r\'eponse.

\item D\'eterminer les coordonn\'ees de l'unique point $D$ ayant pour abscisse $-2$ tel que les droites $(AB)$ et $(CD)$ soient parall\`eles.
\end{enumerate}

\leavevmode\exercice


On consid\`ere le plan muni d'un rep\`ere orthonorm\'e $\Se(O\,;\,I\,;\,J)$ :

\leavevmode\hfil\hbox{\Image{5292_graph-1.pdf}{1}{93}{73}}

\vskip\parskip
\begin{enumerate}
\item Placer les trois points $A$, $B$, $C$ dans le rep\`ere ci-dessous :

\hglue\leftmarginii$A\coord{3}{{-}3}$%
\quad\string;\quad%
$B\coord{-4}{3}$%
\quad\string;\quad%
$C\coord{-5}{{-}1}$

\item D\'eterminer les coordonn\'ees du milieu  $M$ du segment $[AB]$.

\item \begin{enumerate}
\item D\'eterminer les longueurs $AB$ et $MC$

\item Etablir que le triangle $ABC$ est rectangle en $C$.
\end{enumerate}

\item Soit $N$ un point de l'axe des ordonn\'ees. D\'eterminer les coordonn\'ees du point $N$ afin que les vecteurs $\Vec{BN}$ et $\Vec{CM}$ soient colin\'eaires.
\end{enumerate}

\end{multicols*}



\end{document}