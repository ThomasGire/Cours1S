\documentclass{book}
%\usepackage[latin1]{inputenc}
\usepackage[utf8]{inputenc}
\usepackage[T1]{fontenc}
\usepackage{amsmath,amssymb}
\usepackage{array,color,multirow,slashbox,multicol}
\usepackage{mathrsfs}
\usepackage{eurosym}
\usepackage{stmaryrd} %Pour le symbole parallele \\sslash
\usepackage{yhmath}   %pour dessiner les arcs \wideparen
%\usepackage{multicol}
\usepackage[portrait,nofootskip]{chingatome}








\begin{document}

\fontsize{10}{12}\fontfamily{cmr}\selectfont\titre[18pt]{Suites arithmétiques et suites géométriques}
\fontsize{10}{12}\fontfamily{cmr}\selectfont\begin{multicols*}{2}


%%%%%%%%%%%%%%%%%
\leavevmode\exercice


Compl\'eter les suites logiques de nombres pour obtenir les $8$ premiers termes de chacune d'elles :

\def\x #1#2#3#4{$#1$ - $#2$ - $#3$ - $#4$ - \ldots}
\vskip\parskip
\begin{enumerate}
\changelabel
\def\x #1{\leavevmode\hbox to0.8cm{\hfill$#1$\hfill}}
\def\y #1{- \x{$#1$}}
\item \x{4}\y{7}\y{10}\y{13}\y{\ldots}

\item \x{3}\y{6}\y{12}\y{24}\y{\ldots}

\item \x{20}\y{19}\y{17}\y{14}\y{\ldots}

\item \x{5}\y{7}\y{11}\y{17}\y{\ldots}

\item \x{1}\y{4}\y{9}\y{16}\y{\ldots}
\end{enumerate}

\leavevmode\exercice


On consid\`ere les deux proc\'ed\'es d'obtention suivant de nombres :

\leavevmode{\linewidth=0.48\hsize\boite{\leavevmode\hfil\hbox{\bf Proc\'edure A}\newline
On multiplie le nombre donn\'e par $3$}}%
\hfill%
{\linewidth0.48\hsize\boite{\leavevmode\hfil\hbox{\bf Proc\'edure B}\newline
Au nombre donn\'e, on lui soustrait $2$.}}

Pour chaque question, donner les six premiers termes obtenus en r\'ep\'etant les consignes autant de fois que n\'ecessaire.

\vskip\parskip
\begin{enumerate}
\item Le nombre de d\'epart est $3$ et on r\'ep\`ete la proc\'edure {\bf A} ;

\item Le nombre de d\'epart est $11$ et on r\'ep\`ete la proc\'edure {\bf B}.
\end{enumerate}

\leavevmode\exercice


\begin{enumerate}
\item Trouver les coefficients multiplicatifs repr\'esentant chacune des \'evolutions suivantes :

\questionii[0.5cm]{a&+10\,\%&
b&+2,5\,\% &
c&+115\,\%\cr
d&-22\,\%&
e&-10,7\,\%&
f&-65\,\%\cr}

\item Pour chaque coefficient multiplicateur, retrouver l'\'evolution associ\'ee et le pourcentage correspondant :

\questionii{a&1,02&
b&1,375& 
c&2,1\cr
d&0,15&
e&0,85&
f&0,912\cr}
\end{enumerate}

\leavevmode\exercice


%Liban - juin 2005 - 12 points
Des scientifiques \'etudient une culture de bact\'eries contenant deux souches qu'on nommera $A$ et $B$.\newline
Au d\'ebut de l'exp\'erience {\it(au temps ``0'')}, on d\'enombre $200$ de bact\'eries de souches $A$ et $300$ bact\'eries de souches $B$.\newline
Les scientifiques rel\`event les \'evoluations suivantes : \`a chaque minute, la population des bact\'eries $A$ augmente de $10\,\%$, alors que celle de la souche $B$ diminue de $20$ bact\'eries.

\vskip\parskip
\begin{enumerate}
\item  \begin{enumerate}
\item Au temps ``{\sl 0 min}'', quel est le pourcentage repr\'esent\'e par les bact\'eries de la souche $A$ par rapport \`a l'ensemble des bact\'eries?

\item Au temps ``{\sl 1 min}'', quel est le pourcentage repr\'esent\'e par les bact\'eries de la souche $A$ par rapport \`a l'ensemble des bact\'eries?

\item Compl\'eter le tableau ci-dessous :

\hglue-3\leftmarginii\hfil\hbox{\Image{2906_tableurExcel2-1.pdf}{1}{91}{61}}
\end{enumerate}

\item $n$ d\'esigne un nombre entier naturel {\it($\Se n\in\mathbb{R}$)}. 

On note $a_n$ la population de bact\'eries de la souche $A$ au temps ``{\sl $n$ min}''; ainsi, $\Se a_0=200$. 

On note $b_n$ la population de bact\'eries de la souche $B$ au temps ``{\sl $n$ min}'' ; ainsi $b_0=300$.

Compl\'eter les pointill\'es ci-dessous :

\hglue-\leftmargini\hfil\vbox{\offinterlineskip
\def\x{\ \raise-4pt\hbox{\ldots\ldots\ldots\ldots}}
\halign{\vrule width0pt height13pt depth9pt\hfill$a_#={}$&$a_#\x$\qquad\vrule\qquad&\hfill$b_#={}$&$b_#\x$\cr
1&0&1&0\cr
2&1&2&1\cr
3&2&3&2\cr
4&3&4&3\cr
\multispan4\vrule width0pt height15pt depth8pt\hfill On g\'en\'eralise par :\hfill\cr
{n+1}&n&{n+1}&n\cr}}




 
\item Compl\'eter les deux diagrammes ci-dessous :

\vskip\parskip
\begin{enumerate}
\item \hglue-3\leftmarginii\hfil\Baisse{1}{24}\hbox{\Image{2906_dessin-1.pdf}{1}{91}{24}}

\item \hglue-3\leftmarginii\hfil\Baisse{1}{24}\hbox{\Image{2906_dessin-3.pdf}{1}{91}{24}}
\end{enumerate}

\item Compl\'eter les pointill\'ees :

\hglue-3\leftmarginii\hfil\vbox{\offinterlineskip
\def\x{\ \raise-4pt\hbox{\ldots\ldots\ldots\ldots}}
\halign{\vrule width0pt height13pt depth9pt\hfill$a_#={}$&$a_#\x$\qquad\vrule\qquad&\hfill$b_#={}$&$b_#\x$\cr
1&0&1&0\cr
2&0&2&0\cr
3&0&3&0\cr
4&0&4&0\cr
\multispan4\vrule width0pt height15pt depth8pt\hfill On g\'en\'eralise par :\hfill\cr
n&0&n&0\cr}}


\end{enumerate}


\leavevmode\exercice


\begin{enumerate}
\item On consid\`ere la suite de nombres ci-dessous :

\hglue\leftmarginii$2$%
\quad\string;\quad%
$3$%
\quad\string;\quad%
$5$%
\quad\string;\quad%
$8$%
\quad\string;\quad%
$12$%
\quad\string;\quad%
$17$%
\quad\string;\quad%
$23$%
\quad\string;\quad%
$30$%

\vskip\parskip
\begin{enumerate}
\item Dans cette suite, quel est le terme qui succ\`ede \`a $12$?

\item Dans cette suite, quel est le terme qui pr\'ec\`ede $8$?

\item Dans cette suite quel est le rang du terme ayant $2$ pour valeur?

\item Dans cette suite quel est le rang du terme ayant $17$ pour valeur?
\end{enumerate}

\item De mani\`ere g\'en\'erale, on indique les termes d'une suite en utilisant en index la position du terme dans la suite {\it(on commence l'ind\'exation \`a $0$)} :

\hglue-\leftmarginii$u_0$%
\hfill\string;\hfill%
$u_1$%
\hfill\string;\hfill%
$u_2$%
\hfill\string;\hfill%
$u_3$%
\hfill\string;\hfill%
$\cdots$%
\hfill\string;\hfill%
$u_{n-1}$%
\hfill\string;\hfill%
$u_n$%
\hfill\string;\hfill%
$u_{n+1}$%

\vskip\parskip
\begin{enumerate}
\item Quel est le terme succ\'esseur de $u_2$?

\item Quel est le terme pr\'ed\'ecesseur de $u_4$?

\item Quel est le terme succ\'esseur de $u_n$?

\item Quel est le terme succ\'esseur de $u_{n+2}$?

\item Quel est le terme pr\'ed\'ecesseur de $u_n$?

\item Quel est le terme pr\'ed\'ecesseur de $u_{n+2}$?
\end{enumerate}
\end{enumerate}


\leavevmode\exercice


On consid\`ere les suites de nombres ci-dessous :


\vskip\parskip
\begin{enumerate}
\changelabel
\let\labeli=\labelii
\def\z #1{\hbox to1cm{\hss$#1$\hss}}
\def\x #1#2#3#4#5#6#7{\leavevmode\z{#1}%
\string;%
\z{#2}%
\string;%
\z{#3}%
\string;%
\z{#4}%
\string;%
\z{#5}%
\string;%
\z{#6}%
\string;%
\z{#7}\ldots}

\item \x{4}{7}{10}{13}{16}{19}{22}

\item \x{1}{-2}{4}{-8}{16}{-32}{64}

\item \x{2}{2}{3}{5}{8}{12}{17}

\item \x{0}{1}{4}{9}{16}{25}{36}

\item \x{1}{1}{2}{3}{5}{8}{13}

\item \x{1}{2}{1}{2}{1}{2}{1}
\end{enumerate}

Associer \`a chacune de cette suite une relation ci-dessous qui permet d'obtenir un terme en fonction de ses pr\'ed\'edecesseurs :

\questioni{1&u_{n}+u_{n+1}=u_{n+2}&
2&\dfrac{2}{u_n}=u_{n+1}\cr
3&u_n+n=u_{n+1}&
4&-2\times u_n=u_{n+1}\cr
5&u_n+3=u_{n+1}&
6&u_n=n^2\cr}

\leavevmode\exercice


On consid\`ere une suite $\big(u_n\big)$ dont on connait la valeur de ses cinq premiers termes :

\def\x{\hskip10pt}
\hglue\leftmargini$\Se u_0=0$%
\x\string;\x%
$\Se u_1=11$%
\x\string;\x%
$\Se u_2=20$%
\x\string;\x%
$\Se u_3=27$%
\x\string;\x%
$\Se u_4=32$%


Parmi les expressions de suites ci-dessous, lesquelles permettent d'obtenir ces m\^emes cinq premiers termes?

\vskip\parskip
\begin{enumerate}
\changelabel
\item $\left\{\ \vcenter{\hbox{$\begin{array}{rcl}
u_0&=&0\\
u_{n+1}&=&u_n+n+11\quad \hbox{ pour tout $n{\in}\mathbb{N}$}\\
\end{array}$}}\right.$

\item $\left\{\ \vcenter{\hbox{$\begin{array}{rcl}
u_0&=&0\\
u_{n+1}&=&-u_n+3n+11\quad \hbox{ pour tout $n{\in}\mathbb{N}$}\\
\end{array}$}}\right.$

\item $\left\{\ \vcenter{\hbox{$\begin{array}{rcl}
u_0&=&0\\
u_{n+1}&=&u_n-2n+11\quad \hbox{ pour tout $n{\in}\mathbb{N}$}\\
\end{array}$}}\right.$

\item $u_n=13{\cdot}n-2{\cdot}n^2$%
\qquad\labeli{e} $u_n=-n^2+12{\cdot}n$

\setcounter{enumi}5
\item $u_n=2{\cdot}n^2+9{\cdot}n$
\end{enumerate}

\leavevmode\exercice


\begin{enumerate}
\item On consid\`ere la suite $\big(u_n\big)$ arithm\'etique de premier terme $2$ et de raison $-3$. D\'eterminer les quatre premiers termes de la suite $\big(u_n\big)$.

\item On consid\`ere la suite $\big(v_n\big)$ g\'eom\'etrique de premier terme $54$ et de raison $\dfrac13$. D\'eterminer les quatre premiers termes de la suite $\big(v_n\big)$.
\end{enumerate}

\leavevmode\exercice


\begin{enumerate}
\item D\'eterminer les cinq premiers termes de la suite $\big(u_n\big)$ arithm\'etique de premier terme $2$ et de raison $3$.

\item D\'eterminer les cinq premiers termes de la suite $\big(v_n\big)$ arithm\'etique de premier terme $3$ et de raison $-\dfrac32$.
\end{enumerate}

\leavevmode\exercice


On consid\`ere les deux suites de nombres ci-dessous dont on donne les sept premiers termes :

\def\x #1#2#3#4#5#6#7{$#1$%
\quad\string;\quad%
$#2$%
\quad\string;\quad%
$#3$%
\quad\string;\quad%
$#4$%
\quad\string;\quad%
$#5$%
\quad\string;\quad%
$#6$%
\quad\string;\quad%
$#7$}
\questioni{a&\x{3}{5}{7}{10}{12}{14}{16}\cr
b&\x{6}{3,5}{1}{-1,5}{-4}{-6,5}{-9}\cr}

Pour chacune des questions, peut-on conjecturer que la suite est une suite arithm\'etique?\newline
Si oui, donner le premier terme et la raison. Si non, justifier votre rejet de cette affirmation.


\leavevmode\exercice


Soit $\big(u_n\big)$ une suite arithm\'etique de raison $r$. Compl\'eter les expressions suivantes :

\questioni{a&u_{12}=u_5+\ldots&
b&u_{57}=u_{38}+\ldots\cr
c&u_3=u_8+\ldots&
d&u_{23}=u_{38}+\ldots\cr}

\leavevmode\exercice


On consid\`ere la suite $\big(u_n\big)_{n\in\mathbb{N}}$ arithm\'etique de premier terme $3$ et de raison $-2$.

\vskip\parskip
\begin{enumerate}
\item D\'eterminer la valeur des termes $u_{12}$ et $u_{43}$.

\item D\'eterminer la valeur du rang $n$ r\'ealisant les \'egalit\'es :

\questionii{a&u_n=-21&
b&u_n=-57\cr}
\end{enumerate}

\leavevmode\exercice


On consid\`ere la suite $\big(u_n\big)$ d\'efinie par :\newline
\hglue\leftmargini$u_0=1$%
\quad\string;\quad%
$u_{n+1}=\dfrac{\big(n+2\big){\cdot}u_n+1}{n+1}$ pour tout $\Se n\in\mathbb{N}$

\vskip\parskip
\begin{enumerate}
\item D\'eterminer les quatre premiers termes de la suite $\big(u_n\big)$.

\item Conjecturer la nature de la suite $\big(u_n\big)$ en justifiant votre d\'emarche.
\end{enumerate}

\leavevmode\exercice


\begin{enumerate}
\item D\'eterminer les quatre premiers termes de la suite $\big(u_n\big)$ g\'eom\'etrique de premier terme $2$ et de raison $3$.

\item D\'eterminer les quatre premiers termes de la suite $\big(v_n\big)$ g\'eom\'etrique de premier terme $3$ et de raison $-\dfrac32$.
\end{enumerate}

\leavevmode\exercice


On consid\`ere les deux suites de nombres ci-dessous o\`u sont donn\'es les six premiers termes :

\def\x #1#2#3#4#5#6{#1%
\quad\string;\quad%
#2%
\quad\string;\quad%
#3%
\quad\string;\quad%
#4%
\quad\string;\quad%
#5%
\quad\string;\quad%
#6}
\questioni{a&\x{8}{4}{2}{1}{\dfrac12}{\dfrac14}\cr
b&\x{1}{3}{9}{18}{54}{162}\cr}

Pour chacune des questions, peut-on conjecturer que la suite est une suite g\'eom\'etrique?\newline
Si oui, pr\'eciser le premier terme et la raison. Sinon, justifier votre rejet de la conjecture.

\leavevmode\exercice


Soit $\big(v_n\big)$ une suite g\'eom\'etrique de raison $q$. Compl\'eter les expressions suivantes :

\questioni{a&u_{7}=u_3\times\ldots&
b&u_{25}=u_{11}\times\ldots\cr
c&u_3=u_8\times\ldots&
d&u_{15}=u_{23}\times\ldots\cr}

\leavevmode\exercice


On consid\`ere la suite $\big(u_n\big)_{n\in\mathbb{N}}$ g\'eom\'etrique de premier terme $\dfrac{2^4}{3}$ et de raison $\dfrac32$.

\vskip\parskip
\begin{enumerate}
\item D\'eterminer la valeur des termes $u_{11}$ et $u_{28}$.

\item Pour chaque question, d\'eterminer le rang $n$ r\'ealisant l'\'egalit\'e :

\questionii{a&u_n=\dfrac{3^8}{2^5}&
b&u_n=\dfrac{3^{19}}{2^{16}}\cr}
\end{enumerate}

\leavevmode\exercice


\begin{enumerate}
\item Soit $\big(u_n\big)_{n\in\mathbb{N}}$ une suite g\'eom\'etrique de premier terme $\Se u_0=\dfrac38$ et de raison $2$. D\'eterminer les six premiers termes de cette suite.

\item Soit $\big(v_n\big)_{n\in\mathbb{N}}$ une suite g\'eom\'etrique de raison $q$.

\vskip\parskip
\begin{enumerate}
\item Pour passer du terme $v_{11}$ au terme $v_{14}$, par combien de fois multiplie-t-on par la raison?

\item A partir des valeurs des deux termes suivants :\newline
\hglue\leftmarginii$v_{11}=\dfrac47$%
\quad\string;\quad$v_{14}=\dfrac{27}{14}$\newline
D\'eterminer la valeur du premier terme et de la raison de la suite $\big(v_n\big)$.
\end{enumerate}


\item Dans chacun des cas ci-dessous,  la suite $\big(w_n\big)_{n\in\mathbb{N}}$ est une suite g\'eom\'etrique, d\'eterminer son premier terme et sa raison :

\questionii{a&w_0=5\ \string;\ w_3=40&
b&w_3=\dfrac38\ \string;\ w_6=-\dfrac3{64}\cr
c&\multispan3$w_{124}=2\times10^{-4}$\ \string;\ $w_{128}=\dfrac18$\hfil\cr}
\end{enumerate}

\leavevmode\exercice


Soit $\big(u_n\big)_{n\in\mathbb{N}}$ une suite dont on connait la valeur des deux termes suivants :\newline
\hglue\leftmargini$u_6=36$%
\quad\string;\quad%
$u_{10}=\dfrac94$

Montrer qu'il existe au moins deux suites g\'eom\'etriques v\'erifiant ces conditions.

\vspace{1cm}
\leavevmode\exercice


Ci-dessous sont repr\'esent\'es les six premiers ``{\bfseries\slshape flocons de Helge Von Koch}'' repr\'esentant un des fractales les plus simples :

\def\y{\hskip0.35cm}
\def\x #1{\hbox{\Image{2928_flocon-#1.pdf}{1}{27}{12}}}
\leavevmode\hfil\x{0}\y\x{1}\y\x{2}

\leavevmode\hfil\x{3}\y\x{4}\y\x{5}

Pour passer d'une construction \`a la suivante, on r\'ealise la manipulation suivante sur chaque segment :

\leavevmode\hfil\hbox{\Image{2928_dessin2-2.pdf}{1}{81}{23}}

Chaque segment est partag\'e en trois parties \'egales {\it(\'etape 1)}. On construit un triangle \'equilat\'eral sur le segment du milieu {\it(\'etape 2)}. On efface le segment du milieu {\it(\'etape 3)}.

\vskip\parskip
\begin{enumerate}
\item \begin{enumerate}
\item Le passage de l'\'etape n$\null^o0$ \`a l'\'etape n$\null^o1$ fait apparaitre un triangle \'equilat\'eral. Surligner ce triangle en rouge.

\item Combien de segment comprend la figure de l'\'etape n$\null^o1$? Combien de triangles \'equilat\'eral apparaitront \`a l'\'etape n$\null^o2$? Surligner ces triangles en rouge.
\end{enumerate}

\item On note $\big(u_n\big)$ la suite num\'erique dont le terme de rang $n$ est le nombre de segments composant la figure \`a l'\'etape n$\null^{\text{i\`eme}}$ :

\vskip\parskip
\begin{enumerate}
\item Justifier par une phrase que la suite $\big(u_n\big)$ v\'erifie la relation :\newline
\hglue\leftmarginii$u_{n+1}=4{\cdot}u_n$

\item Exprimer le terme $u_n$ en fonction de son rang $n$.

\item Combien de segments comprend la figure de l'\'etape n$\null^o5$?
\end{enumerate}

\item On suppose que le segment $[AB]$ initial a pour longueur $1$. On note $\big(v_n\big)$ la suite num\'erique dont le terme de rang $n$ est la longueur de la ligne polygone formant la figure \`a l'\'etape n$\null^{\text{i\`eme}}$ :

\vskip\parskip
\begin{enumerate}
\item Justifier par une phrase que la suite $\big(v_n\big)$ v\'erifie la relation :\newline
\hglue\leftmarginii$v_{n+1}=\dfrac43{\cdot} v_n$

\item  Exprimer le terme $v_n$ en fonction de son rang $n$.
\end{enumerate}
\end{enumerate}


\leavevmode\exercice


On consid\`ere la suite $\big(u_n\big)$ d\'efinie par :\newline
\hglue\leftmargini$u_0=3$%
\quad\string;\quad%
$u_{n+1}=9\times2^n-u_n$

\vskip\parskip
\begin{enumerate}
\item D\'eterminer la valeur des quatre premiers termes de la suite $\big(u_n\big)$.

\item Conjecturer la nature de la suite $\big(u_n\big)$ en justifiant votre d\'emarche.
\end{enumerate}

\vspace{2cm}
\leavevmode\exercice


\begin{enumerate}
\item Justifier bri\`evement que les premiers termes de la suite $\big(u_n\big)$ pr\'esent\'es ci-dessous peuvent \^etre les termes d'une suite aritm\'etique dont on pr\'ecisera la raison :\newline
\hglue\leftmarginii$u_0=2$%
\quad\string;\quad%
$u_1=\dfrac{9}{2}$%
\quad\string;\quad%
$u_2=7$%
\quad\string;\quad%
$u_3=\dfrac{19}2$%

\item Justifier bri\`evement que les premiers termes de la suite $\big(v_n\big)$ pr\'esent\'es ci-dessous peuvent \^etre les termes d'une suite g\'eom\'etrique dont on pr\'ecisera la raison :\newline
\hglue\leftmarginii$v_0=24$%
\quad\string;\quad%
$v_1=6$%
\quad\string;\quad%
$v_2=\dfrac{3}{2}$%
\quad\string;\quad%
$v_3=\dfrac{3}{8}$%

\item Justifier bri\`evement que les premiers termes de la suite $\big(w_n\big)$ ne repr\'esentent ni les premiers termes d'une suite arithm\'etique, ni les premiers termes d'une suite g\'eom\'etrique \newline
\hglue\leftmarginii$w_0=1$%
\quad\string;\quad%
$w_1=2$%
\quad\string;\quad%
$w_2=4$%
\quad\string;\quad%
$w_3=16$%
\end{enumerate}


\leavevmode\exercice


\begin{enumerate}
\item Soit $\big(u_n\big)_{n\in\mathbb{N}}$ une suite arithm\'etique dont on connait deux termes :\newline
\hglue\leftmarginii$u_4=12$%
\quad\string;\quad%
$u_{22}=-24$

Donner, en justifiant votre d\'emarche, les \'el\'ements caract\'eristiques de cette suite.

\item Soit $\big(v_n\big)_{n\in\mathbb{N}}$ une suite g\'eom\'etrique dont on connait deux termes :\newline
\hglue\leftmarginii$v_4=8$%
\quad\string;\quad%
$v_{7}=\dfrac{64}{27}$

Donner, en justifiant votre d\'emarche, les \'el\'ements caract\'eristiques de cette suite.
\end{enumerate}

\leavevmode\exercice


\begin{enumerate}
\item On consid\`ere la suite $\big(u_n\big)_{n\in\mathbb{N}}$ arithm\'etique dont on connait les valeurs des deux termes suivants :\newline
\hglue\leftmarginii$u_{10}=5$%
\quad\string;\quad%
$u_{16}=14$

D\'eterminer le premier terme $u_0$ et la raison de cette suite.

\item On consid\`ere la suite $\big(v_n\big)_{n\in\mathbb{N}}$ g\'eom\'etrique dont on connait les valeurs des deux termes suivants :\newline
\hglue\leftmarginii$v_4=96$ %
\quad\string;\quad%
$v_7=\dfrac32$

D\'eterminer le premier terme $v_0$ et la raison de cette suite.
\end{enumerate}

\leavevmode\exercice


\begin{enumerate}
\item On consid\`ere la suite $\big(u_n\big)$ d\'efinie par :\newline
\hglue\leftmarginii$u_n=n^2+n+2$%
\quad pour tout entier $\Se n\in\mathbb{N}$

Etablir que la suite $\big(u_n\big)$ n'est pas une suite g\'eom\'etrique.

\item On consid\`ere la suite $\big(v_n\big)$ d\'efinie par :\newline
\hglue\leftmarginii$v_n=\dfrac{1}{n^2+2}$%
\quad pour tout entier $\Se n\in\mathbb{N}$

Etablir que la suite $\big(v_n\big)$ n'est pas une suite arithm\'etique.
\end{enumerate}

\leavevmode\exercice


On consid\`ere la suite $\big(u_n\big)_{n\in\mathbb{N}}$ d\'efinie par la relation :\newline
\hglue\leftmarginii$u_0=8$%
\quad\string;\quad%
$u_{n+1}=\dfrac12u_n-5$\quad pour tout entier naturel $n$.

\vskip\parskip
\begin{enumerate}
\item Soit $\big(v_n\big)_{n\in\mathbb{N}}$ la suite d\'efinie par :\newline
\hglue\leftmarginii$v_n=u_n+10$\quad pour tout $\Se n\in\mathbb{N}$

\vskip\parskip
\begin{enumerate}
\item Montrer que la suite $\big(v_n\big)$ v\'erifie la relation suivante pour tout entier naturel $n$ :\newline
\hglue\leftmarginii$v_{n+1}=\dfrac12{\cdot}v_n$

\item Donner la nature de la suite $(v_n)$ ainsi que ses \'el\'ements caract\'eristiques.

\item Donner la formule explicite donnant l'expression du terme $v_n$ en fonction de son rang $n$.
\end{enumerate}

\item D\'eduire des questions pr\'ec\'edentes, la formule explicite de la suite $\big(u_n\big)$.
\end{enumerate}

\leavevmode\exercice


Soit $\big(u_n\big)$ d\'efinie par son premier terme $u_0$ et, pour tout entier naturel $n$, par la relation :\newline
\hglue\leftmargini$u_{n+1}=a{\cdot}u_n+b$%
\hfill{\it($a$ et $b$ r\'eels non nuls tels que $\Se a\neq 1$)}

On pose, pour tout entier naturel $n$ :%
\quad$\Se v_n=u_n-\dfrac{b}{1-a}$

D\'emontrer que, la suite $\big(v_n\big)$ est g\'eom\'etrique de raison $a$.

\leavevmode\exercice


On consid\`ere la suite $\big(u_n\big)_{n\in\mathbb{N}}$ d\'efinie :\newline
\hglue\leftmargini$\Se u_0=4$%
\quad\string;\quad%
$\Se u_{n+1}=\dfrac{-u_n+6}{u_n-2}$\quad pour tout entier naturel $n$.

\vskip\parskip
\begin{enumerate}
\item D\'eterminer les trois premiers termes de la suite $\big(u_n\big)$.

\item Soit la suite $\big(v_n\big)_{n\in\mathbb{N}}$ d\'efinie par la relation :

\hglue\leftmarginii$v_n=\dfrac{u_n+2}{u_n-3}$\quad pour tout entier naturel $n$.

\begin{enumerate}
\item D\'eterminer les trois premiers termes de cette suite.

\item Montrer que :%
\quad$\Se\dfrac{v_{n+1}}{v_n}=-\dfrac14$

\item En d\'eduire la nature de la suite $\big(v_n\big)$ ainsi que la formule explicite d\'eterminant le terme de rang $n$ en fonction de $n$.
\end{enumerate}

\item \begin{enumerate}
\item D\'eterminer l'expression du terme $u_n$ en fonction du terme $v_n$.

\item En d\'eduire la formule explicite d\'efinissant les termes de $\big(u_n\big)$ en fonction de $n$.
\end{enumerate}
\end{enumerate}

\leavevmode\exercice


On consid\`ere les constructions suivantes :

\leavevmode\hfil\hbox{\Image{7244_dessin-1.pdf}{1}{68}{30}}


On note $\big(u_n\big)$ la suite num\'erique d\'efinie sur $\mathbb{N}^*$ o\`u $u_n$ repr\'esente le nombre d'allumettes n\'ecessaire \`a la construction de la $n^{\text{i\`eme}}$ \'etape.

Conjecturer une relation de r\'ecurrence entre un terme de la suite $\big(u_n\big)$ et de son pr\'ed\'ecesseur.


\end{multicols*}

\end{document}