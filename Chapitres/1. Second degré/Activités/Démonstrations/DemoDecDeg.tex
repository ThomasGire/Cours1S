% In your .tex file
% !TEX program = pdflatex

\documentclass[a4paper,11pt]{article}
\usepackage{pdflscape}
\usepackage[utf8]{inputenc}
\usepackage[T1]{fontenc}
%\usepackage{fourier} % math & rm
%\usepackage{amsthm,amsfonts,amsmath,amssymb,textcomp}
\usepackage{pst-all,pstricks-add,pst-eucl}
\everymath{\displaystyle}
\usepackage{fp,ifthen}
%\usepackage{color}
%\usepackage{graphicx}
\usepackage{setspace}
\usepackage{array}
\usepackage{tabularx}
\usepackage{supertabular}
\usepackage{hhline}
\usepackage{variations}
\usepackage{enumerate}
\usepackage{pifont}
\usepackage{framed}
\usepackage[fleqn]{amsmath}
\usepackage{amssymb}
\usepackage[framed]{ntheorem}
\usepackage{multicol}
\usepackage{kpfonts}
\usepackage{manfnt}
\usepackage{hyperref}



%\usepackage[hmargin=2.5cm, vmargin=2.5cm]{geometry}
\usepackage{vmargin}          % Pour fixer les marges du document
\setmarginsrb
{1.5cm} 	%marge gauche
{0.5cm} 	  %marge en haut
{1.5cm}     %marge droite
{0.5cm}   %marge en bas
{1cm} 	%hauteur de l'entête
{0.5cm}   %distance entre l'entête et le texte
{1cm} 	  %hauteur du pied de page
{0.5cm}     %distance entre le texte et le pied de page

\newcommand{\R}{\mathbb{R}}
\newcommand{\N}{\mathbb{N}}
%\newcommand{\D}{\mathbb{D}}
\newcommand{\Z}{\mathbb{Z}}
\newcommand{\Q}{\mathbb{Q}}
\newcommand{\C}{\mathbb{C}}
\newcommand{\e}{\text{e}}
\newcommand{\dx}{\text{d}x}
\newcommand{\vect}[1]{\mathchoice%
  {\overrightarrow{\displaystyle\mathstrut#1\,\,}}%
  {\overrightarrow{\textstyle\mathstrut#1\,\,}}%
  {\overrightarrow{\scriptstyle\mathstrut#1\,\,}}%
  {\overrightarrow{\scriptscriptstyle\mathstrut#1\,\,}}}
\newcommand\arraybslash{\let\\\@arraycr}
\renewcommand{\theenumi}{\textbf{\arabic{enumi}}}
\renewcommand{\labelenumi}{\textbf{\theenumi.}}
\renewcommand{\theenumii}{\textbf{\alph{enumii}}}
\renewcommand{\labelenumii}{\textbf{\theenumii.}}
\renewcommand{\and}{\wedge}

\theoremstyle{break}
\theorembodyfont{\upshape}
\newframedtheorem{Theo}{Théorème}
\newframedtheorem{Prop}{Propriété}
\newframedtheorem{Def}{Définition}

\newtheorem{Term}{Terminologie}
\newtheorem{Rq}{Remarque}
\newtheorem{Ex}{Exemple}
\newtheorem{exo}{Exercice}
\renewcommand{\theexo}{\empty{}} 

%\theorembodyfont{\small \sffamily}
%\newtheorem{sol}{solution}

\newenvironment{sol}% 
{\def\FrameCommand{\hspace{0.5cm} {\color{black} \vrule width 1pt} \hspace{-0.7cm}}%
  \framed {\advance\hsize-\width}
  \noindent \small \sffamily  %\underline{Solution :}%\\
}%
{\endframed}

\newrgbcolor{vert}{0 0.4 0}
\newrgbcolor{bistre}{1 .50 .30}
\setlength\tabcolsep{1mm}
\renewcommand\arraystretch{1.3}

\everymath{\displaystyle}
\hyphenpenalty 10000 %supprime toutes les césures
%\setcounter{secnumdepth}{0}
%\newcounter{saveenum}

\usepackage[frenchb]{babel}
\usepackage{fancyhdr,lastpage}
\usepackage{fancybox}

%\headheight 15.0 pt
\fancyhead[L]{}
\fancyhead[C]{Une démonstration pour le second degré.}
\fancyhead[R]{}
\fancyfoot[L]{{\scriptsize\textsl{Cité scolaire de Lorgues}}}
%\fancyfoot[C]{\scriptsize\thepage}
%\fancyfoot[C]{\scriptsize\thepage/\pageref{LastPage}}

\title{}
\author{}
\date{}

%\pagestyle{empty}
\pagestyle{fancy}
\usepackage[np]{numprint}

\renewcommand\arraystretch{1.8}

\newcounter{numero}
%\newcommand{\exo}{
%  \addtocounter{numero}{1}%
%  \textbf{\underline{Exercice \arabic{numero}:}}\quad}

\frenchbsetup{StandardEnumerateEnv=true}
\usepackage{etex}
\usepackage{tikz,tkz-tab}


\newframedtheorem{Dev}{Devoirs}
\renewcommand{\theDev}{\empty{}} 

\newcommand{\dm}{
  \textbf{\underline{Devoir à la maison:}}\quad \vspace{0.5cm}}
  
\begin{document}
  \setlength{\unitlength}{1mm}
  \setlength\parindent{0mm}
  
  
  %\exo
  ~
  \medskip
  
  \iffalse
   
   
   
    \item 
  \href{https://github.com/mathlorgues/math1sd1516/raw/master/images/82p41.png}
  {82 p 41}: Problème d'optimisation d'aire.
  
   \item Suite de 
    \href{https://github.com/mathlorgues/math1sd1516/blob/master/Chapitres/Chapitre1/presentation/secondDegre.pdf?raw=true}
    {l'exemple 13} 
     du cours 3. à 6.
     
  
  
      \item Interprétation géométrique des  
    \href{https://github.com/mathlorgues/math1sd1516/blob/master/Chapitres/Chapitre1/geogebra/forme%20canonique.ggb?raw=true}
    {paramètres de la forme canonique}.
   
 
    
    \item
  \href{https://github.com/mathlorgues/math1sd1516/raw/master/images/87p42.png}
  {Ex 87 p 42} : Une méthode astucieuse pour trouver une racine.
 
\item Variations et signes de la dérivée.



\fi

Soit $f(x)=ax^2+bx+c$ un trinôme du second degré avec un discriminant $\Delta >0$.

\begin{enumerate}
 \item Développer l'expression $(x-x_1)(x-x_2)$.
 \item Rappeler les formules permettant de calculer les racines $x_1$ et $x_2$ de $f(x)$.
 \item Calculer $x_1+x_2$ en fonction des coefficients $a,b$ et $c$.
 \item Calculer $x_1x_2$ en fonction des coefficients $a,b$ et $c$.
 \item Calculer l'expression $a(x-x_1)(x-x_2)$ en fonction des coefficients $a,b$ et $c$.
 \item Rappeler la forme canonique d'un trinôme.
 \item Rappeler les formules pour calculer $\alpha$ et $\beta$ en fonction de $a$, $b$ et $c$.
 \item Remplacer $\alpha$ et $\beta$ par ces expressions dans la forme canonique puis développer et réduire.
\end{enumerate}

\end{document}
 


