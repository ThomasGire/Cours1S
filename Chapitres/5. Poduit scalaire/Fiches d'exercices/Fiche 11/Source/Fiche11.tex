\documentclass{book}
\usepackage[utf8]{inputenc}
\usepackage[T1]{fontenc}
\usepackage{amsmath,amssymb}
\usepackage{array,color,multirow,slashbox,multicol}
\usepackage{mathrsfs}
\usepackage{eurosym}
\usepackage{stmaryrd} %Pour le symbole parallele \\sslash
\usepackage{yhmath}   %pour dessiner les arcs \wideparen
%\usepackage{multicol}
\usepackage[portrait,nofootskip]{chingatome}








\begin{document}

\fontsize{10}{12}\fontfamily{cmr}\selectfont\titre[18pt]{Trigonométrie et angles orientés.}
\fontsize{10}{12}\fontfamily{cmr}\selectfont\begin{multicols*}{2}


%%%%%%%%%%%%%%%%%
\leavevmode\exercice


\dimen0=4.5cm\dimen1\hsize\advance\dimen1-\dimen0%
\begin{minipage}[t]{\dimen0}
On consid\`ere le triangle $ABC$ rectangle en $B$ repr\'esent\'e ci-dessous :

\vskip0.2cm
\begin{enumerate}
\item D\'eterminer la longueur du segment $[BC]$ arrondie au millim\`etre pr\`es.

\item En d\'eduire la mesure de l'angle $\widehat{CDB}$ arrondie au degr\'e pr\`es.
\end{enumerate}

\end{minipage}
\begin{minipage}[t]{\dimen0}
\leavevmode\hfill\Baisse{1}{33}\hbox{\Image{6038_dessin-1.pdf}{1}{42}{37}}
\end{minipage}


\leavevmode\exercice


\dimen0=5cm\dimen1\hsize\advance\dimen1-\dimen0
\begin{minipage}[t]{\dimen1}
\parskip0.2cm
On consid\`ere un triangle $ABC$ rectangle en $C$. On note :

\leavevmode$\alpha=\widehat{CAB}$%
\quad\string;\quad%
$\beta=\widehat{ABC}$
\end{minipage}
\begin{minipage}[t]{\dimen0}
\leavevmode\hfill\Baisse{1}{18}\hbox{\Image{2182_triangle-1.pdf}{1}{48}{22}}
\end{minipage}

\vskip\parskip
\begin{enumerate}
\item Justifier que les angles $\widehat{CAB}$ et $\widehat{CBA}$ sont deux angles compl\'ementaires.

\item \begin{enumerate}
\item A l'aide des longueurs des c\^ot\'es du triangle $ABC$, exprimer les valeurs de $\cos\alpha$ et $\sin\beta$.

\item En d\'eduire l'\'egalit\'e :\quad$\cos\alpha=\sin\Big(\dfrac{\pi}2{-}\alpha\Big)$
\end{enumerate}


\item \begin{enumerate}
\item A l'aide des longueurs des c\^ot\'es du triangle $ABC$, exprimer les valeurs de $\tan\alpha$ et $\tan\beta$

\item En d\'eduire l'\'egalit\'e :%
\quad$\tan\Big(\dfrac{\pi}2{-}\alpha\Big)=\dfrac1{\tan\alpha}$
\end{enumerate}

\item Etablir l'\'egalit\'e :%
\quad$\big(\cos\alpha\big)^2+\big(\sin\alpha\big)^2=1$
\end{enumerate}

\leavevmode\exercice


Ci-dessous sont repr\'esent\'ees deux droites gradu\'ees repr\'esentant les mesures d'un angle en radian sur l'intervalle $\Se\big[0\,;\,2\pi\big]$.

\def\x #1{\hglue-\leftmargini\hfil\Baisse{1}{10}\hbox{\Image{2721_dessin#1-1.pdf}{1}{82}{27}}}
\begin{enumerate}
\item \x{2}

\item\x{3}
\end{enumerate}

Compl\'eter la graduation du bas {\it(repr\'esentant une mesure d'angle en radian)}, puis compl\'eter les valeurs du haut repr\'esentant la conversion correspondante en degr\'e :

\leavevmode\exercice


On a repr\'esent\'e ci-dessous les neufs premiers polygones r\'eguliers inscrit dans le cercle trigonom\'etrique. 

\leavevmode\hfil\hbox{\Image{2188_dessin-1.pdf}{1}{90}{30}}

\leavevmode\hfil\hbox{\Image{2188_dessin-2.pdf}{1}{90}{30}}

\leavevmode\hfil\hbox{\Image{2188_dessin-3.pdf}{1}{90}{30}}

\vskip\parskip
\begin{enumerate}
\item Donner la mesure, en radians, de l'angle au centre s\'eparant deux sommets cons\'ecutifs de chacun de ces polygones :

\item Nommer chacun de ces polygones.
\end{enumerate}

\leavevmode\exercice


On consid\`ere la figure ci-dessous, o\`u $\mathscr{C}$ est un demi-cercle de centre $O$ et admettant le segment $[IA]$ pour diam\`etre :

\leavevmode\hfil\hbox{\Image{7550_dessin-1.pdf}{1}{50}{26}}

Les autres points pr\'esents sur cette figure appartiennent au demi-cercle $\mathscr{C}$ et v\'erifient les propri\'et\'es suivantes :

\vskip\parskip
\begin{itemize}
\item Le triangle $OAD$ est un triangle \'equilat\'eral ;

\item Le triangle $OCM$ est un triangle rectangle isoc\`ele en $C$ ;

\item Le triangle $AEO$ est un triangle rectangle en $O$ ;

\item La demi-droite $[OB)$ est la bissectrice de l'angle $\widehat{DOA}$ ;

\item Le point $F$ est le sym\'etrique du point $D$ par rapport \`a la droite $(EO)$ ;

\item Les mesures des angles $\widehat{AOG}$ et $\widehat{AOC}$ sont suppl\'ementaires ;

\item Le point $H$ est le point d'intersection du demi-cercle $\mathscr{C}$ avec la droite parall\`ele \`a la droite $(AI)$ et passant par le point $B$.
\end{itemize}

Donner la mesure exacte des angles ci-dessous en radian :

\questioni[0.75cm]{a&\widehat{AOB}&
b&\widehat{AOC}&
c&\widehat{AOD}&
d&\widehat{AOE}\cr
e&\widehat{AOF}&
f&\widehat{AOG}&
g&\widehat{AOH}&
h&\widehat{AOI}\cr}

\vspace{1cm}
\leavevmode\exercice


\dimen0\hsize\advance\dimen0-5.2cm
\begin{minipage}[t]{\dimen0}
Dans le plan muni d'un rep\`ere orthonomal $\Se(O\,;\,I\,;\,J)$, on consid\`ere le cercle de centre $O$ et de rayon 1 appel\'e {\bf cercle trigonom\'etrique}.

Tout point $M$ d\'efinit un angle g\'eom\'etrique $\widehat{IOM}$.

Le sens de parcours du cercle trigonom\'etrique permet de caract\'eriser tout point du cercle par son angle g\'eom\'etrique :
\end{minipage}
\begin{minipage}[t]{5.2cm}
\Baisse{1}{49}\hbox{\Image{810_triangle-1.pdf}{1}{52}{49}}
\end{minipage}
\begin{itemize}
\item l'angle est positif si l'arc $\wideparen{IM}$ est orient\'e dans le sens inverse des aiguilles d'une montre.

\item l'angle est n\'egatif si l'arc $\wideparen{IM}$ est orient\'e dans le sens des aiguilles d'une montre.
\end{itemize}

Dans la repr\'esentation ci-dessus :
\begin{itemize}
\item On a :\quad $\Se\coord{\Vec{OI}}{\Vec{OM}}=+\dfrac{\pi}{3}\ \rad$\newline
Dans le cercle trigonom\'etrique, on note $M\Big({+}\dfrac{\pi}3\Big)$.

\item On a :\quad$\Se\coord{\Vec{OI}}{\Vec{OM'}}=-\dfrac{\pi}3\ \rad$\newline
Dans le cercle trigonom\'etrique, on note $M'\Big({-}\dfrac{\pi}3\Big)$.
\end{itemize}

\vspace{-0.3cm}
\vskip\parskip
\begin{enumerate}
\item \dimen0=55mm\dimen1\linewidth\advance\dimen1-\dimen0%
\begin{minipage}[t]{\dimen1}
Dans la figure ci-dessous, les points $A_i$ d\'efinissent un angle orient\'e $\Se\coord{\Vec{OI}}{\Vec{OA_i}}$ ayant une mesure ``{\sl remarquable}''. Pour chacun des points, indiquer la mesure de l'angle associ\'erajouter le signe permettant de rep\'erer chaque
\end{minipage}
\begin{minipage}[t]{\dimen0}
\hfill\Baisse{1}{49}\hbox{\Image{810_dessin1-1.pdf}{1}{54}{54}}
\end{minipage}
 point marqu\'e du cercle trigonom\'etrique :


\item Dans le cercle trigonom\'etrique ci-dessus, placer sur cette figure les points $N$, $P$, $Q$, $R$, $S$, $T$ r\'ealisant les mesures suivantes :

\hglue-\leftmargini\questionii[0.5cm]{a&\coord{\Vec{OI}}{\Vec{ON}}=-\dfrac{\pi}{4}\ \rad&
b&\coord{\Vec{OI}}{\Vec{OP}}=\dfrac{5\pi}{6}\ \rad\cr
c&\coord{\Vec{OI}}{\Vec{OQ}}=-\dfrac{2\pi}{3}\ \rad&
d&\coord{\Vec{OK}}{\Vec{OR}}=-\dfrac{\pi}{4}\ \rad\cr
e&\coord{\Vec{OK}}{\Vec{OS}}=\dfrac{\pi}{6}\ \rad&
f&\coord{\Vec{OJ}}{\Vec{OT}}=-\dfrac{\pi}{4}\ \rad\cr}
\end{enumerate}

\leavevmode\exercice

\vspace{-0.5cm}

\dimen0=5cm\dimen1\hsize\advance\dimen1-\dimen0%
\begin{minipage}[t]{\dimen1}
Dans le plan muni d'un rep\`ere $\Se\big(O\,;\,I\,;\,J\big)$, on consid\`ere le cercle trigonom\'etrique repr\'esent\'e ci-dessous sur lequel est plac\'e plusieurs points :


Les points $M$, $N$, $P$ v\'erifient les mesures suivantes :\newline
$\widehat{IOM}=30^o$%
\ \string;\ %
$\widehat{ION}=45^o$\newline
$\widehat{IOP}=60^o$%
\end{minipage}
\begin{minipage}[t]{\dimen0}
\Baisse{1}{42}\hbox{\Image{5464_dessin-1.pdf}{1}{47}{46}}
\end{minipage}

\vskip\parskip
\begin{enumerate}
\item Donner la mesure des angles rep\'erant les points $M$, $N$, $P$ en radians.

\item Les points $M'$, $N'$, $P'$ sont respectivement les sym\'etriques des points $M$, $N$, $P$ par rapport \`a l'axe $(OI)$ :

\vskip\parskip
\begin{enumerate}
\item Que peut-on dire de $\coord{\Vec{OI}}{\Vec{OM}}$ et $\coord{\Vec{OI}}{\Vec{OM'}}$?

\item Donner la mesure en radians des angles suivants :\newline
\hglue\leftmarginii$\coord{\Vec{OI}}{\Vec{OM'}}$%
\quad\string;\quad%
$\coord{\Vec{OI}}{\Vec{ON'}}$%
\quad\string;\quad%
$\coord{\Vec{OI}}{\Vec{OP'}}$%
\end{enumerate}

\item Les points $M''$, $N''$ et $P''$ sont respectivement les sym\'etriques des points $M$, $N$, $P$ par rapport \`a l'axe $(OJ)$ :

\vskip\parskip
\begin{enumerate}
\item Que peut-on dire de $\coord{\Vec{OI}}{\Vec{OM}}$ et $\coord{\Vec{OI}}{\Vec{OM''}}$?

\item Donner la mesure en radians des angles suivants :\newline
\hglue\leftmarginii$\coord{\Vec{OI}}{\Vec{OM''}}$%
\quad\string;\quad%
$\coord{\Vec{OI}}{\Vec{ON''}}$%
\quad\string;\quad%
$\coord{\Vec{OI}}{\Vec{OP''}}$%
\end{enumerate}

\item Les points $M'''$, $N'''$ et $P'''$ sont respectivement les sym\'etriques des points $M$, $N$, $P$ par rapport \`a l'axe $(OJ)$ :

\vskip\parskip
\begin{enumerate}
\item Quelle relation alg\'ebrique v\'erifie les deux angles :\newline
\hglue\leftmarginii$\coord{\Vec{OI}}{\Vec{OM}}$%
\quad\string;\quad%
$\coord{\Vec{OI}}{\Vec{OM'''}}$

\item Donner la mesure en radians des angles suivants :\newline
\hglue\leftmarginii$\coord{\Vec{OI}}{\Vec{OM'''}}$%
\quad\string;\quad%
$\coord{\Vec{OI}}{\Vec{ON''}}$%
\quad\string;\quad%
$\coord{\Vec{OI}}{\Vec{OP''}}$%
\end{enumerate}
\end{enumerate}

\leavevmode\exercice


\dimen0=5cm\dimen1\hsize\advance\dimen1-\dimen0
\begin{minipage}[t]{\dimen1}
On consid\`ere le quadrilat\`ere $ABCD$ repr\'esent\'e ci-dessous qui est constitu\'e de deux triangles $ABC$ et $ACD$ respectivement \'equilat\'eral et isoc\`ele rectangle en $D$.
\end{minipage}
\begin{minipage}[t]{\dimen0}
\hfill\Baisse{1}{33}\hbox{\Image{5465_dessin-1.pdf}{1}{48}{36}}
\end{minipage}

A l'aide des points de cette figure et pour chaque question, donner un angle orient\'e r\'ealisant les mesures suivantes :

\questioni[0.5cm]{a&\dfrac{\pi}3\ \rad&
b&-\dfrac{\pi}4\ \rad&
c&-\dfrac{\pi}6\ \rad&
d&\dfrac{7\pi}{12}\ \rad\cr}

\leavevmode\exercice


\dimen0=5cm\dimen1\hsize\advance\dimen1-\dimen0
\begin{minipage}[t]{\dimen1}
On consid\`ere le cercle trigonom\'etrique $\mathscr{C}$ ci-dessous o\`u est inscrit un dod\'ecagone {\it(polygone r\'egulier \`a 12 c\^ot\'es)}
\vskip0.2cm
\begin{enumerate}
\item D\'eterminer la mesure de l'angle $\coord{\Vec{OI}}{\Vec{OA}}$

\item Placer sur le cercle $\mathscr{C}$ les points $M$, $N$, $P$ tels que :
\end{enumerate}
\end{minipage}
\begin{minipage}[t]{\dimen0}
\leavevmode\hfill\Baisse{1}{44}\hbox{\Image{2153_dessin-1.pdf}{1}{47}{47}}
\end{minipage}

\questionii[0.5cm]{a&\coord{\Vec{OI}}{\Vec{OM}}=\dfrac{2\pi}3\ \rad&
b&\coord{\Vec{OJ}}{\Vec{ON}}=-\dfrac{\pi}6\ \rad\cr
c&\coord{\Vec{OA}}{\Vec{OP}}=-\dfrac{\pi}2\ \rad&
c&\coord{\Vec{OQ}}{\Vec{OJ}}=-\dfrac{5\pi}{6}\ \rad\cr}


\end{multicols*}

\end{document}