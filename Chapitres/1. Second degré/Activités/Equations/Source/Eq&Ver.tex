\documentclass{book}
\usepackage[utf8]{inputenc}
%\usepackage[latin1]{inputenc}
\usepackage[T1]{fontenc}
\usepackage{amsmath,amssymb}
\usepackage{array,color,multirow,slashbox,multicol}
\usepackage{mathrsfs}
\usepackage{eurosym}
\usepackage{stmaryrd} %Pour le symbole parallele \\sslash
\usepackage{yhmath}   %pour dessiner les arcs \wideparen
%\usepackage{multicol}
\usepackage[portrait,nofootskip]{chingatome}








\begin{document}

\titre[18pt]{Equations}

\fontsize{10}{12}\fontfamily{cmr}\selectfont
%\begin{multicols*}{2}

\begin{enumerate}
 \item Résoudre dans $\mathbb{R}$, l'équation $\sqrt{x+1}=\sqrt{-x-3}$.
 
 Le plan est muni d'un repère orthonormé $(O, \vec{i},\vec{j})$
 \item Trouver une équation pour le cercle de centre $A(1;2)$ et de rayon 2.
 \item Trouver une équation pour le cercle de centre $A(x_A,y_A)$ et de rayon $R$.
 \item Identifier le lieu géométrique formé par les points $M(x;y)$ vérifiant l'équation
 $x^2+y^2-x-3y+5=0$.
 \item Identifier le lieu géométrique formé par les points $M(x;y)$ vérifiant l'équation
 $x^2+y^2-x-3y-5=0$.
 \item Identifier le lieu géométrique formé par les points $M(x;y)$ vérifiant l'équation
 $4x^2+4y^2+8xy=1$.
 \item Soient A et B deux points distincts. Identifier le lieu des points $M$ tels que l'angle $\widehat{\mathrm{AMB}}= 90$\degre.
 \item Montrer qu'un parallélogramme est un rectangle si et seulement si ses diagonales sont de même longueur.
\end{enumerate}




\end{document}