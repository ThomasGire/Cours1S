\documentclass{book}
%\usepackage[latin1]{inputenc}
\usepackage[utf8]{inputenc}
\usepackage[T1]{fontenc}
\usepackage{amsmath,amssymb}
\usepackage{array,color,multirow,slashbox,multicol}
\usepackage{mathrsfs}
\usepackage{eurosym}
\usepackage{stmaryrd} %Pour le symbole parallele \\sslash
\usepackage{yhmath}   %pour dessiner les arcs \wideparen
%\usepackage{multicol}
\usepackage[portrait,nofootskip]{chingatome}








\begin{document}

\fontsize{10}{12}\fontfamily{cmr}\selectfont\titre[18pt]{Tangentes et nombre dérivé - Fiche d'exercices 5}
\fontsize{10}{12}\fontfamily{cmr}\selectfont\begin{multicols*}{2}


%%%%%%%%%%%%%%%%%
\leavevmode\exercice


Etablir les \'egalit\'es suivantes :

\questioni{a&\dfrac{x}{x+1}-\dfrac{2x-3}{x-1}=\dfrac{3-x^2}{x^2-1}\cr
b&\dfrac1x+\dfrac1{x+1}+\dfrac1{(x+1)^2}=\dfrac{2x^2+4x+1}{x{\cdot}\big(x+1\big)^2}\cr
c&\dfrac{\dfrac1{x+1}+\dfrac2{2x-1}}{x}=\dfrac{4x+1}{x{\cdot}(x+1){\cdot}(2x-1)}\cr}

\leavevmode\exercice


\begin{enumerate}
\item Soit $f$ la fonction d\'efinie par la relation :\newline
\hglue\leftmarginii$f(x)
=x^2-3{\cdot}x+2$

D\'eterminer, pour $\Se h\in\mathbb{R}$, un expression simplifi\'ee de $f(1{+}h)$.

\item Soit $g$ la fonction d\'efinie par la relation :\newline
\hglue\leftmarginii$g(x)
=\dfrac{\sqrt{2x-1\mathstrut}-3}{x-5}$

Etablir , pour tout $\Se h\in\mathbb{R}^*$, l'\'egalit\'e :\newline
\hglue\leftmarginii$g(h{+}5)
=\dfrac{2}{\sqrt{2h+9\mathstrut}+3}$
\end{enumerate}

\leavevmode\exercice


Ci-dessous est repr\'esent\'ee, dans le rep\`ere $\Se(O\,;\,I\,;\,J)$, la courbe $\mathscr{C}_f$ et quatre de ses tangentes :

\leavevmode\hfil\hbox{\Image{2809_graph-1.pdf}{1}{91}{61}}

\vskip\parskip
\begin{enumerate}
\item La droite $(T_1)$ s'appelle :

\hglue1\leftmarginii``{\sl La tangente \`a la courbe $\mathscr{C}_f$ au point d'abscisse $-1,5$''}

Nommez de m\^eme les trois autres droites.

\item D\'eterminer l'\'equation r\'eduite de chacune de ces quatres tangentes.
\end{enumerate}



\leavevmode\exercice


D\'eterminer les coefficients directeurs des quatre droites repr\'esent\'ees ci-dessous :

\leavevmode\hfil\hbox{\Image{2291_graph-1.pdf}{1}{92}{70}}

\leavevmode\exercice


Tracer la courbe repr\'esentative d'une fonction passant par tous les points indiqu\'ees et respectant en chacun d'eux la tangente repr\'esent\'ee :

\leavevmode\hfil\hbox{\Image{2313_graph-1.pdf}{1}{92}{64}}

\leavevmode\exercice


On consid\`ere le plan muni d'un rep\`ere $\Se\big(O\,;\,I\,;\,J\big)$ dans lequel sont repr\'esent\'ees :
\begin{itemize}
\item La courbe $\mathscr{C}_f$ repr\'esentative de la fonction $f$ ;

\item Les cordes $(d_1)$, $(d_2)$, $(d_3)$ et $(d_4)$ \`a la courbe $\mathscr{C}_f$.
\end{itemize}

\leavevmode\hfil\hbox{\Image{5945_graph-1.pdf}{1}{91}{78}}

\vskip\parskip
\begin{enumerate}
\item D\'eterminer les coefficients directeurs des quatre cordes \`a la courbe $\mathscr{C}_f$.

\item \begin{enumerate}
\item Tracer, \`a l'aide d'un r\`egle, la tangente \`a la courbe $\mathscr{C}_f$ au point de coordonn\'ees $\Se\coord{5}{-1}$.

\item Donner une valeur approch\'ee du coefficient directeur de la tangente $(T)$.
\end{enumerate}
\end{enumerate}

\leavevmode\exercice


%y=1.5x^3-5x^2+x+2
\dimen0=4.5cm\dimen1\hsize\advance\dimen1-\dimen0
\begin{minipage}[t]{\dimen1}
Dans le rep\`ere $\Se(O\,;\,I\,;\,J)$ orthonorm\'e ci-contre est donn\'ee la courbe $\mathscr{C}_f$ repr\'esentative de la fonction $f$.

On consid\`ere les points $A$, $B$, $C$ de la courbe $\mathscr{C}_f$ d'abscisses respectives $0$, $1$ et $2$

\vskip0.2cm
\begin{enumerate}
\item Placer les points $A$, $B$ et $C$ et par lecture graphique, donner leur coordonn\'ee.

\item Calculer le taux de variation de la fonction $f$ :

\vskip\parskip\begin{enumerate}
\item entre $0$ et $2$

\item entre $1$ et $2$
\end{enumerate}
\end{enumerate}
\end{minipage}
\begin{minipage}[t]{\dimen0}
\leavevmode\hfill\Baisse{1}{76}\hbox{\Image{2292_graph-1.pdf}{1}{43}{77}}
\end{minipage}

\leavevmode\exercice


On donne ci-dessous la courbe $\mathscr{C}$ repr\'esentative d'une fonction dans un rep\`ere $\big(0\,;\,I\,;\,J\big)$ :

\leavevmode\hfil{\Image{6878_graph-1.pdf}{1}{92}{75}}

\vskip\parskip
\begin{enumerate}
\item \begin{enumerate}
\item Tracer la tangente $(d)$ \`a la courbe $\mathscr{C}$ au point d'abscisse $4$.

\item Donner le coefficient directeur de la droite $(d)$.
\end{enumerate}

\item \begin{enumerate}
\item Tracer la tangente $(\Delta)$ \`a la courbe $\mathscr{C}$ au point d'abscisse $1$.

\item Donner le coefficient directeur de la droite $(\Delta)$.
\end{enumerate}
\end{enumerate}




\leavevmode\exercice


On consid\`ere trois fonctions $f$, $g$ et $h$ trois fonctions d\'efinies sur $\mathbb{R}_+^*$ par les relations :\newline
\leavevmode$f(x)=\dfrac{x+2}{x^2+3}$%
\hfill\string;\hfill%
$g(x)=\dfrac{2{\cdot}x-1}{\sqrt{x}}$
\hfill\string;\hfill%
$h(x)=\dfrac{2{\cdot}x^2+x}{5{\cdot}x}$

On donne ci-dessous un tableau de valeurs pour chacune des fonctions :

\def\x{\vrule width0pt height12pt depth4pt}
\def\y{\vrule width0pt height16pt depth7pt}

\def\esp{\hglue8pt}
\leavevmode\hfil\vbox{\offinterlineskip
\halign{\vrule\ $#$\ \hfil\vrule&&\esp\hfil$#$\hfil\esp\vrule\cr
\noalign{\hrule}
\x x&1&0,1&0,01&0,001&0,0001\cr
\noalign{\hrule}
\y f(x)&\dfrac34&\dfrac{2,1}{3,01}&\dfrac{2,01}{3,000\,1}&\dfrac{2,000\,1}{3,000\,001}&\dfrac{2,0000\,1}{3,000\,000\,01}\cr
\noalign{\hrule}}}

\def\esp{\hglue19.8pt}
\leavevmode\hfil\vbox{\offinterlineskip
\halign{\vrule\ $#$\ \hfil\vrule&&\esp\hfil$#$\hfil\esp\vrule\cr
\noalign{\hrule}
\x x&0,01&0,000\,1&0,000\,001\cr
\noalign{\hrule}
\y g(x)&\dfrac{-0,98}{0,1}&\dfrac{-0,999\,8}{0,01}&\dfrac{-0,999\,998}{0,00\,1}\cr
\noalign{\hrule}}}

\def\esp{\hglue8pt}
\leavevmode\hfil\vbox{\offinterlineskip
\halign{\vrule\ $#$\ \hfil\vrule&&\esp\hfil$#$\hfil\esp\vrule\cr
\noalign{\hrule}
\x x&1&0,1&0,01&0,001&0,000\,1\cr
\noalign{\hrule}
\y h(x)&\dfrac{3}{5}&\dfrac{0,12}{0,5}&\dfrac{0,010\,2}{0,05}&\dfrac{0,001\,002}{0,005}&\dfrac{0,000\,100\,02}{0,000\,5}\cr
\noalign{\hrule}}}

Remarquez que, dans chaque tableau, les valeurs de $x$ ``{\sl progressent lentement}'' vers $0$.

\vskip\parskip
\begin{enumerate}
\item \begin{enumerate}
\item Pour chaque tableau et \`a l'aide de la calculatrice, observer la progression de des valeurs approch\'ees de ces quotients.

\item Dans chaque cas, faire une conjecture sur la valeur limite de ces images lorsque :\newline
\hglue\leftmarginii``{\sl $x$ tend vers $0$ par des valeurs sup\'erieures \`a 0}''\newline
Pour la fonction $f$, cette valeur se note :\newline
\hglue\leftmarginii$\lim_{x\mapsto 0\atop x>0} f(x)$ ou $\lim_{x\mapsto 0^+} f(x)$
\end{enumerate}

\item A l'aide de votre calculatrice, tracer les courbes repr\'esentatives de ces fonctions et observer la courbe au ``{\sl voisinage}'' de l'axe des ordonn\'ees.
\end{enumerate}

\leavevmode\exercice


D\'eterminer les limites suivantes :

\def\x{\lim_{h\mapsto0}\ }
\questioni{a&\x h-2&
b&\x\dfrac{2h^2+h}h\cr
c&\x\dfrac{5h^3+2h^2}h&
d&\x\dfrac{h+1}{h+2}\cr
e&\x\dfrac{2h^2+h}{3h}&
f&\x\dfrac{h^3+2h^2}{2h}\cr}

\leavevmode\exercice


\dimen0=4.5cm\dimen1\hsize\advance\dimen1-\dimen0
\begin{minipage}[t]{\dimen1}
Dans un rep\`ere $\Se\big(O\,;\,I\,;\,J\big)$, on consid\`ere la courbe repr\'esentative de la fonction $f$ d\'efinie sur $\mathbb{R}$ par la relation :\newline
\hglue\leftmargini$f(x)=x^2-4x+2$

\vskip0.2cm
Au cours de cet exercice, nous allons d\'eterminer l'\'equation r\'eduite de la tangente \`a la courbe $\mathscr{C}_f$ au point $\Se A\coord{1}{-1}$.
\end{minipage}
\begin{minipage}[t]{\dimen0}
\leavevmode\hfill\Baisse{1}{39}\hbox{\Image{6059_graph-1.pdf}{1}{43}{43}}
\end{minipage}

\vskip\parskip
\begin{enumerate}
\item \begin{enumerate}
\item Etablir l'\'egalit\'e suivante :\newline
\hglue\leftmarginii$\dfrac{f(x)-f(1)}{x-1}=x-3$\qquad pour tout $\Se x\in\mathbb{R}\backslash\{1\}$

\item En d\'eduire le coefficient directeur de la corde \`a la courbe $\mathscr{C}_f$ passant par les points $A$ et $\Se B\coord{2}{-2}$. V\'erifier graphiquement votre r\'eponse.
\end{enumerate}

\item On consid\`ere les deux fonctions $u$ et $v$ d\'efinie sur $\mathbb{R}\backslash\{1\}$ par les relations :\newline
\hglue\leftmarginii$u(x)=\dfrac{f(x)-f(1)}{x-1}$%
\quad\string;\quad%
$v(x)=x-3$

\vskip\parskip
\begin{enumerate}
\item Voici deux tableaux de valeurs de $u$ et de $v$ :

\hglue-3\leftmarginii\hfil\vbox{\vbox{\offinterlineskip
\def\h{\vrule width0pt height12pt}
\def\b{\vrule width0pt depth4pt}
\halign{\vrule\vrule width0pt height12pt depth4pt\ $#$\ \hfill\vrule&&\hbox to2.1cm{\hfill$#$\hfill}\vrule\cr
\noalign{\hrule}
x&1,1&1,01&1,001&1,000\,1\cr
\noalign{\hrule}
u(x)&\dfrac{-0,19}{0,1}&\dfrac{-0,0199\h}{0,01\b}&\dfrac{-0,001999}{0,001}&\dfrac{-0,00019999}{0,0001}\cr
\noalign{\hrule}}}
\vskip\parskip
\vbox{\offinterlineskip
\def\h{\vrule width0pt height12pt}
\def\b{\vrule width0pt depth4pt}
\halign{\vrule\vrule width0pt height12pt depth4pt\ $#$\ \hfill\vrule&&\hbox to2.1cm{\hfill$#$\hfill}\vrule\cr
\noalign{\hrule}
x&1,1&1,01&1,001&1,000\,1\cr
\noalign{\hrule}
v(x)&-1,9&-1,99&-1,999&-1,9999\cr
\noalign{\hrule}}}}

Que peut-on dire de la valeur de $u(x)$ lorsque le nombre $x$ se rapproche de la valeur $1$?

\item Tracer dans le rep\`ere la droite $(d)$ d'\'equation :\newline
\hglue\leftmarginii$(d):\ y=-2x+1$
\end{enumerate}
\end{enumerate}



\leavevmode\exercice


On consid\`ere la fonction $f$ d\'efinie sur $\mathbb{R}$ par la relation :\newline
\hglue\leftmargini$f{:x}\longmapsto \dfrac12{\cdot}x^2-2{\cdot}x+1$

\vskip\parskip
\begin{enumerate}
\item \begin{enumerate}
\item Montrer que, pour tout $\Se h\in\mathbb{R}$, on a :\newline
\hglue\leftmarginii$f(4{+}h)-f(4)=\dfrac12{\cdot}h^2+2{\cdot}h$

\item D\'eterminer la valeur de la limite : $\lim_{h\mapsto 0}\dfrac{f(4{+}h)-f(4)}h$
\end{enumerate}

\item Dans le plan muni d'un rep\`ere $\Se(O\,;\,I\,;\,J)$ orthonormal, on donne la courbe $\mathscr{C}_f$ repr\'esentative de la fonction $f$ :

\hglue-\leftmargini\hfil{\Image{2810_graph-1.pdf}{1}{92}{56}}

\begin{enumerate}
\item Tracer dans le rep\`ere ci-dessous, la droite $(d)$ admettant pour \'equation r\'eduite :%
\quad$y=2x-7$

\item Justifier que la droite $(d)$ est une tangente \`a la courbe $\mathscr{C}_f$ dont on pr\'ecisera le point de contact.
\end{enumerate}
\end{enumerate}




\leavevmode\exercice


\begin{enumerate}
\item Le nombre d\'eriv\'e de la fonction carr\'e en $2$ :

\vskip\parskip
\begin{enumerate}
\item Pour $\Se x\neq2$, \'etablir l'\'egalit\'e suivante :%
\quad$\Se \dfrac{x^2-2^2}{x-2}=x+2$

\item Soit $f$ la fonction carr\'ee.\newline
En d\'eduire la valeur de la limite :%
\quad$\lim_{x\mapsto2} \dfrac{f(x)-f(2)}{x-2}$
\end{enumerate}


\item Le nombre d\'eriv\'ee de la fonction inverse en $3$ :

\vskip\parskip
\begin{enumerate}
\item Pour $\Se x\neq3$, \'eetablir l'\'egalit\'e suivante :%
\quad$\Se\dfrac{\ \dfrac1{x}-\dfrac13\ }{x-3}=-\dfrac1{3x}$

\item Soit $g$ la fonction inverse.\newline
En d\'eduire la limite suivante :%
\quad$\lim_{x\mapsto3} \dfrac{g(x)-g(3)}{x-3}$
\end{enumerate}


\item Le nombre d\'eriv\'ee de la fonction racine carr\'ee en $3$ :

\vskip\parskip
\begin{enumerate}
\item Pour $\Se x\neq2$, \'etablir que :%
\quad$\Se\dfrac{\sqrt{x\mathstrut}-\sqrt{2\mathstrut}}{x-2}=\dfrac1{\sqrt{x\mathstrut}+2}$

\item Soit $h$ la fonction racine carr\'ee.\newline
En d\'eduire la valeur de la limite :%
\quad$\lim_{x\mapsto2} \dfrac{h(x)-h(2)}{x-2}$
\end{enumerate}
\end{enumerate}




\leavevmode\exercice


D\'eterminer l'expression des fonctions d\'eriv\'ees des fonctions polynomiales suivantes :

\questioni[0.5cm]{1&f{:x}\longmapsto -3x+2&
2&g{:x}\longmapsto 4x^2-4\cr
3&h{:x}\longmapsto 2x^2+3x&
4&j{:x}\longmapsto 5x^3-2x^2\cr
5&k{:x}\longmapsto -2x^2+2x&
6&\ell{:x}\longmapsto (3x+11)(4-x)\cr}

\leavevmode\exercice


On consid\`ere la fonction $f$ dont l'image de $x$ est d\'efinie par la relation :

\hglue\leftmargini$f(x)=\dfrac18x^3-\dfrac12x^2-\dfrac12x+1$

\leavevmode\hfil\hbox{\Image{2347_graph-1.pdf}{1}{93}{53}}

On note $\mathscr{C}_f$ la courbe repr\'esentative de la fonction $f$ dans un rep\`ere orthonorm\'e.

\vskip\parskip
\begin{enumerate}
\item Donner l'expression de la fonction $f'$ d\'eriv\'ee de la fonction $f$.

\item On consid\`ere la tangente $(T)$ \`a la courbe $\mathscr{C}_f$ au point d'abscisse $2$.

\vskip\parskip\begin{enumerate}
\item Donner la valeur du coefficient directeur de $(T)$.

\item D\'eterminer l'\'equation r\'eduite de la tangente $(T)$.

\item Dans le rep\`ere ci-dessous, tracer la tangente $(T)$.
\end{enumerate}

\item On consid\`ere la droite $(d)$ admettant l'\'equation r\'eduite :\newline
\hglue\leftmarginii$(d):\ y=-x+1$\newline
D\'eterminer les coordonn\'ees des points d'intersection de la droite $(d)$ et de la courbe $\mathscr{C}_f$.
\end{enumerate}

\leavevmode\exercice


D\'eterminer les fonctions d\'eriv\'ees associ\'ees aux fonctions suivantes :

\def\x {{:x}\longmapsto}
\questioni{a&f\x x-2\sqrt{x\mathstrut} &
b&g\x 2\times\dfrac1x\cr
c&h\x \dfrac{-5}x+\sqrt{x\mathstrut}&
d&k\x x^2-\dfrac1x\cr}

\leavevmode\exercice


D\'eterminer l'expression de la d\'eriv\'ee de chacune des fonctions ci-dessous :

\questioni{1&f{:x}\longmapsto x-\dfrac1x&
2&g{:x}\longmapsto 2{\cdot}\sqrt{x\mathstrut}\cr
3&h{:x}\longmapsto \dfrac{3}x-2\sqrt{x\mathstrut}&
4&j{:x}\longmapsto 2x^3+\dfrac2x\cr}

On pr\'esentera l'expression des d\'eriv\'ees sous la forme d'un quotient.

\leavevmode\exercice


D\'eterminer l'expression des d\'eriv\'ees des fonctions suivantes :

\hglue-\leftmargini\questioni[0.75cm]{a&f(x)=3x^2&
b&g(x)=\dfrac1{12}x^6&
c&h(x)=4\sqrt{x\mathstrut}\cr
d&j(x)=\dfrac{\sqrt{x\mathstrut}}2&
e&k(x)=\dfrac1{2x}&
f&l(x)=-\dfrac2x\cr}

\leavevmode\exercice[*]


D\'eterminer l'expression de la d\'eriv\'ee de chacune des fonctions suivantes :

\questioni{a&f{:x}\longmapsto 3x^2+5x&
b&g{:x}\longmapsto \dfrac3x+2\sqrt{x}\cr
c&h{:x}\longmapsto 5x^3-\dfrac3{x}&
d&j{:x}\longmapsto \dfrac{8x^3-2x^2}{x}\cr}

Les d\'eriv\'ees des fonctions $g$ et $h$ seront pr\'esent\'ees sous forme de quotient.

\leavevmode\exercice[*]


On consid\`ere la fonction $f$ d\'efinie sur $\Se\big]0\,;\,+\infty\big[$ par la relation :\newline
\hglue\leftmargini$f(x)=x+\dfrac2x-2$

La courbe $\mathscr{C}_f$ repr\'esentative de la fonction $f$ est donn\'ee ci-dessous dans un rep\`ere $\Se\big(O\,;\,I\,;\,J\big)$ orthonorm\'e :

\leavevmode\hfil\hbox{\Image{5216_graph-1.pdf}{1}{92}{65}}

\vskip\parskip
\begin{enumerate}
\item Montrer que la fonction $f$ admet pour d\'eriv\'ee la fonction $f'$ dont l'expression est donn\'ee par :\newline
\hglue\leftmarginii$f'(x)=\dfrac{x^2-2}{x^2}$

\item On souhaite d\'eterminer l'\'equation r\'eduite de la tangente $(T)$ \`a la courbe $\mathscr{C}_f$ au point d'abscisse $2$.

\vskip\parskip
\begin{enumerate}
\item Donner le coefficient directeur de la tangente $(T)$. Justifier votre d\'emarche.

\item D\'eterminer l'\'equation r\'eduite de la tangente $(T)$.

\item Tracer la droite $(T)$ dans le rep\`ere ci-dessus.
\end{enumerate}

\item On consid\`ere la droite $(d)$ d'\'equation r\'eduite :\newline
\hglue\leftmarginii$(d):\ y=\dfrac12{\cdot}x$

\vskip\parskip
\begin{enumerate}
\item Sur $\Se\big]0\,;\,+\infty\big[$, \'etudier le signe de l'expression :\newline
\hglue\leftmarginii$f(x)-\dfrac12{\cdot}x$

\item En d\'eduire la position relative de la courbe $\mathscr{C}_f$ et de la droite $(d)$.
\end{enumerate}
\end{enumerate}

\leavevmode\exercice


On consid\`ere la fonction $f$ d\'efinie sur $\mathbb{R}_+$ par la relation :\newline
\hglue\leftmargini$f(x)=-x+2\sqrt{x\mathstrut}$

Dans le rep\`ere $\Se\big(O\,;\,I\,;\,J\big)$ ci-dessous, est donn\'ee la courbe $\mathscr{C}_f$ repr\'esentative de la fonction $f$.

\leavevmode\hfil\hbox{\Image{5220_graph-1.pdf}{1}{93}{53}}

\vskip\parskip
\begin{enumerate}
\item \begin{enumerate}
\item Montrer que la fonction $f$ admet pour d\'eriv\'ee, sur $\mathbb{R}^*_+$, la fonction $f'$ dont l'expression est donn\'ee par :\newline
\hglue\leftmarginii$f'(x)=\dfrac{1-\sqrt{x\mathstrut}}{\sqrt{x\mathstrut}}$

\item D\'eterminer la valeur des nombres d\'eriv\'ees de la fonction $f$ en $\dfrac14$ et en $4$.
\end{enumerate}

\item On note $(d)$ et $(\Delta)$ les tangentes \`a la courbe $\mathscr{C}_f$ aux points d'abscisse respectifs $\dfrac14$ et $4$.

\vskip\parskip
\begin{enumerate}
\item D\'eterminer les \'equations r\'eduites des tangentes $(d)$ et $(\Delta)$.

\item Montrer que les deux droites $(d)$ et $(\Delta)$ s'interceptent au point de coordonn\'ees $\coord{1}{\dfrac32}$.

\item Tracer sur le graphique les droites $(d)$ et $(\Delta)$.
\end{enumerate}
\end{enumerate}

\leavevmode\exercice


Soit $f$ d\'efinie sur $\mathbb{R}$ par la relation :%
\quad$\Se f(x)=4x^2-4x-3$

\vskip\parskip
\begin{enumerate}
\item Calculer le nombre d\'eriv\'e de la fonction $f$ en 2.

\item D\'eterminer l'\'equation de la tangente \`a la courbe $\mathscr{C}_f$ au point d'abscisse 2.
\end{enumerate}


\leavevmode\exercice


\begin{enumerate}
\item Donner l'\'equation r\'eduite de la tangente \`a la courbe de la fonction carr\'ee au point d'abscisse $-2$.

\item Donner l'\'equation r\'eduite de la tangente \`a la  courbe repr\'esentative de la fonction inverse au point d'abscisse 3.
\end{enumerate}



\leavevmode\exercice


On souhaite d\'eterminer les expressions des d\'eriv\'ees des fonctions suivantes :\newline
\hglue\leftmargini\vtop{\openup6pt
\halign{$#$\hfill&&\quad\string;\quad$#$\hfill\cr
f{:x}\longmapsto (3x^2+3x)(2x+2)&
g{:x}\longmapsto (2x^2+1)\sqrt{x\mathstrut}\cr
h{:x}\longmapsto \dfrac1x{\cdot}(3-x^2)&
j{:x}\longmapsto \dfrac2x{\cdot}\sqrt{x\mathstrut}\cr}}

\vskip\parskip
\begin{enumerate}
\item L'expression de chacune de ces fonctions est donn\'ee sous la forme d'un produit $u{\cdot}v$. Compl\'eter le tableau ci-dessous afin d'identifier les deux facteurs de ce produit et leur d\'eriv\'ee respective.

\hglue-\leftmargini\hfil\vbox{\offinterlineskip
\halign{\vrule\vrule width0pt height15pt depth8pt\hbox to1.5cm{\hfil$#$\hfil}\vrule&&\hbox to1.9cm{\hfill$#$\hfill}\vrule\cr
\omit\hfil\vrule&\multispan4\hrulefill\cr
\omit\hfil\vrule depth6pt height13pt&u(x)&v(x)&u'(x)&v'(x)\cr
\noalign{\hrule}
f(x)&&&&\cr
\noalign{\hrule}
g(x)&&&&\cr
\noalign{\hrule}
h(x)&&&&\cr
\noalign{\hrule}
j(x)&&&&\cr
\noalign{\hrule}}}

\item En utilisant la formule de d\'erivation d'un produit :\newline
\hglue\leftmarginii$\big(u{\cdot}v\big)'=u'{\cdot}v+u{\cdot}v'$\newline
Etablir que ces fonctions admettent pour d\'eriv\'ee les fonctions ci-dessous :\newline
\hglue\leftmarginii\vtop{\openup6pt
\halign{$#$\hfill&&\quad\string;\quad$#$\hfill\cr
f'{:x}\longmapsto 18x^2+24x+6&
g'{:x}\longmapsto \dfrac{10x^2+1}{2\sqrt{x\mathstrut}}\cr
h'{:x}\longmapsto \dfrac{-x^2-3}{x^2}&
j'{:x}\longmapsto -\dfrac1{x{\cdot}\sqrt{x\mathstrut}}\cr}}
\end{enumerate}


\leavevmode\exercice


On consid\`ere la fonction $f$ d\'efinie sur $\mathbb{R}_+$ par la relation :\newline
\hglue\leftmargini$f(x)=(x-4)\sqrt{x\mathstrut}$

La courbe $\mathscr{C}_f$ repr\'esentative de la fonction $f$ est donn\'ee dans le rep\`ere $\Se\big(O\,;\,I\,;\,J\big)$ orthonorm\'e :

\leavevmode\hfil\hbox{\Image{5227_graph-1.pdf}{1}{91}{78}}

\vskip\parskip
\begin{enumerate}
\item \begin{enumerate}
\item D\'eterminer l'expression de la fonction $f'$ d\'eriv\'ee de la fonction $f$.

\item D\'eterminer l'image et le nombre d\'eriv\'e de la fonction $f$ en $4$.

\item D\'eterminer l'\'equation r\'eduite de la tangente $(T_1)$ \`a la courbe $\mathscr{C}_f$ au point d'abscisse $4$.

\item Tracer la tangente $(T_1)$.
\end{enumerate}

\item \begin{enumerate}
\item D\'eterminer l'\'equation r\'eduite de la tangente $(T_2)$ \`a la courbe $\mathscr{C}_f$ au point d'abscisse $1$.

\item Tracer la tangente $(T_2)$.
\end{enumerate}
\end{enumerate}

\leavevmode\exercice


On souhaite d\'eterminer les expressions des d\'eriv\'ees des fonctions suivantes :\newline
\hglue\leftmargini\vtop{\openup6pt
\halign{$#$\hfill&&\quad\string;\quad$#$\hfill\cr
f{:x}\longmapsto \dfrac{3-2x}{x+1}&
g{:x}\longmapsto \dfrac{x^2+4x-1}{2x-1}\cr
h{:x}\longmapsto \dfrac{3}{2-x}&
j{:x}\longmapsto \dfrac{\sqrt{x\mathstrut}}{x+1}\cr}}

\vskip\parskip
\begin{enumerate}
\item L'expression de chacune de ces fonctions est donn\'ee sous la forme d'un produit $\dfrac uv$. Compl\'eter le tableau ci-dessous afin d'identifier le num\'erateur et le d\'enominateur de ce quotient et leurs d\'eriv\'ees respectives.

\hglue-\leftmargini\hfil\vbox{\offinterlineskip
\halign{\vrule\vrule width0pt height15pt depth8pt\hbox to1.5cm{\hfil$#$\hfil}\vrule&&\hbox to1.9cm{\hfill$#$\hfill}\vrule\cr
\omit\hfil\vrule&\multispan4\hrulefill\cr
\omit\hfil\vrule depth6pt height13pt&u(x)&v(x)&u'(x)&v'(x)\cr
\noalign{\hrule}
f(x)&&&&\cr
\noalign{\hrule}
g(x)&&&&\cr
\noalign{\hrule}
h(x)&&&&\cr
\noalign{\hrule}
j(x)&&&&\cr
\noalign{\hrule}}}

\item En utilisant la formule de d\'erivation d'un produit :\newline
\hglue\leftmarginii$\Big(\dfrac uv\Big)'=\dfrac{u'{\cdot}v-u{\cdot}v'}{v^2}$\newline
Etablir que ces fonctions admettent pour d\'eriv\'ee les fonctions ci-dessous :\newline
\hglue\leftmarginii\vtop{\openup6pt
\halign{$#$\hfill&&\quad\string;\quad$#$\hfill\cr
f'{:x}\longmapsto -\dfrac{5}{(x+1)^2}&
g'{:x}\longmapsto \dfrac{2x^2-2x-2}{(2x-1)^2}\cr
h'{:x}\longmapsto \dfrac{3}{(x-2)^2}&
j'{:x}\longmapsto \dfrac{1-x}{2(x+1)^2{\cdot}\sqrt{x\mathstrut}}\cr}}
\end{enumerate}


\leavevmode\exercice


D\'eterminer l'expression des fonctions d\'eriv\'ees de chacune des fonctions ci-dessous :
 
\questioni[0.75cm]{a&f{:x}\longmapsto\dfrac{2-2x}{5x+1} &
b&g{:x}\longmapsto (3x-2)(2x^2+1)\cr
c&h{:x}\longmapsto\dfrac1{3x+1}&
d&j{:x}\longmapsto(2x^2+3x)\cdot\sqrt{x\mathstrut}\cr}

\leavevmode\exercice


On consid\`ere les deux fonctions $f$ et $g$ d\'efinies par les relations :\newline
\hglue\leftmargini$f(x)=\big(x^2-3x\big){\cdot}\sqrt{x}$%
\quad\string;\quad%
$g(x)=\dfrac{3-2x}{x^2-3x-1}$

D\'eterminer les expressions des fonctions d\'eriv\'ees $f'$ et $g'$. {\it(On donnera l'expression de la fonction $f'$ sous la forme d'un quotient simplifi\'e)}.

\leavevmode\exercice


\begin{enumerate}
\item On consid\`ere les deux fonctions $f$ et $g$ par :\newline
\hglue\leftmarginii$f(x)=(2x+1)(3x^2-x+1)$%
\quad\string;\quad%
$g(x)=\dfrac{2x+5}{1-4x}$

D\'eterminer l'expression de la fonction d\'eriv\'ee de chacune de ces deux fonctions.

\item On consid\`ere la fonction $h$ dont l'image de $x$ est d\'efini par la relation :

\hglue\leftmarginii$h(x)=\dfrac{x^2-2x+1}{x^2-5x+6}$

\vskip\parskip
\begin{enumerate}
\item D\'eterminer l'ensemble de d\'efinition de la fonction $h$.

\item Montrer que le nombre de d\'eriv\'ee de $h$ en $x$ s'exprime par :

\hglue\leftmarginii$h'(x)=-\dfrac{3x^2-10x+7}{(x^2-5x+6)^2}$
\end{enumerate}
\end{enumerate}

\leavevmode\exercice


Le tableau ci-dessous vous pr\'esente, pour chaque ligne, l'expression de l'image de $x$ par une fonction et l'expression du nombre d\'eriv\'e en $x$ de cette fonction. V\'erifier l'exactitude de l'expression du nombre d\'eriv\'ee en $x$ :

\leavevmode\hfil\vbox{\offinterlineskip
\def\hh{\vrule width0pt height0pt}
\def\bb{\vrule width0pt depth2pt}
\def\h{\vrule width0pt height12pt}
\def\b{\vrule width0pt depth6pt}
\halign{\vrule\h\b\hfil\ $#$\ \hfil\vrule &&\quad\hfill$#$\quad\hfill\vrule\cr
\noalign{\hrule}
\text{Fonction}&\hfill\text{\centredeuxlignes{\hh Image de}{\bb $x$}}&\hfill\text{\centredeuxlignes{Nombre d\'eriv\'e\hh }{en $x$\bb}}\cr
\noalign{\hrule}
f&x^3-5x^2+x-3&3x^2-10x+1\cr
\noalign{\hrule}
g&\dfrac{2x-1}{x^2+x}&-\dfrac{2x^2-2x-1\h }{x^2{\cdot}(x+1)^2\b}\cr
\noalign{\hrule}
h&(x^2-3){\cdot}\sqrt{\mathstrut x}&\dfrac{5x^2-3\h }{2{\cdot}\sqrt{\mathstrut x}\b}\cr
\noalign{\hrule}
j&\dfrac{3x-2\h }{2-x\b }&\dfrac4{(x-2)^2}\cr
\noalign{\hrule}}}

\end{multicols*}

\end{document}