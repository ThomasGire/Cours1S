\documentclass{book}
\usepackage[utf8]{inputenc}
\usepackage[T1]{fontenc}
\usepackage{amsmath,amssymb}
\usepackage{array,color,multirow,slashbox,multicol}
\usepackage{mathrsfs}
\usepackage{eurosym}
\usepackage{stmaryrd} %Pour le symbole parallele \\sslash
\usepackage{yhmath}   %pour dessiner les arcs \wideparen
%\usepackage{multicol}
\usepackage[portrait,nofootskip]{chingatome}








\begin{document}

\fontsize{10}{12}\fontfamily{cmr}\selectfont\titre[18pt]{2nd degré - Fiche d'exercices 1}

\fontsize{10}{12}\fontfamily{cmr}\selectfont\begin{multicols*}{2}


%%%%%%%%%%%%%%%%%
\leavevmode\exercice


On consid\`ere la fonction $f$ d\'efinie sur $\mathbb{R}$ dont l'image d'un nombre $x$ est d\'efinie par la relation alg\'ebrique :

\hglue\leftmargini$f(x)=4x^2+4x-3$

\vskip\parskip
\begin{enumerate}
\item \begin{enumerate}
\item D\'emontrer que pour tout $\Se x\in\mathbb{R}$, on a :

\hglue\leftmarginii$f(x)=(2x-1)(2x+3)$

\item D\'emontrer que pour tout $\Se x\in\mathbb{R}$, on a :

\hglue\leftmarginii$f(x)=(2x+1)^2-4$
\end{enumerate}

\item Pour chacune des questions suivantes, utiliser la forme la plus adapt\'ee :

\vskip\parskip
\begin{enumerate}
\item D\'eterminer les ant\'ec\'edents de 0 par la fonction $f$.

\item Sachant que le carr\'e d'un nombre est toujours positif ou nul, \'etablir que la fonction $f $ est minor\'ee par $-4$.

\item D\'eterminer le signe de la fonction $f$ sur $\mathbb{R}$.

\item R\'esoudre l'in\'equation :%
\quad$\Se f(x)\geq 5$.
\end{enumerate}
\end{enumerate}


\leavevmode\exercice


On consid\`ere la fonction $f$ d\'efinie sur $\mathbb{R}$ par la relation :

\hglue\leftmargini$f(x)=6x^2-9x-6$

\vskip\parskip
\begin{enumerate}
\item \begin{enumerate}
\item Montrer que l'expression de $f(x)$ peut s'\'ecrire :

\hglue\leftmarginii$f(x)=6\bigg[\Big(x-\dfrac34\Big)^2-\dfrac{25}{16}\bigg]$

\item En d\'eduire que la fonction $f$ est minor\'ee par $-\dfrac{75}{8}$.

\item Soit $a$ et $b$ deux nombres r\'eels, \'etablir l'implication suivante :\newline
\hglue\leftmarginii$\Se a<b<\dfrac34
\quad\Longrightarrow\quad
f(a)>f(b)$

{\it(Cette implication \'etablit que,  sur $\Se\big]-\infty\,;\,\frac34\big]$, la fonction $f$ est d\'ecroissante.)}
\end{enumerate}


\item \begin{enumerate}
\item D\'eduire de la question {\fboxsep1.5pt\labeli{1}\,\labelii{a}} la factorisation suivante :

\hglue\leftmarginii$f(x)=6\Big(x+\dfrac12\Big)\big(x-2\big)$

\item Donner les ant\'ec\'edents de 0 par la fonction $f$.

\item D\'eterminer la partie de $\mathbb{R}$ sur laquelle la fonction $f$ est strictement positive.
\end{enumerate}
\end{enumerate}

\leavevmode\exercice


Donner la forme canonique de chacun des trin\^omes du second degr\'e ci-dessous :

\questioni{a&2x^2+8x-6&
b&3x^2+3x+6\cr
c&9x^2+18x+27&
d&5x^2+10x+2\cr
e&2x^2+5x-4&
f&\sqrt{2\mathstrut}x^2-3x+1\cr}

\leavevmode\exercice


On d\'efinit la fonction $f$ sur $\mathbb{R}$ dont l'image de $\Se x\in\mathbb{R}$ est d\'efinie par la relation :

\hglue\leftmargini$f(x)=8x^2-2x+1$

\vskip\parskip
\begin{enumerate}
\item Donner la forme canonique de la fonction $f$.

\item Etablir que la fonction $f$ est minor\'ee par $\dfrac7{8}$.

\item \begin{enumerate}
\item Etablir, sans justification, le tableau de variation de la fonction $f$.

\item En d\'eduire que la fonction $f$ n'admet pas de z\'ero sur $\mathbb{R}$.
\end{enumerate}
\end{enumerate}

\leavevmode\exercice


R\'esoudre les \'equations suivantes :

\questioni{a&x^2+4x-5=0&
b&2x^2-13x+15=0\cr
c&x^2+x+1=0&
d&x^2+5x+2=0\cr
e&-3x^2+6x-2=0&
f&3x^2-2x+1=0\cr}

\leavevmode\exercice


R\'esoudre les \'equations suivantes :

\questioni{a&3x^2-5x+6=0&
b&3x^2-24x+48=0\cr
c&\multispan3$x(x-2)(x+1)=(x-2)(-7-3x)$\hfil\cr}

\leavevmode\exercice


D\'eterminer les racines, sous forme simplifi\'ee, des polyn\^omes suivants :

\questioni{a&2x^2-3x-9&
b&5x^2-8x+5\cr
c&2x^2-8x+8&
d&x^2+2x-1\cr}

\leavevmode\exercice


On consid\`ere la figure ci-dessous :

\leavevmode\hfil\hbox{\Image{5706_dessin-1.pdf}{1}{81}{65}}

Quel doit-\^etre la valeur de $x$ pour que la figure gris\'ee ait une aire de $25\,cm^2$?

\leavevmode\exercice


Factoriser les expressions suivantes :

\questioni{a&5x^2-x-4&
b&-2x^2-3x-1\cr
c&-x^2+2x-1&
d&4x^2+x-3\cr
e&4x^2+4x-5&
f&x^2-2x-4\cr}

\leavevmode\exercice


\begin{enumerate}
\item Factoriser les expressions suivantes :

\questionii{a&2x^2-3x-2&
b&12x^2-12x+3\cr}

\item Simplifier la fraction rationnelle suivante :\newline
\hglue\leftmarginii$\dfrac{x^2-x-2}{2x^2-3x-2}$
\end{enumerate}

\leavevmode\exercice


Simplifiez l'expression des fractions rationnelles ci-dessous  :

\questioni{a&\dfrac{3x-1}{3x^2+2x-1}&
b&\dfrac{6x^2-5x+1}{1-4x^2}\cr
c&\multispan3$\dfrac{3x^2-6x-6}{x^2-\big(\sqrt{3\mathstrut}+2\big)x+\big(\sqrt{3\mathstrut}+1\big)}$\hfil\cr}

\leavevmode\exercice


On consid\`ere la fonction polynome $P$ de degr\'e 3 d\'efinie par :\newline
\hglue\leftmargini$P(x)=3x^3+x^2-8x+4$

\vskip\parskip
\begin{enumerate}
\item D\'eterminer les valeurs de $a$, $b$, $c$ tel que :\newline
\hglue\leftmarginii$P(x)=(x+2)\big(a{\cdot}x^2+b{\cdot} x+c\big)$

\item En d\'eduire l'ensemble des z\'eros du polyn\^ome $P$.
\end{enumerate}

\leavevmode\exercice


Etablir le tableau de signes des polynomes du second degr\'e suivant :

\questioni{a&x^2+3x+4&
b&-8x^2+32x+32\cr
c&4x^2+3x-10&
d&-5x^2-3x-1\cr
e&4x^2-16x+16&
f&2x^2+11x+5\cr}

\leavevmode\exercice


\begin{enumerate}
\item \begin{enumerate}
\item Etablir que le polyn\^ome $\se P(x)=2x^2-x+1$ est strictement positif sur $\mathbb{R}$.

\item En d\'eduire le signe du polyn\^ome :

\hglue\leftmarginii$Q(x)=(2x^2-x+1)^2+3{\cdot}(2x^2-x+1)+1$
\end{enumerate}

\item Justifier que l'\'equation ci-dessous n'admet aucune solution :\newline
\hglue\leftmarginii$4x^4-4x^3+11x^2-5x+5=0$
\end{enumerate}

\leavevmode\exercice


On consid\`ere le polyn\^ome du troisi\`eme degr\'e :

\hglue\leftmargini$\mathcal{P}=3x^3+5x^2-5x+1$

On sait que le polyn\^ome $P$ admet une factorisation de la forme :\newline
\hglue\leftmargini$\mathcal{P}=\big(3x-1\big)\big(a{\cdot}x^2+b{\cdot} x+c\big)$

\vskip\parskip
\begin{enumerate}
\item D\'eterminer les valeurs de $a$, $b$, $c$ v\'erifiant cette factorisation.

\item En d\'eduire l'ensemble des racines du polyn\^ome $\mathcal{P}$.

\item Dresser le tableau de signe de $\mathcal{P}$.
\end{enumerate}

\end{multicols*}

\end{document}