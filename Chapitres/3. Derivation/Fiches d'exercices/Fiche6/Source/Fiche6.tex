\documentclass{book}
%\usepackage[latin1]{inputenc}
\usepackage[utf8]{inputenc}
\usepackage[T1]{fontenc}
\usepackage{amsmath,amssymb}
\usepackage{array,color,multirow,slashbox,multicol}
\usepackage{mathrsfs}
\usepackage{eurosym}
\usepackage{stmaryrd} %Pour le symbole parallele \\sslash
\usepackage{yhmath}   %pour dessiner les arcs \wideparen
%\usepackage{multicol}
\usepackage[portrait,nofootskip]{chingatome}








\begin{document}

\fontsize{10}{12}\fontfamily{cmr}\selectfont\titre[18pt]{Variations et problèmes - Fiche d'exercices 6}
\fontsize{10}{12}\fontfamily{cmr}\selectfont\begin{multicols*}{2}


%%%%%%%%%%%%%%%%%
\leavevmode\exercice


Voici le tableau de variation d'un fonction $f$ d\'efinie sur $\Se\big[-4\,;\,4\big[$.

\leavevmode\hfil\hbox{\Image{6061_tableauVariation2-1.pdf}{1}{81}{26}}

D\'eterminer le signe du nombre d\'eriv\'ee de la fonction $f$ en $1$.

\leavevmode\exercice


On consid\`ere une fonction $f$ dont on donne ci-dessous le tableau de signe de sa fonction d\'eriv\'ee :

\def\x{1.2cm}
\def\y{2cm}
\leavevmode\hfil\vbox{\offinterlineskip
\def\ligne #1#2#3{\omit\hbox to\y{\hbox to0pt{\,$#1$\hss}\hbox to0pt{\hss$#2$\hss}\hfill\hbox to0pt{\hss$#3$\,}}}
\def\cel #1#2{\ifx\vrule#1\,\vrule\else\hbox to0pt{\hss$#1$\hss}\fi\hfil$#2$\hfil\vrule}
\halign{\vrule\hbox to\x{\ $#$\hfill}\vrule width0pt height14pt depth5pt\vrule&&\cel#\cr
\noalign{\hrule}
x &\ligne{-5}{}{}&\ligne{}{-2}{}&\ligne{}{1}{4}\vrule\cr
\noalign{\hrule}
f'(x) &{}{-}&{0}{+}&{0}{-}\cr
\noalign{\hrule}}}

On consid\`ere la tangente $(T)$ \`a la courbe $\mathscr{C}_f$ au point d'abscisse $2$.

Quel est le sens de variation de la tangente $(T)$?

\leavevmode\exercice


On consid\`ere la fonction $f$ d\'efinie et d\'erivable sur $\mathbb{R}$ qui admet dans le rep\`ere $\Se\big(O\,;\,I\,;\,J\big)$ la courbe $\mathscr{C}_f$ pour repr\'esentation :

\leavevmode\hfil\hbox{\Image{5020_graph3-1.pdf}{1}{93}{40}}

Parmi les quatres courbes $\mathscr{C}'_1$, $\mathscr{C}'_2$, $\mathscr{C}'_3$ et $\mathscr{C}'_4$ pr\'esent\'ees ci-dessous, d\'eterminer la courbe repr\'esentative de la fonction $f'$, d\'eriv\'ee de la fonction $f$. Justifier votre choix :

\leavevmode\hfil\hbox{\Image{5020_graph4-1.pdf}{1}{45}{25}}%
\hfil%
\hbox{\Image{5020_graph4-2.pdf}{1}{45}{25}}%

\leavevmode\hfil\hbox{\Image{5020_graph4-3.pdf}{1}{45}{25}}%
\hfil%
\hbox{\Image{5020_graph4-4.pdf}{1}{45}{25}}

\vfill\null
\columnbreak

\leavevmode\exercice


On consid\`ere la fonction $f$ d\'efinie sur $\mathbb{R}$ dont la courbe repr\'esentative $\mathscr{C}$ est donn\'ee ci-dessous dans un rep\`ere orthonormal :

\leavevmode\hfil\hbox{\Image{5731_graph-1.pdf}{1}{89}{43}}

\begin{itemize}
\item La courbe $\mathscr{C}_f$ admet deux tangentes horizontales aux points $A$ et $C$ d'abscisses respectives $-7$ et $\dfrac12$ ;

\item La courbe $\mathscr{C}_f$ intercepte l'axe des abscisses aux points $B$ et $D$ de coordonn\'ees respectives $\coord{-1}{0}$ et $\coord{3}{0}$.
\end{itemize}

\vskip\parskip
\begin{enumerate}
\item On consid\`ere la fonction $g$ qui admet pour d\'eriv\'ee la fonction $f$ {\it($g'{=}f$)}. Dresser le tableau de variation de la fonction $g$.

\item On consid\`ere la fonction $h$ qui est la d\'eriv\'ee de la fonction $f$ {\it($f'{=}h$)}. Dresser le tableau de signe de la fonction $h$.
\end{enumerate}

\leavevmode\exercice[*]


On consid\`ere les six fonctions $f$, $g$, $h$, $j$, $k$, $l$ d\'efinies sur $\Se[-4\,;\,4]$ dont les courbes repr\'esentatives sont donn\'ees ci-dessous :

\def\x #1#2{\hbox{\Image{2926_graph#1-#2.pdf}{0.95}{48}{31}}}

\leavevmode\hfil\x{}{2}\hfil\x{3}{2}

\leavevmode\hfil\x{2}{2}\hfil\x{2}{1}

\leavevmode\hfil\x{3}{1}\hfil\x{}{1}

Associer \`a trois de ces fonctions leurs trois d\'eriv\'ees correspondantes.

\vfill\null
\columnbreak

\leavevmode\exercice


On consid\`ere la fonction $f$ d\'efinie et d\'erivable sur l'intervalle $\Se\big[-4\,;\,4\big]$ dont la courbe repr\'esentative $\mathscr{C}_f$ est donn\'ee dans le rep\`ere $\Se\big(O\,;\,I\,;\,J\big)$ orthonorm\'e ci-dessous :

\leavevmode\hfil\hbox{\Image{4730_graph-1.pdf}{1}{91}{43}}

On r\'epondra \`a l'ensemble des questions de cet exercice en se r\'ef\'erant au graphique ci-dessus.

\vskip\parskip
\begin{enumerate}
\item Dresser le tableau de variation de la fonction $f$ sur $\Se\big[-4\,;\,4\big]$.

\item \begin{enumerate}
\item On consid\`ere la tangente $(T_1)$ la tangente \`a la courbe $\mathscr{C}_f$ au point d'abscisse $-3$. Donner le signe du coefficient directeur de la tangente $(T_1)$.

\item On consid\`ere la tangente $(T_2)$ la tangente \`a la courbe $\mathscr{C}_f$ au point d'abscisse $0$. Donner le signe du coefficient directeur de la tangente $(T_2)$.

\item On consid\`ere la tangente $(T_3)$ la tangente \`a la courbe $\mathscr{C}_f$ au point d'abscisse $-2$. Donner le signe du coefficient directeur de la tangente $(T_3)$.
\end{enumerate}

\item \begin{enumerate}
\item Quel est le signe du nombre d\'eriv\'ee de la fonction $f$ en $\Se x={-}1$?

\item Quel est le signe du nombre d\'eriv\'ee de la fonction $f$ en $\Se x=2$?

\item Quel est le signe du nombre d\'eriv\'ee de la fonction $f$ en $\Se x=2,5$?
\end{enumerate}

\item On note $f'$ la fonction d\'eriv\'ee de la fonction $f$. Dresser le tableau de signe de la fonction $f'$.
\end{enumerate}


\leavevmode\exercice


On consid\`ere la fonction $f$ d\'efinie sur $\mathbb{R}$ par la relation :\newline
\hglue\leftmargini$f(x)=\dfrac13{\cdot}x^3-x^2-3x+1$

La courbe $\mathscr{C}_f$ repr\'esentative de la fonction $f$ est donn\'ee dans le rep\`ere $\Se\big(O\,;\,I\,;\,J\big)$ orthogonal ci-dessous :

\leavevmode\hfil\hbox{\Image{5230_graph-1.pdf}{1}{93}{52}}

\vskip\parskip
\begin{enumerate}
\item Graphiquement, dresser le tableau de variation de la fonction $f$ sur l'intervalle $\Se\big[-3\,;\,6\big]$. {\it(on n'indiquera pas les valeurs des images)}

\item \begin{enumerate}
\item D\'eterminer l'expression de la fonction $f'$.

\item Dresser le tableau de signe de la fonction $f'$ sur $\mathbb{R}$.
\end{enumerate}

\item Que remarque-t-on?
\end{enumerate}

\leavevmode\exercice


On consid\`ere la fonction $f$ d\'efinie par la relation :

\hglue\leftmargini$f(x)=\dfrac{3x^2-2x-2}{2x^2+x+1}$

\vskip\parskip
\begin{enumerate}
\item Justifier que la fonction $f$ est d\'efinie sur $\mathbb{R}$.

\item \begin{enumerate}
\item Etablir que la fonction d\'eriv\'ee $f'$ admet l'expression suivante :\newline
\hglue\leftmarginii$f'(x)=\dfrac{7x^2+14x}{\big(2x^2+x+1\big)^2}$

\item Dresser le tableau de variation de la fonction $f$.\newline
On admet les deux limites suivantes :\newline
\hglue\leftmarginii$\lim_{x\mapsto-\infty} f(x)=\dfrac32$%
\quad\string;\quad%
$\lim_{x\mapsto+\infty} f(x)=\dfrac32$
\end{enumerate}

\item En d\'eduire que la fonction $f$ admet pour minorant le nombre $-2$ et pour majorant le nombre $2$.
\end{enumerate}


\leavevmode\exercice


On consid\`ere la fonction d\'efinie par la relation suivante :

\hglue\leftmargini$f{:x}\longmapsto (5x^2+5x-4){\cdot}\sqrt{x\mathstrut}$

\vskip\parskip
\begin{enumerate}
\item D\'eterminer l'ensemble $\mathcal{D}_f$ de d\'efinition de la fonction $f$.

\item \begin{enumerate}
\item D\'eterminer l'expression de la fonction d\'eriv\'ee de $f$.

\item Dresser le tableau de signe de la fonction $f'$.
\end{enumerate}

\item Donner le tableau de variation de la fonction $f$.\newline
On admet les deux limites suivantes :\newline
\hglue\leftmarginii$\lim_{x\mapsto0^+} f(x)=0$%
\quad\string;\quad%
$\lim_{x\mapsto+\infty} f(x)=+\infty$
\end{enumerate}

\leavevmode\exercice


On consid\`ere la fonction $f$ d\'efinie par :

\hglue\leftmargini$f{:x}\longmapsto \sqrt{x\mathstrut}{\cdot}\big({-}5x^2-5x-1\big)+\dfrac12$

\begin{enumerate}
\item Donner l'ensemble de d\'efinition de la fonction $f$.

\item D\'eterminer l'expression de la fonction d\'eriv\'ee $f$.

\item Dresser le tableau de variation complet de la fonction $f$.\newline
On admet la limite suivante :%
\quad$\lim_{x\mapsto+\infty} f(x)=-\infty$

\item \begin{enumerate}
\item Justifier que la fonction $f$ s'annule une seule fois sur son ensemble de d\'efinition.

\item Justifier, \`a l'aide de valeur approch\'ee,  que la fonction $f$ s'annule entre $\dfrac{1}{10}$ et $\dfrac{15}{100}$.
\end{enumerate}
\end{enumerate}

\vfill\null
\columnbreak

\leavevmode\exercice


\dimen0=52mm\dimen1\hsize\advance\dimen1-\dimen0
\begin{minipage}[t]{\dimen1}
On consid\`ere la fonction $f$ d\'efinie sur $\mathbb{R}$ par la relation :\newline
\hglue\leftmargini$f(x)=4-x^2$

\vskip0.2cm
Ci-dessous, est donn\'ee la courbe $\mathscr{C}_f$ repr\'esentative de la fonction $f$ dans le plan muni d'un rep\`ere $\Se\big(O\,;\,I\,J\big)$ :
\end{minipage}
\begin{minipage}[t]{\dimen0}
\leavevmode\hfill\Baisse{1}{47}\hbox{\Image{5279_graph-1.pdf}{1}{50}{50}}
\end{minipage}

Le point $M$ est un point de l'axe des abscisses de coordonn\'ees $\coord{x}{0}$ o\`u $\Se x\in\big[0\,;\,2\big]$. A partir du point $M$, on construit le rectangle $MNPQ$ dont les c\^ot\'es sont parall\`eles aux axes.

D\'eterminer la position du point $M$ afin que l'aire du rectangle $MNPQ$ soit maximale.

{\sl Dans cet exercice, toute trace de recherche ou d'initiative, m\^eme incompl\`ete, sera prise en compte dans l'\'evaluation.}

\leavevmode\exercice


On consid\`ere la fonction $f$ d\'efinie sur $\mathbb{R}_+$ par la relation :\newline
\hglue\leftmargini$f(x)=\dfrac1{x^2-x+1}$

Dans le plan muni d'un rep\`ere $\Se\big(O\,;\,I\,;\,J\big)$, on consid\`ere la courbe $\mathscr{C}_f$ repr\'esentative de la fonction $f$ :

\leavevmode\hfil\hbox{\Image{5244_graph-1.pdf}{1}{85}{34}}

On consid\`ere un point $M$ appartenant \`a la courbe $\mathscr{C}_f$ d'abscisse $x$ et on construit comme l'indique la figure ci-dessus un rectangle o\`u les points $O$ et $M$ sont des sommets de celui-ci.

On note $\mathcal{A}(x)$ l'aire de ce rectangle en fonction de la valeur de $x$.

\vskip\parskip
\begin{enumerate}
\item Donner l'expression de la fonction $\mathcal{A}$.

\item \begin{enumerate}
\item D\'eterminer l'expression de la fonction $\mathcal{A}'$ d\'eriv\'ee de la fonction $\mathcal{A}$.

\item Dresser le tableau de signe de la fonction $\mathcal{A}'$.

\item Dresser le tableau de variations de la fonction $\mathcal{A}$.
\end{enumerate}

\item Quel est la position du point $M$ afin que l'aire du rectangle soit maximale?
\end{enumerate}

\leavevmode\exercice


On consid\`ere la fonction $f$ d\'efinie sur l'intervalle $\Se\Big]\dfrac23\,;\,+\infty\Big[$ par la relation :\newline
\hglue\leftmargini$f(x)=\dfrac{x+1}{3x-2}$

La repr\'esenation $\mathscr{C}_f$ est donn\'ee ci-dessous :

\leavevmode\hfil\hbox{\Image{2828_graph-1.pdf}{1}{87}{47}}

On consid\`ere un point $M$ appartenant \`a la courbe $\mathscr{C}_f$ et le rectangle $MNOP$ construit \`a partir du point $O$ et $M$ et dont les c\^ot\'es sont parall\`eles aux axes.

On note $\mathcal{A}(x)$ l'aire du rectangle $MNOP$ o\`u $x$ est l'abscisse du point $M$. Le but de l'exercice est de d\'eterminer pour quelles valeurs de $x$, l'aire $\mathcal{A}(x)$ est minimale.

\vskip\parskip
\begin{enumerate}
\item Donner l'expression de $\mathcal{A}(x)$ en fonction de $x$.

\item D\'eterminer l'expression de la fonction d\'eriv\'ee de $\mathcal{A}$.

\item Etablir le sens de variation de la fonction $\mathcal{A}$.

\item En d\'eduire la position du point $M$ afin que l'aire du rectangle $MNOP$ soit minimale.
\end{enumerate}

\leavevmode\exercice


Sous un hangar, dont le toit est de forme ``{\sl parabolique}'', on souhaite installer une habitation de forme parall\'el\'epip\'edique. Le dessin ci-dessous illustre le probl\`eme :

\leavevmode\hfil\hbox{\Image{5243_dessin-1.pdf}{1}{76}{28}}

On suppose l'habitat s'\'etalant sur toute la longueur du hangar. Le but de cet exercice est de d\'eterminer les dimensions de la fa\c cade de cet habitat afin d'en maximaliser le volume.

On mod\'elise ce probl\`eme par la figure ci-dessous :

\leavevmode\hfil\hbox{\Image{5243_dessin-2.pdf}{1}{67}{33}}

Le rectangle $DEFG$ admet la droite $(CO)$ pour axe de sym\'etrie. On note $x$ la mesure de la longueur $AG$. 

Dans le rep\`ere $\Se\big(A\,;\,I\,;\,J\big)$, la courbe $\mathscr{C}_f$ est la courbe repr\'esentative de la fonction $f$ d\'efinie sur $\Se\big[0\,;\,6\big]$ par la relation :\newline
\hglue\leftmargini$f(x)=-\dfrac14{\cdot}x^2+\dfrac32{\cdot}x$

On note $\mathcal{A}(x)$ l'aire du rectangle $DEFG$ en fonction de $x$.

\vskip\parskip
\begin{enumerate}
\item Le point $G$ appartenant au segment $[AO]$, quelles sont les valeurs possibles pour la variable $x$ exprim\'ee en m\`etre?

\item D\'emontrer que pour $\Se x\in\big[0\,;\,3\big]$ :\newline
\hglue\leftmarginii$\mathcal{A}(x)=\dfrac12{\cdot}x^3-\dfrac92{\cdot}x^2+9{\cdot}x$

\item \begin{enumerate}
\item D\'eterminer le tableau de variation de la fonction $\mathcal{A}$ sur l'intervalle $\Se\big[0\,;\,3\big]$.

\item En d\'eduire la valeur de $x$ pour laquelle l'aire du rectangle $DEFG$ est maximale.
\end{enumerate}
\end{enumerate}


\leavevmode\exercice


\dimen0=53mm\dimen1\hsize\advance\dimen1-\dimen0
\begin{minipage}[t]{\dimen1}
\parskip0.2cm
On consid\`ere la fonction $f$ d\'efinie par la relation :\newline
\hglue\leftmargini$f(x)=x^2+x$

La repr\'esentation graphique est donn\'ee ci-contre :

On consid\`ere le point $J$ de coordonn\'ee $\coord01$ et $M$ un point de la courbe $\mathscr{C}_f$.

D\'eterminer la position du point $M$ pour laquel la longueur $JM$ est minimale.
\end{minipage}
\begin{minipage}[t]{\dimen0}
\leavevmode\hfil\Baisse{1}{60}\hbox{\Image{5246_graph-1.pdf}{1}{51}{63}}
\end{minipage}

{\sl Au cours de l'exercice, on utilisera la factorisation :\newline
\hglue\leftmargini$4x^3+6x^2-2=2(x+1)^2(2x-1)$}

\leavevmode\exercice


On consid\`ere la fonction $f$ dont l'image d'un nombre $x$ est d\'efinie par la relation :\newline
\hglue\leftmargini$f(x)=\dfrac{-x^2-3{\cdot}x+1}{x-1}$

\vskip\parskip
\begin{enumerate}
\item Donner l'ensemble $\mathcal{D}_f$ de d\'efinition de la fonction $f$.

\item D\'eterminer l'expression de la fonction $f'$  d\'eriv\'ee de la fonction $f$.

\item Etablir le tableau de signe de la fonction $f'$ sur $\mathcal{D}_f$.

\item En d\'eduire le tableau de variations de la fonction $f$ sur $\mathcal{D}_f$.\newline
{\it(on ne compl\'etera pas les valeurs du tableau ...)}
\end{enumerate}

\leavevmode\exercice


On consid\`ere la plan muni d'un rep\`ere $\Se\big(O\,;\,I\,;\,J\big)$ orthonorm\'e repr\'esent\'e ci-dessous :

\leavevmode\hfil\hbox{\Image{6667_graph-1.pdf}{1}{81}{48}}

Le point $A$ a pour coordonn\'ees $A\coord{1}{1}$.

Pour tout nombre r\'eel $x$ appartenant \`a l'intervalle $\Se\big[0\,;\,1]$, on consid\`ere les deux points $M$ et $N$ d\'efinis par :

\begin{itemize}
\item $\Se M\in[OJ]$%
\quad\string;\quad%
$\Se JM=x$

\item $\Se N\in[OI)$%
\quad\string;\quad%
$\Se N\not\in[OI]$%
\quad\string;\quad%
$\Se IN=x$
\end{itemize}

Le point $P$ est d\'efinit par l'intersection des droites $(MN)$ et $(AI)$.

D\'eterminer la valeur de $x$ afin que l'ordonn\'ee du point $P$ soit maximale.

\end{multicols*}

\end{document}