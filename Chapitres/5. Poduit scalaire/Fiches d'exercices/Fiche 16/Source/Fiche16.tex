\documentclass{book}
\usepackage[utf8]{inputenc}
\usepackage[T1]{fontenc}
\usepackage{amsmath,amssymb}
\usepackage{array,color,multirow,slashbox,multicol}
\usepackage{mathrsfs}
\usepackage{eurosym}
\usepackage{stmaryrd} %Pour le symbole parallele \\sslash
\usepackage{yhmath}   %pour dessiner les arcs \wideparen
%\usepackage{multicol}
\usepackage[portrait,nofootskip]{chingatome}








\begin{document}

\fontsize{10}{12}\fontfamily{cmr}\selectfont\titre[18pt]{Lieux géométriques et formules trigonométriques}
\fontsize{10}{12}\fontfamily{cmr}\selectfont\begin{multicols*}{2}


%%%%%%%%%%%%%%%%%
\leavevmode\exercice


On rappelle la formule de la m\'ediane :

\boite{Soient $A$ et $B$ deux points du plan et $I$ le milieu du segment $[AB]$. Pour tout point $M$ du plan, on a la relation :

\hglue\leftmargini$AM^2+BM^2=\dfrac12{\cdot}AB^2+2{\cdot}MI^2$}

On consid\`ere le plan muni d'un rep\`ere orthonormal $\Se\big(O\,;\,I\,;\,J\big)$ et les trois points $A$, $B$ et $C$ de coordonn\'ees respectives :\newline
\hglue\leftmargini$\Se A\coord{-2}{-3}$%
\quad\string;\quad%
$\Se B\coord{-1}{2}$%
\quad\string;\quad%
$\Se C\coord{3}{1}$


\vskip\parskip
\begin{enumerate}
\item D\'eterminer les mesures $AB$, $AC$ et $BC$.

\item \begin{enumerate}
\item On note $I$ le milieu du segment $[AB]$. D\'eterminer la mesure de la m\'ediane dans le triangle $ABC$ issue du sommet $C$.

\item On note $J$ le milieu du segment $[AC]$. D\'eterminer la mesure de la m\'ediane dans le triangle $ABC$ issue du sommet $B$.
\end{enumerate}
\end{enumerate}

\leavevmode\exercice


On consid\`ere un triangle $ABC$ rectangle en $A$ ayant les mesures suivantes :

\hglue\leftmargini$AB=6$%
\quad\string;\quad%
$AC=3$

On note $I$ le milieu du segment $[AB]$ et $J$ le milieu de $[IC]$.

On s'int\'eresse \`a l'ensemble $\mathcal{E}$ des points $M$ v\'erifiant la relation :

\hglue\leftmargini$MA^2+MB^2+2{\cdot}MC^2=72$

{\bf $1^{\text{er}}$ m\'ethode :}

\vskip\parskip
\begin{enumerate}
\item Montrer que tous points $M$ v\'erifient la relation :

\hglue\leftmarginii$MA^2+MB^2+2{\cdot}MC^2=4{\cdot}MJ^2+JA^2+JB^2+2{\cdot}JC^2$

\item En utilisant par deux fois le th\'eor\`eme de la m\'ediane, d\'emontrer la relation suivante :

\hglue\leftmarginii$\Se M\in\mathcal{E}
\quad\Longleftrightarrow\quad
4{\cdot}MJ^2+\dfrac{AB^2}2+IC^2=72$

\item En d\'eduire la nature de l'ensemble $\mathcal{E}$.
\end{enumerate}

\vskip\parskip
{\bf $2^{\text{\`eme}}$ m\'ethode :}

On munit le plan du rep\`ere $\Se\Big(A\,;\,\dfrac16{\cdot}\Vec{AB}\,;\,\dfrac13{\cdot}\Vec{AC}\Big)$

\vskip\parskip
\begin{enumerate}
\item D\'eterminer les coordonn\'ees des points $A$, $B$, $C$ dans ce rep\`ere.

\item En notant $\coord{x}{y}$ les coordonn\'ees du point $M$, d\'eterminer une \'equation de $\mathcal{E}$ dans ce rep\`ere.

\item En d\'eduire la nature de l'ensemble $\mathcal{E}$.
\end{enumerate}


\leavevmode\exercice


On consid\`ere le plan muni d'un rep\`ere $\Se(O\,;\,I\,;\,J)$ orthonorm\'e et les deux points du plan suivants :\newline
\hglue\leftmargini$A\coord{-3}{2}$%
\quad\string;\quad%
$B\coord{3}{{-}6}$

\vskip\parskip
\begin{enumerate}
\item On d\'esignera par $M$ le point de coordonn\'ees $\Se\coord xy$ :

\vskip\parskip
\begin{enumerate}
\item D\'eterminer les coordonn\'ees du point $I$ milieu de $[AB]$.

\item D\'eterminer les coordonn\'ees des vecteurs $\Vec{IM}$ et $\Vec{AB}$.

\item D\'eterminer la longueur $AB$.
\end{enumerate}

\item \begin{enumerate}
\item D\'eterminer l'\'equation cart\'esienne de l'ensemble des points v\'erifiants la relation :

\hglue\leftmarginii$MA^2-MB^2=40$

\item Donner la nature et les \'el\'ements caract\'eristiques des points $M$ v\'erifiant cette relation.
\end{enumerate}

\item \begin{enumerate}
\item D\'eterminer l'\'equation de l'ensemble des points v\'erifiants la relation :

\hglue\leftmarginii$\Vec{MA}\cdot\Vec{MB}=34$

\item Donner la nature et les \'el\'ements caract\'eristiques des points $M$ v\'erifiant cette relation.
\end{enumerate}

\item \begin{enumerate}
\item D\'eterminer l'\'equation de l'ensemble des points v\'erifiants la relation :

\hglue\leftmarginii$MA^2+MB^2=150$


\item Donner la nature et les \'el\'ements caract\'eristiques des points $M$ v\'erifiant cette relation.
\end{enumerate}
\end{enumerate}




\leavevmode\exercice


\dimen0=38mm\dimen1\hsize\advance\dimen1-\dimen0
\begin{minipage}[t]{\dimen1}
Dans le plan muni d'un rep\`ere orthonorm\'e $\Se\big(O\,;\,I\,;\,J\big)$, on consid\`ere les points $M$ et $N$ tel que :

\vskip0.2cm
\hglue\leftmarginii$\big\|\Vec{OM}\big\|=a$%
\quad\string;\quad%
$\big\|\Vec{ON}\big\|=b$%


\vskip0.2cm
$\Se \coord{\Vec{OI}}{\Vec{OM}}=\alpha$%
\quad\string;\quad%
$\Se \coord{\Vec{OI}}{\Vec{ON}}=\beta$%

\end{minipage}
\begin{minipage}[t]{\dimen0}
\leavevmode\hfill\Baisse{1}{36}\hbox{\Image{6808_dessin-1.pdf}{1}{38}{38}}
\end{minipage}

\vskip\parskip
\begin{enumerate}
\item \begin{enumerate}
\item D\'eterminer les coordonn\'ees des points $M$ et $N$.

\item Donner une expression du produit scalaire $\Vec{OM}{\cdot}\Vec{ON}$.
\end{enumerate}

\item \begin{enumerate}
\item Donner la mesure de l'angle orient\'e :%
\quad$\coord{\Vec{OM}}{\Vec{ON}}$

\item Donner une autre expression de  $\Vec{OM}{\cdot}\Vec{ON}$.
\end{enumerate}

\item En d\'eduire l'\'egalit\'e :\newline
\hglue\leftmarginii$\cos\big(\beta{-}\alpha\big)
=\cos\beta{\cdot}\cos\alpha-\sin\beta{\cdot}\sin\alpha$
\end{enumerate}

\leavevmode\exercice


D\'eterminer une simplification des expressions suivantes :

\vskip\parskip
\begin{enumerate}
\item $\cos 2x\cdot\cos x-\sin2x\cdot \sin x$

\item $\sin 3x\cdot \cos 2x-\sin 2x\cdot \cos 3x$
\end{enumerate}

\leavevmode\exercice


\begin{enumerate}
\item En remarquant l'\'egalit\'e $\Se \dfrac{\pi}{12}=\dfrac{\pi}3-\dfrac{\pi}4$, d\'eterminer les valeurs de $\cos\dfrac{\pi}{12}$ et $\sin\dfrac{\pi}{12}$.

\item D\'eterminer les valeurs de : %
\quad$\cos\dfrac{7\pi}{12}$%
\quad\string;\quad%
$\sin\dfrac{7\pi}{12}$
\end{enumerate}

\leavevmode\exercice[*]


Montrer la relation suivante :

\hglue\leftmargini$\sin(a{+}b){\cdot}\cos(a{-}b)
=\sin a{\cdot}\cos a+\cos b{\cdot}\sin b$

\vspace{3cm}

\leavevmode\exercice[*]


\begin{enumerate}
\item Simplifier l'expression suivante :\newline
\hglue\leftmarginii$\cos\Big(x{+}\dfrac{\pi}4\Big)-\cos\Big(x{-}\dfrac{\pi}4\Big)$

\item Etablir l'\'egalit\'e suivante  :\newline
\hglue\leftmarginii$\dfrac{\sin 5x}{\sin2x}+\dfrac{\sin 2x}{\sin x}
=\dfrac{\big(\sin 3x\big)^2}{\sin(2x){\cdot}\sin(x)}$

\item R\'esoudre l'\'equation suivante :\newline
\hglue\leftmarginii$\dfrac{\sqrt{3\mathstrut}}2{\cdot}\cos(2x)+\dfrac12{\cdot}\sin(2x)=\cos\dfrac{\pi}7$
\end{enumerate}

\vspace{-0.1cm}
\leavevmode\exercice


\begin{enumerate}
\item Etablir la relation suivante :%
\quad$\Big(\cos\dfrac{\pi}8\Big)^2=\dfrac14{\cdot}\big(\sqrt{2\mathstrut}+2\big)$

\item En d\'eduire la valeur de $\cos\dfrac{\pi}8$.

\item Etablir la relation :%
\quad$\sin\dfrac{\pi}8=\dfrac12\sqrt{2-\sqrt{2\mathstrut}\ }$
\end{enumerate}

\vspace{-0.2cm}
\leavevmode\exercice

\vspace{-0.1cm}

On consid\`ere le plan munit d'un rep\`ere $\Se(O\,;\,I\,;\,J)$ orthonorm\'e.

\vskip\parskip
\begin{enumerate}
\item Dans chaque cas, d\'eterminer une \'equation cart\'esienne de la droite perpendiculaire au vecteur $\Vec{u}$  et passant par le point $A$ :

\questionii{a&\Vec{u}=\coord{2}{3}\hbox{\quad et\quad}A\coord{1}{0}\cr
b&\Vec{u}=\coord{-1}{1}\hbox{\quad et\quad}A\coord{-2}{1}\cr}

\item Tracer la repr\'esentation de chacune de ces droites dans le rep\`ere ci-dessous ainsi qu'un repr\'esentant de chaque vecteur $\Vec{u}$ au point $A$ correspondant :

\hglue-\leftmargini\hfil\hbox{\Image{2591_graph-1.pdf}{1}{91}{55}}
\end{enumerate}

\leavevmode\exercice


Dans le plan $\big(O\,;\,I\,;\,J\big)$, on consid\`ere les points $A$, $B$, $C$, $D$ de coordonn\'ees :

\hglue\leftmarginii\vtop{\openup6pt
\halign{$#$\hfil&&\quad\string;\quad$#$\hfil\cr
\Se A\coord{-1}{-1}&
\Se B\coord{2}{-4}&
C\coord{\dfrac{22}{5}}{\dfrac45}&
D\coord{\dfrac15}{\dfrac75}\cr}}

\vskip\parskip
\begin{enumerate}
\item \begin{enumerate}
\item Soit $K$ le milieu du segment $[AB]$. On consid\`ere l'ensemble des points $M\coord{x}{y}$ du plan qui v\'erifie la relation :\newline
\hglue\leftmarginii$\Vec{AB}\cdot \Vec{KM}=0$\newline
D\'eterminer une \'equation v\'erifi\'ee par les coordonn\'ees $\Se\coord{x}{y}$ du point $M$.

\item En d\'eduire une \'equation cart\'esienne de la m\'ediatrice $(d)$ du segment $[AB]$.
\end{enumerate}

\item \begin{enumerate}
\item D\'eterminer les coordonn\'ees d'un vecteur orthogonal \`a la droite $(CD)$.

\item En d\'eduire l'\'equation de la droite $(CD)$.
\end{enumerate}

\item D\'eterminer les coordonn\'ees du point d'intersection des droites $(d)$ et $(d')$.
\end{enumerate}

\leavevmode\exercice

\vspace{-0.1cm}

On consid\`ere les trois \'equations cart\'esiennes suivantes :

\questionii{a&x^2+y^2+6x-4y+9=0\cr
b&x^2+y^2-2x+6y+10=0\cr
c&x^2+y^2+4x-4y+9=0\cr}

\vskip\parskip
\begin{enumerate}
\item Ecrire chacune des \'equations ci-dessus sous la forme :\newline
\hglue\leftmarginii$\big(x-a\big)^2+\big(y-b\big)^2=c$\newline
o\`u $a$, $b$, $c$ sont des nombres r\'eels \`a d\'eterminer.

\item Pour chaque \'equation, en d\'eduire la nature de l'ensemble des points d\'efini par cette \'equation et en pr\'eciser les \'el\`ements caract\'eristiques
\end{enumerate}


\leavevmode\exercice


Le plan est muni d'un rep\`ere orthonorm\'e $\Se(O\,;\,I\,;\,J)$ dont l'unit\'e est le centim\`etre.

\vskip\parskip
\begin{enumerate}
\item On consid\`ere le cercle $\mathscr{C}$ de centre $I$ et de rayon $r$. Dans chaque cas pr\'esent\'e ci-dessous, d\'eterminer l'\'equation du cercle :

\questionii{a&I\coord{1}{2}\hbox{\quad et\quad}r{=}3\,cm\cr
b&I\coord{-3}{1}\hbox{\quad et\quad}r{=}5\,cm\cr}

\item On consid\`ere le cercle $\mathscr{C}'$ dont les points $A$ et $B$ sont diam\'etralement oppos\'es. D\'eterminer l'\'equation du cercle dans chacun des cas suivants :

\questionii{a&A\coord{-2}{0}\hbox{\quad et\quad}B\coord{4}{0}\cr
b&A\coord{2}{{-}3}\hbox{\quad et\quad}B\coord{-1}{2}\cr}
\end{enumerate}



\leavevmode\exercice[*]


Dans le plan muni d'un rep\`ere $\Se\big(O\,;\,I\,;\,J\big)$, on consid\`ere les points $A\coord{-\dfrac{11}5}{\dfrac{12}5}$, $B\coord{\dfrac{21}5}{{-}\dfrac{12}5}$, $C\coord{-3}{2}$ ; les points $A$ et $B$ sont diam\'etralement oppos\'es dans le cercle $\mathscr{C}$ ; le cercle $\mathscr{C}'$ a pour centre $C$ et a pour rayon $2$. On note $M$ et $N$ les deux points d'intersection des cercles $\mathscr{C}$ et $\mathscr{C}'$.

\leavevmode\hfil\hbox{\Image{3088_graph-1.pdf}{1}{93}{63}}

\begin{enumerate}
\item D\'eterminer les \'equations des cercles $\mathscr{C}$ et $\mathscr{C}'$.

\item D\'eterminer les coordonn\'ees des points $M$ et $N$.

\item En d\'eduire l'\'equation cart\'esienne de la droite $(d)$.
\end{enumerate}


\end{multicols*}

\end{document}