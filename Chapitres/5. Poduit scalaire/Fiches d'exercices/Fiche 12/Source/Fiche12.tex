\documentclass{book}
%\usepackage[latin1]{inputenc}
\usepackage[T1]{fontenc}
\usepackage{amsmath,amssymb}
\usepackage{array,color,multirow,slashbox,multicol}
\usepackage{mathrsfs}
\usepackage{eurosym}
\usepackage{stmaryrd} %Pour le symbole parallele \\sslash
\usepackage{yhmath}   %pour dessiner les arcs \wideparen
%\usepackage{multicol}
\usepackage[portrait,nofootskip]{chingatome}








\begin{document}

\fontsize{10}{12}\fontfamily{cmr}\selectfont\titre[18pt]{Mesures principales d'angles orientés.}

\fontsize{10}{12}\fontfamily{cmr}\selectfont\begin{multicols*}{2}


%%%%%%%%%%%%%%%%%
\leavevmode\exercice

\vspace{-0.2cm}
On consid\`ere la droite gradu\'ee ci-dessous o\`u sont plac\'es les points $A\Big(\dfrac{20}3\pi\Big)$, $B\Big({-}\dfrac{17}5\Big)$ et $C\Big(\dfrac{43}8\pi\Big)$.

\leavevmode\hfil\hbox{\Image{2738_dessin-1.pdf}{1}{91}{7}}

\vskip\parskip
\begin{enumerate}
\item \begin{enumerate}
\item Graphiquement, d\'eterminer le nombre de fois dont on doit enlever $2{\cdot}\pi$ \`a l'abscisse du point $A$ afin d'obtenir la mesure principale de ce nombre?

\item En d\'eduire la mesure principale de $\dfrac{20}3$.
\end{enumerate}

\item D\'eterminer la mesure principale des abscisses des points $B$ et $C$.
\end{enumerate}

\leavevmode\exercice
\vspace{-0.2cm}

On munit le plan d'un rep\`ere orthonorm\'e $\Se\big(O\,;\,I\,;\,J\big)$. On d\'esigne par $M$ et $N$ deux points du cercle trigonom\'etrique.

\vskip\parskip

\begin{enumerate}
\item Parmi les mesures d'angles ci-dessous, lesquelles appartiennent \`a l'intervalle des mesures principales :

\questionii{a&\dfrac{5\pi}3&
b&-\dfrac{7\pi}4&
c&\dfrac{-2\pi}3&
d&1,1\pi\cr}

\item D\'eterminer la mesure principale des angles d\'efinis par les points $M$, $N$, $P$ et $Q$ ci-dessous, puis placer chacun de ces points sur le cercle trigonom\'etrique ci-contre :

\dimen0=45mm\dimen1\linewidth\advance\dimen1-\dimen0%
\begin{minipage}[t]{\dimen1}
\questionii{a&\coord{\Vec{OI}}{\Vec{OM}}=\dfrac{7\pi}{3}\cr
b&\coord{\Vec{OI}}{\Vec{ON}}=-\dfrac{15\pi}{4}\cr
c&\coord{\Vec{OI}}{\Vec{OP}}=\dfrac{5\pi}{3}\cr
d&\coord{\Vec{OI}}{\Vec{OQ}}=\dfrac{19\pi}{6}\cr}
\end{minipage}
\begin{minipage}[t]{\dimen0}
\hfill\Baisse{1}{37}\hbox{\Image{535_dessin1-1.pdf}{1}{42}{42}}
\end{minipage}
\end{enumerate}

\vspace{-0.5cm}
\leavevmode\exercice


D\'eterminer la mesure principale des angles orient\'es de mesure suivante : 

\questioni{a&\dfrac{9\pi}4&
b&\dfrac{192\pi}6&
c&-\dfrac{5\pi}4\cr
d&-\dfrac{33\pi}2&
e&\dfrac{16\pi}7&
f&\dfrac{52\pi}3\cr}

\leavevmode\exercice


\begin{enumerate}
\item On se propose, dans cette question, de d\'eterminer la mesure principale de l'angle $\Se \alpha=\dfrac{73}5\pi$ :

\begin{enumerate}
\item Soit $k$ un entier relatif r\'ealisant l'encadrement suivant :

\hglue\leftmarginii$\Se -\pi\ <\ \dfrac{73}5\pi+2{\cdot}k{\cdot}\pi\ \leq\  \pi$

R\'ealiser un encadrement de $k$ \`a l'aide de l'encadrement ci-dessus.

\item A l'aide de la calculatrice, d\'eterminer l'unique nombre entier $k$ r\'ealisant cet encadrement.

\item En d\'eduire la mesure principale de l'angle $\alpha$.
\end{enumerate}

\item De la m\^eme mani\`ere, d\'eterminer la mesure principale des angles suivants :

\questionii{a&-\dfrac{29}{3}\pi&
b&-\dfrac{27}4\pi&
c&\dfrac{70}{9}\pi\cr}
\end{enumerate}

\vspace{-0.3cm}
\leavevmode\exercice


\begin{enumerate}
\item Donner, sous forme de r\'eunions d'intervalles, l'ensemble form\'e par les mesures principales des angles rep\'erant les points surlign\'es du cercle trigonom\'etrique :

\def\x #1{\hglue-5pt\Baisse{1}{30}\hbox{\Image{2799_dessin-#1.pdf}{1}{35}{35}}}
\hglue\leftmarginii\labelii{a} \x{1}
\hglue\leftmarginii\labelii{b} \x{2}

\vspace{-0.3cm}
\item Pour chaque question, surligner l'ensemble des points ayant pour angle orient\'e l'ensemble pr\'ecis\'e sous le cercle trigonom\'etrique :

\def\x #1{\hbox{\hglue-5pt\Baisse{1}{30}\hbox{\Image{2799_dessin2-1.pdf}{1}{34}{34}}}}
\hglue\leftmarginii\labelii{a} \vtop{\x{1}
\hbox{\quad$\Se\Big[\dfrac{13\pi}{3}\,;\;\dfrac{29\pi}{6}\Big]$}}
\hglue\leftmarginii\labelii{b} \vtop{\x{2}
\hbox{\quad$\Se\Big[\dfrac{\pi}{2}\,;\;\dfrac{8\pi}{3}\Big]$}}
\end{enumerate}

\vspace{-0.8cm}

\leavevmode\exercice


On consid\`ere le carr\'e $ABCD$.\newline
Soit le point $E$ ext\'erieur au carr\'e tel que $BCE$ soit \'equilat\'eral.\newline
Soit $F$ le point int\'erieur au carr\'e tel que le triangle $ABF$ soit \'equilat\'eral.

\leavevmode\hfil\hbox{\Image{2233_dessin-1.pdf}{1}{83}{47}}

\vspace{-0.2cm}

On souhaite montrer que les points $D$, $F$ et $E$ sont align\'es.

\vskip\parskip
\begin{enumerate}
\item \begin{enumerate}
\item Donner la mesure des deux angles orient\'es suivants:

\hglue\leftmarginii$\coord{\Vec{AF}}{\Vec{AD}}$%
\quad\string;\quad%
$\coord{\Vec{DF}}{\Vec{DA}}$

\item En d\'eduire la mesure de l'angle orient\'e $\coord{\Vec{DC}}{\Vec{DF}}$.
\end{enumerate}

\item \begin{enumerate}
\item Donner la mesure de l'angle orient\'e $\coord{\Vec{CD}}{\Vec{CE}}$.

\item En d\'eduire la mesure de l'angle orient\'e $\coord{\Vec{DC}}{\Vec{DE}}$.
\end{enumerate}

\item En d\'eduire que les points $D$, $F$ et $E$ sont align\'es.
\end{enumerate}

Les questions suivantes ont pour objectif d'utiliser la relation de Chasles.

\vskip\parskip
\begin{enumerate}
\setcounter{enumi}3
\item D\'etermmminer la mesure des angles orient\'es :

\questionii{a&\coord{\Vec{BE}}{\Vec{CF}}&
b&\coord{\Vec{AF}}{\Vec{CE}}\cr}
\end{enumerate}

\end{multicols*}


\end{document}