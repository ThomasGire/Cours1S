\documentclass{book}
\usepackage[utf8]{inputenc}
\usepackage[T1]{fontenc}
\usepackage{amsmath,amssymb}
\usepackage{array,color,multirow,slashbox,multicol}
\usepackage{mathrsfs}
\usepackage{eurosym}
\usepackage{stmaryrd} %Pour le symbole parallele \\sslash
\usepackage{yhmath}   %pour dessiner les arcs \wideparen
%\usepackage{multicol}
\usepackage[portrait,nofootskip]{chingatome}








\begin{document}

\fontsize{10}{12}\fontfamily{cmr}\selectfont\titre[18pt]{Suites explicites et récurrentes}
\fontsize{10}{12}\fontfamily{cmr}\selectfont\begin{multicols*}{2}

%%%%%%%%%%%%%%%%%
\leavevmode\exercice


On consid\`ere l'algorithme suivant :

\leavevmode\hfil\begin{myAlgo}[14]
Pour $i$ allant de $0$ \`a 5
    $a$ $\leftarrow$ $\Se i\times (i-1)$
Fin Pour
\end{myAlgo}

\vskip\parskip
\begin{enumerate}
\item Lors de l'ex\'ecution pas \`a pas de cet algorithme, donner les valeurs prises par la variable {\tt a}.

\item Donner l'expression d'une suite $\big(u_n\big)$ dont les six premiers termes sont les valeurs affich\'ees par l'algorithme.
\end{enumerate}

\leavevmode\exercice


On consid\`ere la fonction $f$ d\'efinie sur $\mathbb{R}_+$ dont la courbe repr\'esentative $\mathscr{C}_f$ est donn\'ee dans le rep\`ere orthonormal $\Se\big(O\,;\,I\,;\,J\big)$ ci-dessous :

\leavevmode\hfil\hbox{\Image{5089_graph-1.pdf}{1}{90}{43}}

On d\'efinie la suite $\big(u_n\big)$ par la relation :\newline
\hglue\leftmargini$u_n=f(n)$\quad pour tout entier $\Se n\in\mathbb{N}$.

\vskip\parskip
\begin{enumerate}
\item Justifier que le terme $u_4$ a pour valeur $\dfrac32$.

\item D\'eterminer la valeur des termes :\newline
\hglue\leftmarginii$u_0$%
\quad\string;\quad%
$u_1$%
\quad\string;\quad%
$u_2$%
\quad\string;\quad%
$u_3$%
\quad\string;\quad%
$u_4$%
\quad\string;\quad%
$u_5$%
\quad\string;\quad%
$u_6$%
\end{enumerate}



\leavevmode\exercice


D\'eterminer les 5 premiers termes des suites suivantes :

\questioni[0.75cm]{a&u_n=2{\cdot}n^2-n+1&
b&v_n=\dfrac{2{\cdot} n+1}{2-3{\cdot}n}\cr
c&w_n=\sqrt{3n+25\mathstrut\,}&
d&x_n=3{\cdot}\big[1+(-1)^n\big]+2\cr}

\leavevmode\exercice


\begin{enumerate}
\item On consid\`ere la suite $\big(u_n\big)_{n\in\mathbb{N}}$ d\'efinie par la formule explicite :\newline
\hglue\leftmarginii$u_n=5+2\times n$\quad pour tout entier naturel $n$.

\vskip\parskip
\begin{enumerate}
\item Exprimer la valeur $u_{n-3}$ en fonction de $n$.

\item Donner la forme simplifi\'ee de $u_{n-3}{+}u_3$.

\item Donner la forme simplifi\'ee de $u_{n-5}{+}u_5$.

\item Soit $k$ et $n$ deux entiers tels que $\Se k\leq n$. Montrer que $u_k{+}u_{n-k}$ a sa valeur ind\'ependante de $k$.
\end{enumerate}

\item On consid\`ere la suite $\big(v_n\big)_{n\in\mathbb{N}}$ d\'efinie par la formule explicite :\newline
\hglue\leftmarginii$v_n=2n^2-3n+2$\quad pour tout entier naturel $n$.

On souhaite \'etudier la diff\'erence entre deux termes cons\'ecutifs de la suite $(v_n)$ :

\vskip\parskip
\begin{enumerate}
\item Donner l'expression du terme $v_{n+1}$ en fonction de $n$.

\item Etudier la valeur de $v_{n+1}{-}v_n$ en fonction de $n$.
\end{enumerate}
\end{enumerate}

\leavevmode\exercice


Dans le plan muni d'un rep\`ere orthonorm\'e $\Se\big(O\,;\,I\,;\,J\big)$, on consid\`ere la repr\'esentation $\mathscr{C}_f$ d'une fonction $f$ d\'efinie sur l'intervalle $[-4\,;\,4]$ :

\leavevmode\hfil\hbox{\Image{2978_graph-1.pdf}{1}{91}{80}}

On consid\`ere les suites $(u_n)_{n\in\mathbb{N}}$ et $(v_n)_{n\in\mathbb{N}}$ v\'erifiant les relations :\newline
\hglue\leftmargini$u_{n+1}=f\big(u_n\big)$%
\quad\string;\quad%
$v_{n+1}=f\big(v_n\big)$\quad pour tout $\Se n\in\mathbb{N}$\newline
v\'erifiant les conditions initiales suivantes :\newline
\hglue\leftmargini$u_0=-1$%
\quad\string;\quad%
$v_0=-4$

D\'eterminer les $100$ premiers termes de chacune de ces deux suites.

\leavevmode\exercice


\begin{enumerate}
\item On d\'efinie la suite par r\'ecurrence $\big(u_n\big)_{n\in\mathbb{N}}$ par la relation :\newline
\hglue\leftmarginii\Se $u_0=5$%
\quad\string;\quad%
$u_{n+1}=2{\cdot}u_n-1$\quad pour tout $\Se n\in\mathbb{N}$

D\'eterminer les cinq premiers termes de la suite $(u_n)$.

\item On d\'efinie la suite par r\'ecurrence $\big(v_n\big)_{n\in\mathbb{N}^*}$ par la relation :\newline
\hglue\leftmarginii$\Se v_1=-2$%
\quad\string;\quad%
$v_{n+1}=\dfrac{1-v_n}n$\quad pour tout $\Se n\in\mathbb{N}^*$

D\'eterminer les cinq premiers termes de la suite $(v_n)$.
\end{enumerate}

\leavevmode\exercice


On consid\`ere la construction d'un ch\^ateau de cartes :

\leavevmode\hfil\hbox{\Image{2986_dessin-1.pdf}{1}{86}{45}}

On consid\`ere la suite $\big(u_n\big)_{n\in\mathbb{N}}$ d\'esignant le nombre de cartes utilis\'ees dans la construction du ch\^ateau \`a l'\'etape $n$.

\vskip\parskip
\begin{enumerate}
\item D\'eterminer les quatre premiers termes de la suite $\big(u_n\big)$.

\item Pour tout entier naturel $n$, d\'eterminer une expression du terme $u_{n+1}$ en fonction du terme pr\'ec\'edent $u_n$ et du rang $n$.

\item A quel \'etape de construction peut-on arriver avec deux jeux de 72 cartes?
\end{enumerate}

\leavevmode\exercice


Justifier que, dans chaque question, les informations ci-dessous ne d\'efinissent pas de suites :

\vskip\parskip
\begin{enumerate}
\changelabel
\item $u_0=5$%
\quad\string;\quad%
$u_{n+1}=2{\cdot}u_n-3$\quad pour tout $\Se n\in\mathbb{N}^*$

\item $u_0=1$%
\quad\string;\quad%
$u_1=4$%
\quad\string;\quad%
$u_{n+1}=u_n-3$\quad pour tout $\Se n\in\mathbb{N}$

\item $u_0=3$%
\quad\string;\quad%
$u_n=2{\cdot}u_{n-1}-2$\quad pour tout $\Se n\in\mathbb{N}$

\item $u_0=-1$%
\quad\string;\quad%
$u_n=\dfrac{u_{n-1}-2}{u_{n-1}+1}$\quad pour tout $\Se n\in\mathbb{N}^*$
\end{enumerate}


\leavevmode\exercice


On consid\`ere l'algorithme suivant :

\leavevmode\hfil\begin{myAlgo}[15]
$a$ $\leftarrow$ $2$
Pour $i$ allant de $0$ \`a $5$
    $a$ $\leftarrow$ $a\times2$
Fin Pour
\end{myAlgo}

\vskip\parskip
\begin{enumerate}
\item Lors de son ex\'ecution pas \`a pas, indiquer les diff\'erentes valeurs prises par la variable {\tt a}

\item Parmi les exrpressions choisies qu'elle{\it(s)} peuvent \^etre l'expression d'une suite $\big(u_n\big)$ afin que ses six premiers termes soient les valeurs prises par la variable {\tt a} lors de l'ex\'ecution de l'algorithme pr\'ec\'edent :

\hglue-3\leftmarginii\questionii[0.65cm]{a&u_n=2{\cdot}n,\quad\forall n{\in}\mathbb{N}&
b&u_n=2^n,\quad\hbox{$\Se \forall n\in\mathbb{N}$}\cr
c&u_n=2^{n+1},\quad\forall n{\in}\mathbb{N}&
d&\left\{\ \vcenter{\hbox{$\begin{array}{rcl}
u_0&=&2\\[4pt]
u_{n+1}&=&2{\cdot}u_n,\ \hbox{$\Se \forall n\in\mathbb{N}$}\\
\end{array}$}}\right.\cr
e&\left\{\ \vcenter{\hbox{$\begin{array}{rcl}
u_0&=&2\\[4pt]
u_n&=&2{\cdot}u_{n+1},\ \hbox{$\Se \forall n\in\mathbb{N}$}\\
\end{array}$}}\right.&
f&\left\{\ \vcenter{\hbox{$\begin{array}{rcl}
u_0&=&2\\[4pt]
u_{n}&=&2{\cdot}u_{n-1},\ \hbox{$\Se \forall n\in\mathbb{N}^*$}\\
\end{array}$}}\right.\cr}
\end{enumerate}

\leavevmode\exercice


On consid\`ere l'algorithme suivant :

\leavevmode\hfil\begin{myAlgo}[14]
$a$ $\leftarrow$ $-1$
Pour $i$ allant de $0$ \`a 4
    $a$ $\leftarrow$ $\se a\times2-i+1$
Fin Pour
\end{myAlgo}

\vskip\parskip
\begin{enumerate}
\item Donner les diff\'erentes valeurs prises par la variable {\tt a} lors d'une ex\'ecution pas \`a pas de cet algorithme.

\item Donner l'expression d'une suite dont les cinq premiers termes soient les diff\'erentes valeurs prises par la variable {\tt a} prises lors de l'ex\'ecution de cet algorithme.
\end{enumerate}

\leavevmode\exercice


Dans chaque cas, d\'eterminer les quatre premiers termes de la suite $\big(u_n\big)_{n\in\mathbb{N}}$ :

\hglue-\leftmargini\questioni[0pt]{a&u_n=\dfrac{2n^2+n+5}{n+1}\hbox{\quad pour tout $\Se n\in\mathbb{N}$}\cr
b&u_0=2\quad\string;\quad u_{n+1}=\dfrac12{\cdot}u_n+3\text{\quad pour tout $\Se n\in\mathbb{N}$}\cr
c&u_0=-1\quad\string;\quad u_{n+1}=u_n+n-2\text{\quad pour tout $\Se n\in\mathbb{N}$}\cr
d&u_0=2\ \string;\ u_1=3\ \string;\ u_{n+1}=u_n+2{\cdot}u_{n-1}\hbox{\ pour tout $\Se n\in\mathbb{N}^*$}\cr}

\leavevmode\exercice


\begin{enumerate}
\item On consid\`ere la suite $\big(u_n\big)$ d\'efinie par :\newline
\hglue\leftmarginii$\left\{\ \vcenter{\openup4pt\hbox{$u_0=3$\vrule width0pt height8pt%
\quad\string;\quad%
$u_1=1$}
\hbox{$u_{n+2}=2{\cdot}u_{n+1}+u_n$\quad pour tout $\Se n\in\mathbb{N}$}}\right.$

Donner les cinq premiers termes de la suite $\big(u_n\big)$.

\item On consid\`ere la suite $\big(v_n\big)$ d\'efinie par :\newline
\hglue\leftmarginii$v_0=-3$%
\quad\string;\quad%
$v_{n+1}=n-2{\cdot}v_{n}$\quad pour tout $\Se n\in\mathbb{N}$

Donner les quatre premiers termes de la suite $\big(v_n\big)$.
\end{enumerate}

\leavevmode\exercice


On construit successivement un objet comme le repr\'esente le sch\'ema ci-dessous :

\leavevmode\hfil\hbox{\Image{5858_dessin-1.pdf}{1}{84}{37}}

Pour tout entier naturel $n$ non-nul, on note $u_n$ le nombre de planches n\'ecessaires pour construire la figure \`a l'\'etape $n$.

Donner une relation de r\'ecurrence caract\'erisant la suite $\big(u_n\big)$.

\leavevmode\exercice


On consid\`ere la suite $\big(u_n\big)$ d\'efinie par :\newline
\hglue\leftmargini$u_0=1$%
\quad\string;\quad%
$u_{n+1}=2{\cdot}u_n+3^n$\quad pour tout $\Se n\in\mathbb{N}$.

\vskip\parskip
\begin{enumerate}
\item \begin{enumerate}
\item D\'eterminer les cinq premiers termes de $\big(u_n\big)$.

\item Quelle conjecture peut-on faire sur la nature de $\big(u_n\big)$
\end{enumerate}

\item Montrer que la suite g\'eom\'etrique $\big(v_n\big)$ de premier terme $v_0$ et de raison $3$ v\'erifie la relation :\newline
\hglue\leftmarginii$v_{n+1}=2{\cdot}v_n+3^n$.
\end{enumerate}

\leavevmode\exercice


On consid\`ere la suite $\big(u_n\big)_{n\in\mathbb{N}}$ d\'efinie par la relation de r\'ecurrence suivante :\newline
\hglue\leftmargini$u_0=3$%
\quad\string;\quad%
$u_{n+1}=\dfrac{1-u_n}{1+u_n}$\quad pour tout $\Se n\in\mathbb{N}$

\vskip\parskip
\begin{enumerate}
\item D\'eterminer les cinq premiers termes de la suite $\big(u_n\big)$.

\item Montrer qu'on a la relation suivante :\newline
\hglue\leftmarginii$u_{n+2}=u_n$\quad pour tout $\Se n\in\mathbb{N}$

\item Que peut-on dire des termes de cette suite?

\item D\'eterminer la valeur des r\'eels $a$ et $b$ v\'erifiant la relation suivante :\newline
\hglue\leftmargini$u_n=a{\cdot}\big[1-(-1)^n\big]+b$\quad pour tout $\Se n\in\mathbb{N}$
\end{enumerate}

\end{multicols*}

\end{document}