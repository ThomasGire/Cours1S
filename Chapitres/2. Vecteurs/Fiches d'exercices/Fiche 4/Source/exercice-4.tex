\documentclass{book}
%\usepackage[latin1]{inputenc}
\usepackage[utf8]{inputenc}
\usepackage[T1]{fontenc}
\usepackage{amsmath,amssymb}
\usepackage{array,color,multirow,slashbox,multicol}
\usepackage{mathrsfs}
\usepackage{eurosym}
\usepackage{stmaryrd} %Pour le symbole parallele \\sslash
\usepackage{yhmath}   %pour dessiner les arcs \wideparen
%\usepackage{multicol}
\usepackage[portrait,nofootskip]{chingatome}








\begin{document}

\fontsize{10}{12}\fontfamily{cmr}\selectfont\titre[18pt]{Vecteurs et équations de droites - Fiche d'exercices 4}

\fontsize{10}{12}\fontfamily{cmr}\selectfont\begin{multicols*}{2}


%%%%%%%%%%%%%%%%%
\leavevmode\exercice


On consid\`ere le plan muni d'un rep\`ere $\Se\big(O\,;\,\Vec{i}\,;\,\Vec{j}\big)$ orthogonal :

\leavevmode\hfil\hbox{\Image{5315_graph-1.pdf}{1}{93}{48}}

et les points $A$ et $B$ de coordonn\'ees :%
\hfill$\Se A\coord{-3}{{-}\dfrac12}$%
\ \string;\ %
$B\coord{1}{1}$

\vskip\parskip
\begin{enumerate}
\item Tracer la droite $(AB)$ dans le rep\`ere ci-dessus.

\item Donner quatre vecteurs directeurs de la droite $(AB)$ dont un, au moins, a des coordonn\'ees enti\`eres.
\end{enumerate}

\leavevmode\exercice


On consid\`ere les fonctions affines $f$ et $g$ d\'efinie par la relation :\newline
\hglue\leftmargini$f(x)=\dfrac32{\cdot}x+2$%
\quad\string;\quad%
$g(x)=-2x+1$\newline
Dans le plan muni d'un rep\`ere, on note $(d)$ et $(d')$ les droites repr\'esentatives respectives des fonctions $f$ et $g$. 

\vskip\parskip
\begin{enumerate}
\item Donner trois vecteurs directeurs de la droite $(d)$.

\item Donner trois vecteurs directeurs de la droite $(d')$.
\end{enumerate}

\leavevmode\exercice


Dans le plan muni d'un rep\`ere $\Se\big(O\,;\,\Vec{i}\,;\,\Vec{j}\big)$, on consid\`ere la droite $(d)$ admettant pour \'equation :\newline
\hglue\leftmargini$2x-y+5=0$

\vskip\parskip
\begin{enumerate}
\item Parmi les points ci-dessous, lesquels appartiennent \`a la droite $(d)$ :\newline
\hglue\leftmarginii$A\coord{1}{7}$%
\quad\string;\quad%
$B\coord{-\dfrac32}{2}$%
\quad\string;\quad%
$\Se C\coord{-4}{-4}$%

\item D\'eterminer les coordonn\'ees du point $D$ appartenant \`a la droite $(d)$ ayant pour abscisse $2$.

\item D\'eterminer les coordonn\'ees du point $E$ appartenant \`a la droite $(d)$ ayant pour ordonn\'ee $-\dfrac12$.
\end{enumerate}

\leavevmode\exercice


Dans le plan muni d'un rep\`ere $\Se\big(O\,;\,\Vec{i}\,;\,\Vec{j}\big)$, on donne la repr\'esentation des quatres droites $(d_1)$, $(d_2)$, $(d_3)$ et $(d_4)$ ci-dessous :

\leavevmode\hfil\hbox{\Image{5334_graph-1.pdf}{1}{93}{48}}

Associer \`a chacune des droites ci-dessous une des \'equations cart\'esiennes pr\'esent\'ees ci-dessous :

\hglue\leftmarginii\vtop{\openup6pt
\halign{$#$\hfill&&\quad\string;\quad$#$\hfill\cr
(E_1):\ 3{\cdot}x+4{\cdot}y+4=0&
(E_2):\ -x+2{\cdot}y-3=0\cr
(E_3):\ \dfrac12{\cdot}x-y-1=0&
(E_4):\ \dfrac34{\cdot}x+y-\dfrac32=0\cr}}

\leavevmode\exercice


Dans le plan muni d'un rep\`ere $\Se\big(O\,;\,\Vec{i}\,;\,\Vec{j}\big)$, on consid\`ere les quatres droites ci-dessous d\'efinies par leur \'equation cart\'esienne :

\hglue\leftmargini\vtop{\openup6pt
\halign{$#$\hfill&&\quad\string;\quad$#$\hfill\cr
(d_1):\  2x-3y+3=0&
(d_2):\  -2x-y+1=0\cr
(d_3):\  4x+8y-10=0&
(d_4):\  -3x+y+4=0\cr}}

\vskip\parskip
\begin{enumerate}
\item Pour chacune des droites, donner un point et un vecteur directeur de cette droite.

\item Tracer chacune de ces droites dans le rep\`ere ci-dessous :
\end{enumerate}

\leavevmode\hfil\hbox{\Image{5328_graph-1.pdf}{1}{93}{70}}

\leavevmode\exercice


Dans le plan muni d'un rep\`ere $\Se\big(O\,;\,\Vec{i}\,;\,\Vec{j}\big)$, on consid\`ere les droites ci-dessous :

\hglue\leftmargini\vtop{\openup6pt
\halign{$#$\hfill&\hglue1cm$#$\hfill\cr
(d_1){:}\ \sqrt{3\mathstrut}{\cdot}x-\sqrt{12\mathstrut}{\cdot}y+\sqrt{10\mathstrut}=0\cr
(d_2){:}\ (1+\sqrt{2\mathstrut}){\cdot}x+\sqrt{3\mathstrut}{\cdot}y-1=0\cr
(d_3){:}\ -\sqrt{3\mathstrut}{\cdot}x-(-1+\sqrt{2\mathstrut}){\cdot}y+2=0\cr
(d_4){:}\ (1+\sqrt{2\mathstrut}){\cdot}x+(1-\sqrt{2\mathstrut}){\cdot}y-1=0\cr}}

\vskip\parskip
\begin{enumerate}
\item Donner les coordonn\'ees d'un vecteur directeur de la droite $(d_1)$ ayant ses coordonn\'ees enti\`eres.

\item Donner les coordonn\'ees d'un vecteur directeur des droites $(d_2)$, $(d_3)$, $(d_4)$ ayant pour abscisse une valeur enti\`ere.
\end{enumerate}


\leavevmode\exercice


Dans le plan muni d'un rep\`ere $\Se\big(O\,;\,I\,;\,J\big)$, on consid\`ere les quatre droites suivantes :

\hglue\leftmargini\vtop{\openup6pt
\halign{$#$\hfil&\quad\string;\quad$#$\hfil\cr
(d_1):\ 3{\cdot}x-2{\cdot}y-2=0&
(d_2):\ -x+3{\cdot}y+1=0\cr
(d_3):\ 2{\cdot}x+y=0&
(d_4):\ -2 {\cdot}x-2{\cdot}y+1=0\cr}}

\vskip\parskip
\begin{enumerate}
\item Donner un vecteur directeur de chacune de ces droites.

\item Donner le coefficient directeur de chacune de ces droites.
\end{enumerate}

\leavevmode\exercice


On consid\`ere le plan muni d'un rep\`ere $\Se\big(O\,;\,\Vec{i}\,;\,\Vec{j}\big)$ et les trois droites $(d_1)$, $(d_2)$ et $(d_3)$ d'\'equations cart\'esiennes :

\hglue\leftmargini$(d_1){:}\ 4x-6y+2=0$%
\quad\string;\quad%
$(d_2){:}\ x+2y-3=0$

\hglue\leftmargini$(d_3){:}\ x-\dfrac32{\cdot}y+2=0$%

\vskip\parskip
\begin{enumerate}
\item Les droites $(d_1)$ et $(d_2)$ sont-elles parall\`eles entre elles? Si non, d\'eterminer le point d'intersection de ces deux droites.

\item Les droites $(d_1)$ et $(d_3)$ sont-elles parall\`eles entre elles? Si non, d\'eterminer le point d'intersection de ces deux droites.
\end{enumerate}

\leavevmode\exercice


On consid\`ere le plan muni d'un rep\`ere $\Se\big(O\,;\,I\,;\,J\big)$ et les deux droites $(d_1)$ et $(d_2)$ admettant pour \'equations cart\'esiennes :

\hglue\leftmargini$(d_1):\ x-2y+3=0$%
\quad\string;\quad%
$(d_2):\ 3x+4y-13=0$

\vskip\parskip
\begin{enumerate}
\item Donner les coordonn\'ees d'un vecteur directeur et d'un point de chaque droite.

\item Repr\'esenter dans le graphique ci-dessous les deux droites $(d_1)$ et $(d_2)$.

\hglue-\leftmargini\hfil\hbox{\Image{5395_graph-1.pdf}{1}{93}{59}}

\item D\'eterminer les coordonn\'ees du point d'intersection des deux droites $(d_1)$ et $(d_2)$.
\end{enumerate}

\leavevmode\exercice


Dans le plan muni d'un rep\`ere $\Se\big(O\,;\,\Vec{i}\,;\,\Vec{j}\big)$, on consid\`ere les trois points suivants :\newline
\hglue\leftmargini$\Se A\coord{-3}{-2}$%
\quad\string;\quad%
$B\coord{1}{1}$%
\quad\string;\quad%
$C\coord{-2}{2}$

\vskip\parskip
\begin{enumerate}
\item D\'eterminer une \'equation cart\'esienne de la droite $(AB)$.

\item D\'eterminer une \'equation cart\'esienne de la droite $(d)$ passant par le point $C$ et parall\`ele \`a la droite $(AB)$.

\item \begin{enumerate}
\item D\'eterminer les coordonn\'ees du point $M$ milieu du segment $[AC]$.

\item D\'eterminer une \'equation cart\'esienne de la droite $(BM)$

\item D\'eterminer les coordonn\'ees du point $D$ intersection des droites $(BM)$ et $(d)$.

\item Quelle est la nature du quadrilat\`ere $ABCD$? Justifier votre r\'eponse.
\end{enumerate}
\end{enumerate}

\leavevmode\exercice


On munit le plan d'un rep\`ere $\Se\big(O\,;\,\Vec{i}\,;\,\Vec{j}\big)$ quelconque repr\'esent\'e ci-dessous :

\leavevmode\hfil\hbox{\Image{4968_quadrillageNonOrtho-1.pdf}{1}{91}{42}}

\vskip\parskip
\begin{enumerate}
\item \begin{enumerate}
\item Dans le rep\`ere ci-dessous, placer les deux points :\newline
\hglue\leftmarginii$A\coord{-1}{2}$%
\quad\string;\quad%
$B\coord{4}{1}$

\item Justifier graphiquement que le vecteur $\Vec{AB}$ a pour coordonn\'ees $\Se\coord{5}{-1}$.
\end{enumerate}

\item On consid\`ere les deux vecteurs suivants :\newline
\hglue\leftmarginii$\Vec{u}\coord{3}{2}$%
\quad\string;\quad%
$\Se\Vec{v}\coord{-2}{-2}$

Donner un repr\'esentant de votre choix de chacun de ces deux vecteurs dans le rep\`ere ci-dessus.
\end{enumerate}


\leavevmode\exercice


Dans le plan, on consid\`ere les deux vecteurs $\Vec{i}$ et $\Vec{j}$ non-colin\'eaire repr\'esent\'es ci-dessous :

\leavevmode\hfil\hbox{\Image{5744_quadrillageNonOrtho-1.pdf}{1}{92}{40}}

La repr\'esentation des vecteurs $\Vec{u}$ et $\Vec{v}$ sont \'egalement repr\'esent\'es ci-dessus.

\vskip\parskip
\begin{enumerate}
\item Dans la base vectorielle de $\Se\big(\Vec{i}\,;\,\Vec{j}\big)$, donner les coordonn\'ees des vecteurs $\Vec{u}$ et $\Vec{v}$.

\item Par la m\'ethode de votre choix, d\'eterminer les coordonn\'ees du vecteur somme :%
\quad$\Se \Vec{w}=\Vec{u}+\Vec{v}$.

\item Par la m\'ethode de votre choix, d\'eterminer les coordonn\'ees du vecteur $\Vec{t}$ r\'ealisant l'\'egalit\'e suivante :\newline
\hglue\leftmarginii$\Vec{v}=\Vec{u}+\Vec{t}$
\end{enumerate}


\leavevmode\exercice


Dans le plan, on consid\`ere le triangle quelconque $ABC$. On note respectivement $I$ et $J$ les sym\'etriques respectifs de $B$ et de $C$ par rapport \`a $A$ :

\leavevmode\hfil\hbox{\Image{5290_dessin-1.pdf}{1}{67}{33}}

Exprimer en fonctions des vecteurs $\Vec{AB}$ et $\Vec{AC}$ les vecteurs suivants :

\questioni{a&\Vec{IA}&
b&\Vec{AJ}&
c&\Vec{BC}\cr
d&\Vec{CB}&
e&\Vec{IJ}&
f&\Vec{IC}\cr}

\leavevmode\exercice


Consid\'erons un triangle $ABC$ et $M$ un point appartenant au c\^ot\'e $[AB]$ v\'erifiant la relation :\newline
\hglue\leftmargini$AM=\dfrac23{\cdot}AB$

$P$ est le point d'intersection de la droite $(BC)$ et de la parall\`ele \`a $(AC)$ passant par le point $M$. $N$ est le point d'intersection des droites $(AC)$ et de la parall\`ele \`a $(AB)$ passant par le point $P$

\vskip\parskip
\begin{enumerate}
\item R\'ealiser une repr\'esentation de cette configuration.

\item Montrer que :\quad$AN=\dfrac13{\cdot}AC$%
\quad\string;\quad%
$CP=\dfrac23{\cdot}CB$.

\item D\'ecomposer les vecteurs ci-dessous en fonction des vecteurs $\Vec{AB}$ et $\Vec{AC}$ :

\questionii{a&\Vec{AP}&
b&\Vec{MC}\cr}

\item D\'ecomposer les vecteurs ci-dessous en fonction des vecteurs $\Vec{CA}$ et $\Vec{CB}$ :

\questionii{a&\Vec{AP}&
b&\Vec{NM}\cr}
\end{enumerate}

\leavevmode\exercice


\dimen0=5cm\dimen1\hsize\advance\dimen1-\dimen0
\begin{minipage}{\dimen1}
On consid\`ere le triangle ci-contre o\`u $I$ et $G$ sont les milieux respectifs des segments $[AB]$ et $[CI]$, le point $J$ est d\'efinie par la relation :

\vskip0.2cm
\hglue\leftmargini$\Vec{CJ}=\dfrac13{\cdot}\Vec{CA}$
\end{minipage}
\begin{minipage}{\dimen0}
\null\hfill\hbox{\Image{5393_dessin-1.pdf}{1}{48}{30}}
\end{minipage}

On munit le plan du rep\`ere $\big(A\,;\,\Vec{AB}\,;\,\Vec{AC}\big)$.

\vskip\parskip
\begin{enumerate}
\item Donner les coordonn\'ees des points $I$ et $J$.

\item Etablir que le point $G$ a pour coordonn\'ees $\coord{\dfrac14}{\dfrac12}$. Justifier votre r\'eponse.

\item En d\'eduire l'alignement des points $B$, $G$, $J$.
\end{enumerate}

\end{multicols*}

\end{document}