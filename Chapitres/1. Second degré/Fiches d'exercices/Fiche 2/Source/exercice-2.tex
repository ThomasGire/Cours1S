\documentclass{book}
\usepackage[utf8]{inputenc}
%\usepackage[latin1]{inputenc}
\usepackage[T1]{fontenc}
\usepackage{amsmath,amssymb}
\usepackage{array,color,multirow,slashbox,multicol}
\usepackage{mathrsfs}
\usepackage{eurosym}
\usepackage{stmaryrd} %Pour le symbole parallele \\sslash
\usepackage{yhmath}   %pour dessiner les arcs \wideparen
%\usepackage{multicol}
\usepackage[portrait,nofootskip]{chingatome}








\begin{document}

\fontsize{10}{12}\fontfamily{cmr}\selectfont\titre[18pt]{2nd degré - Fiche d'exercices 2}

\fontsize{10}{12}\fontfamily{cmr}\selectfont\begin{multicols*}{2}


%%%%%%%%%%%%%%%%%
\leavevmode\exercice


On consid\`ere la parabole $\mathcal{P}$ d'\'equation $\Se y=x^2-x-10$ et la droite $\mathcal{D}$ d'\'equation $\Se y=2x-1$.

\vskip\parskip
\begin{enumerate}
\item D\'eterminer les coordonn\'ees des points d'intersection de $\mathcal{D}$ et $\mathcal{P}$.

\item Donner les valeurs de $x$ pour lesquelles le point d'abscisse de $\mathcal{P}$ se trouve au dessus du point, de m\^eme abscisse, de $\mathcal{D}$.
\end{enumerate}

\leavevmode\exercice


Dans le plan muni d'un rep\`ere $\Se\big(O\,;\,I\,;\,J\big)$, on consid\`ere les courbes $\mathscr{C}_f$ et $\mathscr{C}_g$ repr\'esentatives des fonctions $f$ et $g$ d\'efinies par :\newline
\hglue\leftmargini$f(x)=x^2+\dfrac32{\cdot}x-1$%
\quad\string;\quad%
$g(x)=-\dfrac12{\cdot}x^2+x+1$

\leavevmode\hfil\hbox{\Image{2973_graph-1.pdf}{1}{92}{48}}

On r\'epondra alg\'ebriquement aux questions ci-dessous :

\vskip\parskip
\begin{enumerate}
\item D\'eterminer les z\'eros des fonctions $f$ et $g$.

\item D\'eterminer, alg\'ebriquement, la position relative des courbes $\mathscr{C}_f$ et $\mathscr{C}_g$.
\end{enumerate}

\leavevmode\exercice


On consid\`ere la fonction $f$ d\'efinie sur $\mathbb{R}$ par la relation :\newline
\hglue\leftmargini$f(x)=-x^2+2x+1$

Ci-dessous est donn\'ee la courbe $\mathscr{C}_f$ repr\'esentative de la fonction $f$ dans un rep\`ere $\Se\big(O\,;\,I\,;\,J\big)$ orthonorm\'e :

\leavevmode\hfil\hbox{\Image{5742_graph-1.pdf}{1}{93}{70}}

\vskip\parskip
\begin{enumerate}
\item D\'eterminer les z\'eros de la fonction $f$.

\item On consid\`ere la fonction affine $g$ d\'efinie par la relation :\newline
\hglue\leftmarginii$g(x)=-\dfrac12{\cdot}x-\dfrac12$

\vskip\parskip
\begin{enumerate}
\item Tracer dans le rep\`ere ci-dessous la droite $(d)$ repr\'esentative de la fonction $g$.

\item Alg\'ebriquement, \'etudier les positions relatives des courbes $\mathscr{C}_g$ et $\mathscr{C}_f$
\end{enumerate}
\end{enumerate}

\leavevmode\exercice


R\'esoudre les in\'equations suivantes :

\questioni{a&2x^2-8x+2\geq 0&
b&\dfrac{3x^2-5x+2}{-3x^2+4x-2}\leq 0\cr
c&\dfrac{2x-5}{2x-1}<\dfrac{x+1}{x+3}\cr}

\leavevmode\exercice


R\'esoudre dans $\mathbb{R}$ l'in\'equation :\qquad$\dfrac{-3x^2+4x+4}{5x^2+x-4}\geq0$



\leavevmode\exercice


On consid\`ere la fonction $f$ dont l'image d'un nombre r\'eel $x$ est donn\'ee par la relation :\newline
\hglue\leftmargini$f(x)=\dfrac{x+2}{x^2+2x-3}$

\vskip\parskip
\begin{enumerate}
\item R\'esoudre l'in\'equation :%
\quad$\Se f(x)\geq0$

\item Parmi les quatre courbes ci-dessous, une seule est la courbe repr\'esentative de la fonction $f$. Laquelle?

\hglue-\leftmargini\hfil\hbox{\Image{5743_graph-3.pdf}{1}{43}{26}}%
\hfil%
\hbox{\Image{5743_graph-2.pdf}{1}{43}{26}}

\hglue-\leftmargini\hfil\hbox{\Image{5743_graph-1.pdf}{1}{43}{26}}%
\hfil%
\hbox{\Image{5743_graph-4.pdf}{1}{43}{26}}
\end{enumerate}

\leavevmode\exercice


On consid\`ere les deux fonctions $f$ et $g$ d\'efinies respectivement ur $\mathbb{R}\backslash\{-1\}$ et $\mathbb{R}$ par les relations :\newline
\hglue\leftmargini$f(x)=\dfrac{2{\cdot}x^2+3{\cdot}x-3}{x+1}$%
\quad\string;\quad%
$g(x)=x-2$

R\'esoudre l'in\'equation :%
\quad$\Se f(x)\geq g(x)$

\leavevmode\exercice


On consid\`ere les fonctions $f$ et $g$ d\'efinies sur $\mathbb{R}$ d\'efinies par les relations :

\hglue\leftmargini$f(x)=x^2+x+1$%
\quad\string;\quad%
$g(x)=-2x^2-3x+5$

\vskip\parskip
\begin{enumerate}
\item Etablir le tableau de variation de chacune de ces fonctions.

\item Etablir le tableau de signe de chacune de ces fonctions.
\end{enumerate}


\leavevmode\exercice


On consid\`ere la fonction $f$ dont l'image de $x$ est d\'efinie par la relation :

\hglue\leftmargini$f(x)=\sqrt{-x^2+2x+1\mathstrut}$

\vskip\parskip
\begin{enumerate}
\item D\'eterminer l'ensemble de d\'efinition de la fonction $f$.

\item Dresser le tableau de variation de la fonction $f$.
\end{enumerate}

\leavevmode\exercice


On consid\`ere la fonction $f$ d\'efinie sur $\mathbb{R}^*$ d\'efinie par la relation :\newline
\hglue\leftmargini$f(x)=2\cdot\Big(\dfrac1x\Big)^2-3\cdot\dfrac1x+5$

\vskip\parskip
\begin{enumerate}
\item Dresser le tableau de signe et le tableau de variation du polyn\^ome du second degr\'e :

\hglue\leftmarginii$P(x)=2x^2-3x+5$

\item En remarquant que la fontion $f$ est la compos\'ee de la fonction inverse avec ce polyn\^ome du second degr\'e, \'etablir que la fonction est d\'ecroissante $\Se\Big]0\,;\,\dfrac43\Big]$

\item Dresser le tableau de variation de la fonction $f$ {\it(ne pas chercher \`a remplir le tableau avec les valeurs des images.)}
\end{enumerate}

\leavevmode\exercice


On consid\`ere la fonction $f$ d\'efinie sur $\mathbb{R}$ par :\newline
\hglue\leftmargini$f(x)=\dfrac1{\big(x+1)^2+2}$

D\'eterminer le sens de variation de la fonction $f$ sur l'intervalle $\Se\big[-1\,;\,+\infty\big[$.

\leavevmode\exercice


Dans le plan muni d'un rep\`ere $\Se\big(O\,;\,I\,;\,J\big)$, on consid\`ere la droite $(d)$ passant par les points $A\coord{4}{0}$ et $J$.

\leavevmode\hfil\hbox{\Image{6691_graph-1.pdf}{1}{81}{33}}

On consid\`ere un point $M$ appartenant \`a la droite $(d)$ et d'abscisse $x$ tel que $\Se x\in\big]0\,;\,4\big[$.

D\'eterminer la position du point $M$ sur la droite $(d)$ telle que le rectangle $OPMQ$ et le triangle $MPA$ aient la m\^eme aire.

{\sl\bfseries Toute trace de recherche et de prise d'iniatives seront prises en compte au cours de l'\'evaluation.}

\leavevmode\exercice


\dimen1=62mm\dimen0\hsize\advance\dimen0-\dimen1
\begin{minipage}[t]{\dimen1}
\leavevmode\Baisse{1}{56}\hbox{\Image{2955_dessin-1.pdf}{1}{58}{59}}\hfill
\end{minipage}
\begin{minipage}[t]{\dimen0}
On consid\`ere un carr\'e $ABCD$ de 5 centim\`etres de c\^ot\'e ; un point $I$ appartient \`a la diagonale $[AC]$, il est rep\'er\'e comme l'indique la figure ci-dessous par la longueur $x$ :

\vskip0.2cm
A partir de ce point $I$, on construit deux carr\'es de diagonale respectives $[AI]$ et $[IC]$.
\end{minipage}

D\'eterminer la valeur de $x$ pour laquelle la somme des aires de ces deux carr\'es vaut les $\quotient34$ de l'aire du carr\'e $ABCD$.

\leavevmode\exercice


{\bf Dans cet exercice, toute trace de recherche m\^eme incompl\`ete, ou d'initiative m\^eme non fructueuse, sera prise en compte dans l'\'evaluation.}

\dimen0=4.6cm\dimen1\hsize\advance\dimen1-\dimen0
\begin{minipage}[t]{\dimen1}
On consid\`ere un rectangle $ABCD$ dont les dimensions sont donn\'ees ci-dessous :\newline
\hglue\leftmargini$\Se AB=6\,m$%
\quad\string;\quad%
$\Se AD=4\,m$.

Pour un nombre r\'eel $x$ compris entre $0$ et $4$, on place les points $M$ et $N$ respectivement sur les
\end{minipage}
\begin{minipage}[t]{\dimen0}
\leavevmode\hfill\Baisse{1}{28}\hbox{\Image{5972_dessin-1.pdf}{1}{44}{31}}
\end{minipage}

c\^ot\'es $[AB]$ et $[BC]$ tels que :%
\quad$\Se AM=x$%
\quad\string;\quad%
$\Se BN=x$

D\'eterminer la ou les valeurs possibles de $x$ pour que l'aire du triangle $MBN$ soit \'egales \`a $\dfrac16$ de l'aire totale du rectangle $ABCD$.

\leavevmode\exercice


Dans le plan muni d'un rep\`ere $\Se(O\,;\,I\,;\,J)$ orthonorm\'e, on consid\`ere la repr\'esentation des deux fonctions $f$ et $g$ dont l'image de $x$ est d\'efini par :

\hglue\leftmargini$f(x)=\dfrac8{x+1}$%
\quad\string;\quad%
$g(x)=\dfrac{-6}{x+1}+6$

\leavevmode\hfil\hbox{\Image{2956_graph-1.pdf}{1}{85}{55}}

Le nombre $x$ appartient \`a l'intervalle $\Se[0\,;\,8]$. On consid\`ere les points $A$ et $B$ d'abscisse $x$ appartenant respectivement aux courbes repr\'esentatives $\mathscr{C}_f$ et $\mathscr{C}_g$.

Parall\`element aux axes, on construit deux rectangles repr\'esent\'es ci-dessus ; on note $\mathcal{A}_1$ et $\mathcal{A}_2$ chacune de leurs aires.

\vskip\parskip
\begin{enumerate}
\item D\'eterminer l'expression des aires $\mathcal{A}_1$ et $\mathcal{A}_2$ en fonction de la valeur de $x$.

\item D\'eterminer pour quelles valeurs de $x$, on a :%
\quad$\mathcal{A}_2\geq \mathcal{A}_1$
\end{enumerate}

\end{multicols*}

\end{document}