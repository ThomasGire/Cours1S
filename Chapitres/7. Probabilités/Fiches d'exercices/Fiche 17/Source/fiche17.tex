\documentclass{book}
\usepackage[utf8]{inputenc}
\usepackage[T1]{fontenc}
\usepackage{amsmath,amssymb}
\usepackage{array,color,multirow,slashbox,multicol}
\usepackage{mathrsfs}
\usepackage{eurosym}
\usepackage{stmaryrd} %Pour le symbole parallele \\sslash
\usepackage{yhmath}   %pour dessiner les arcs \wideparen
%\usepackage{multicol}
\usepackage[portrait,nofootskip]{chingatome}








\begin{document}

\fontsize{10}{12}\fontfamily{cmr}\selectfont\titre[18pt]{Lois de probabilités}
\fontsize{10}{12}\fontfamily{cmr}\selectfont\begin{multicols*}{2}


%%%%%%%%%%%%%%%%%
\leavevmode\exercice


Un tournoi d'\'echec affronte deux \'equipes contenant chacune un homme et une femme. Une partie oppose une personne de chaque \'equipe.
On choisi au hasard une personne de chaque \'equipe pour s'affronter au cours d'une partie. On consid\`ere les trois \'ev\`enements qui ``{\sl omposent}'' l'univers des possibilit\'es :

\vskip\parskip
\begin{itemize}
\item $A$ : ``{\sl Deux hommes s'affrontent dans cette partie}''

\item $B$ : ``{\sl Deux femmes s'affrontent dans cette partie}''

\item $C$ : ``{\sl Un homme et une femme s'affrontent dans cette partie}''
\end{itemize}

\vskip\parskip
\begin{enumerate}
\item Conjecturer la probabilit\'e de chacun de ces \'ev\`enements.

\item On utilise la notation suivante pour d\'esigner la composition de chaque groupe :\newline
\hglue\leftmarginii$\mathcal{G}_1=\big\{H_1\,;\,F_1\big\}$%
\quad\string;\quad%
$\mathcal{G}_2=\big\{H_2\,;\,F_2\big\}$

\vskip\parskip
\begin{enumerate}
\item D\'ecrire toutes les parties organisables lors de ce tournoi.

\item Donner la probabilit\'e des \'ev\`enements $A$, $B$ et $C$.
\end{enumerate}
\end{enumerate}


\leavevmode\exercice


Apr\`es \'etude d'un d\'es truqu\'e dont les faces sont num\'erot\'ees de 1 \`a 6, on obtient la loi de probabilit\'e suivante :

\leavevmode\hfil\vbox{\halign{\vrule\vrule width0pt depth5pt height12pt\ $#$\ \hfil\vrule&&\hbox to1cm{\hfil$#$\hfil}\vrule\cr
\noalign{\hrule}
x_i&1&2&3&4&5&6\cr
\noalign{\hrule}
p_i&0,2&0,15&0,12&0,17&0,08&0,28\cr
\noalign{\hrule}}}

D\'eterminer les probabilit\'es de chacun des \'el\'ements suivants :

\vskip\parskip
\begin{enumerate}
\item A : ``{\slshape Le r\'esultat est sup\'erieur ou \'egal \`a 4''}.

\item B : ``{\slshape Le r\'esultat est un nombre impair}''.

\item C : ``{\slshape Le r\'esultat est un nombre pair}''.
\end{enumerate}

\leavevmode\exercice


On lance deux d\'es \'equilibr\'es. D\'eterminer la probabilit\'e de chacun des \'ev\'enements ci-dessous :

\vskip\parskip
\begin{enumerate}
\item Ev\'enement A : ``{\sl on obtient un 6 et un 2}'' ;
\item Ev\'enement B : ``{\sl la somme obtenu est strictement sup\'erieure \`a 8}'' ;
\item Ev\'enement C : ``{\sl les deux nombres obtenus sont pairs}''.
\end{enumerate}


\leavevmode\exercice


\dimen0=5.3cm\dimen1\hsize\advance\dimen1-\dimen0
\begin{minipage}[t]{\dimen1}
Une exp\'erience al\'eatoire consiste \`a lancer deux d\'es, rouge et bleu, \`a six faces sumultan\'ement et \`a consid\'erer la somme obtenue par ces deux d\'es. On suppose les d\'es parfaitement \'equilibr\'es.

\vskip0.2cm
\begin{enumerate}
\item D\'ecrire l'univers des issues possibles.

\item \begin{enumerate}
\item Compl\'eter le tableau ci-dessous :

\end{enumerate}
\end{enumerate}
\end{minipage}
\begin{minipage}[t]{\dimen0}
\leavevmode\hfill\Baisse{1}{50}\hbox{\Image{5187_dessin-1.pdf}{1}{52}{52}}
\end{minipage}

\vskip\parskip
\begin{enumerate}
\item[] \begin{enumerate}
\item D\'eterminer la loi de probabilit\'e associ\'ee \`a cette exp\'erience al\'eatoire.
\end{enumerate}
\end{enumerate}

\leavevmode\exercice


On consid\`ere une urne contenant $11$ boules. Certaines sont rondes, d'autres carr\'es. Certaines sont blanches, d'autres sont ray\'es. Elles sont repr\'esent\'es ci-dessous :

On suppose qu'en appuyant sur un bouton, les boules sortent au hasard de l'urne. On vient de constituer une exp\'erience al\'eatoire suivant la loi d'\'equiprobabilit\'e.

On associe un gain \`a chacune des boules de la mani\`ere suivante :
\vskip\parskip
\begin{itemize}
\item Une boule rapporte $1\,$\euro{} alors qu'un carr\'e rapporte $2\,$\euro{}.

\item De plus, si l'\'el\'ement est ray\'e, le gain est augment\'e de $1\,$\euro{}.
\end{itemize}

Cette association d'une valeur \`a chaque \'ev\`enement \'el\'ementaire constitue une variable al\'eatoire. Notons la $\mathcal{X}$.

\vskip\parskip
\begin{enumerate}
\item D\'eterminer la loi de probabilit\'e de la variable $\mathcal{X}$ lorsque le contenu de l'urne est repr\'esent\'e ci-dessous :

\hglue-\leftmargini\hfil\hbox{\Image{7663_dessin-1.pdf}{1}{78}{8}}

\item D\'eterminer la loi de probabilit\'e de la variable $\mathcal{X}$ lorsque le contenu de l'urne est repr\'esent\'e ci-dessous :

\hglue-\leftmargini\hfil\hbox{\Image{7663_dessin-2.pdf}{1}{78}{8}}
\end{enumerate}

\leavevmode\exercice


Une urne contient quatre boules bleues num\'erot\'ees de $1$ \`a $4$, trois boules rouges num\'erot\'ees de $1$ \`a $3$ et deux boules vertes num\'erot\'ees de $1$ \`a $2$.

\vskip\parskip
\begin{enumerate}
\item On note $\mathcal{X}$ la variable al\'eatoire qui associe \`a chacune des boules le num\'ero inscrit sur celui-ci.\newline
D\'eterminer la loi de probabilit\'e de la variable al\'eatoire $\mathcal{X}$.

\item Au tirage d'une boule dans cette urne, on associe les r\`egles de jeu suivantes :
\vskip\parskip
\begin{itemize}
\item Si la boule tir\'ee est bleu et porte un entier pair, le joueur gagne $2\,$\euro.

\item Si la boule tir\'ee n'est pas bleu et porte un entier pair, le joueur gagne $3\,$\euro.

\item Sinon le joueur ne gagne rien.
\end{itemize}

On note $\mathcal{Y}$ la variable al\'eatoire qui associe au tirage d'une boule le gain obtenu.

D\'eterminer la loi de probabilit\'e de la variable al\'eatoire $\mathcal{Y}$.
\end{enumerate}

\leavevmode\exercice


Dans une jeu bas\'ee sur une exp\'erience al\'eatoire, la variable al\'eatoire $\mathcal{X}$ mesure le gain r\'ealis\'e par le participant. Le tableau suivant pr\'esente la loi de probabilit\'e de la variable $\mathcal{X}$ :

\leavevmode\hfil\vbox{\offinterlineskip
\halign{\vrule\vrule width0pt height12pt depth5pt\ $#$\ \hfill\vrule&&\hbox to1.2cm{\hfill$#$\hfill}\vrule\cr
\noalign{\hrule}
x&0&1&2&3&6\cr
\noalign{\hrule}
\mathcal{P}(\mathcal{X}{=}x)&0,34&0,3&0,19&0,15&0,02\cr
\noalign{\hrule}}}

\vskip\parskip
\begin{enumerate}
\item D\'eterminer les probabilit\'es suivantes :\newline
\hglue\leftmarginii$\mathcal{P}(\mathcal{X}{<}3)$%
\quad\string;\quad%
$\mathcal{P}(\mathcal{X}{\geq}3)$%
\quad\string;\quad%
$\mathcal{P}(2{\leq}\mathcal{X}{<}5)$

\item D\'eterminer l'esp\'erance de cette variable al\'eatoire.
\end{enumerate}

\leavevmode\exercice


\dimen0=35mm\dimen1\hsize\advance\dimen1-\dimen0
\begin{minipage}[t]{\dimen1}
Un QCM {\it(questionnaire \`a choix multiples)} est propos\'e \`a des \'el\`eves : il comporte trois questions et pour chacune de ces questions, quatre r\'eponses sont propos\'ees dont une seule est juste.

On souhaite \'etudier le pourcentage de r\'eussite \`a ce QCM si les \'el\`eves y r\'epondent compl\'etement de r\'eponse al\'eatoire ; on suppose alors que les r\'eponses donn\'ees \`a chacune des questions sont ind\'ependantes entre elles.
\end{minipage}
\begin{minipage}[t]{\dimen0}
\leavevmode\hfill\Baisse{1}{44}\hbox{\Image{5190_arbreProbabilite2-1.pdf}{1}{33}{48}}
\end{minipage}

On note :
\vskip\parskip
\begin{itemize}
\item $F_i$ : ``{\sl La r\'eponse fournit \`a la question $i$ est fausse}'' ;
\item $V_i$ : ``{\sl La r\'eponse fournit \`a la question $i$ est vraie}'' ;
\end{itemize}

\vskip\parskip
\begin{enumerate}
\item Compl\'eter l'arbre pond\'er\'e pr\'esent\'e ci-dessus.

\item On note $\mathcal{X}$ la variable al\'eatoire comptant le nombre de bonnes r\'eponses fournies au QCM.

\vskip\parskip
\begin{enumerate}
\item D\'eterminer la loi de probabilit\'e de la variable al\'eatoire $\mathcal{X}$.

\item Calculer l'esp\'erance de la variable al\'eatoire $\mathcal{X}$.
\end{enumerate}
\end{enumerate}


\leavevmode\exercice


En fin d'ann\'ee l'association des \'el\`eves d'un lyc\'ee organise une tombola : $100$ tickets sont mis en vente \`a $10$ euros l'unit\'e.

Voici les diff\'erents tickets gagnants :

\vskip\parskip
\begin{itemize}
\item 2 tickets gagnet $50\,$\euro ;

\item 10 tickets gagnent $20\,$\euro ;

\item 20 tickets gagnent $10\,$\euro.
\end{itemize}

\vskip\parskip
\begin{enumerate}
\item Quelle est la somme des gains de cette tombola?

\item D\'eterminer les probabilit\'es des \'ev\'enements suivants  :

\vskip\parskip
\begin{itemize}
\item $A$ : ``{\sl le ticket ne gagne rien}'' ;
\item $B$ : ``{\sl le ticket gagne $10\,$\euro}'' ;
\item $C$ : ``{\sl le ticket gagne $20\,$\euro}'' ;
\item $D$ : ``{\sl le ticket gagne $50\,$\euro}'' ;
 \end{itemize}

\item On consid\`ere la variable al\'eatoire $\mathcal{X}$ qui associe \`a chaque ticket la valeur du ticket gagnant :

\vskip\parskip
\begin{enumerate}
\item D\'eterminer l'esp\'erance $E(\mathcal{X})$ de la variable al\'eatoire $\mathcal{X}$.

\item D\'eterminer la variance $V(\mathcal{X})$ et l'\'ecart type $\sigma(\mathcal{X})$ de la variable al\'eatoire $\mathcal{X}$. {\it(on arrondira les valeurs au dixi\`eme pr\`es)}.
\end{enumerate}
\end{enumerate}


\leavevmode\exercice


\dimen0=2.8cm\dimen1\hsize\advance\dimen1-\dimen0
\begin{minipage}[t]{\dimen1}
Un jeu consiste \`a tirer une carte dans un jeu de $32$ cartes. On associe \`a chaque carte un gain :
\vskip0.2cm
\begin{itemize}
\item Le Roi de Coeur rapporte $5$\,\euro.

\item Une autre figure de Coeur rapporte $3$\,\euro.

\item Une autre figure rapporte $1$\,\euro.

\item Les autres cartes ne font pas gagner.
\end{itemize}
\end{minipage}
\begin{minipage}[t]{\dimen0}
\leavevmode\hfill \Baisse{1}{42}\hbox{\Image{5189_dessin-1.pdf}{1}{25}{46}}
\end{minipage}

On mod\'elise le gain de ce jeu par la variable al\'eatoire $\mathcal{X}$.

\vskip\parskip
\begin{enumerate}
\item Donner la loi de probabilit\'e de la variable al\'eatoire $\mathcal{X}$.

\item \begin{enumerate}
\item D\'eterminer la valeur exacte de l'esp\'erance de la variable al\'eatoire $\mathcal{X}$.

\item Si la mise d'une partie est de $1$\,\euro, ce jeu est-il favorable ou d\'efavorable \`a l'organisateur.
\end{enumerate}

\item D\'eterminer la variance et l'\'ecart type de la variable al\'eatoire $\mathcal{X}$  arrondie au centi\`eme pr\`es.
\end{enumerate}

\leavevmode\exercice


Une fabrique de chocolats construit dans l'ann\'ee des bo\^ites de chocolats dont $50\,\%$ avec du chocolats au lait, $30\,\%$ de chocolats noirs et $20\,\%$ de chocolats blancs.

$70\,\%$ des bo\^ites pr\'esentent des chocolats natures alors que les autres bo\^ites contiennent des chocolats sont fourr\'es de caramel. Ces proportions sont ind\'ependantes du chocolat utilis\'e pour confectionner la boite.

On consid\`ere les \'ev\`enements :
\vskip\parskip
\begin{itemize}
\item $L$ : ``{\sl le chocolat au lait est utilis\'e}'' ;
\item $N$ : ``{\sl le chocolat noir est utilis\'e}'' ;
\item $B$ : ``{\sl le chocolat blanc est utilis\'e}'' ;
\item $Na$ : ``{\sl les chocolats sont natures}'' ;
\item $C$ : ``{\sl les chocolats sont fourr\'es au caramel}'' ;
\end{itemize}

Tous les r\'esultats seront donn\'es sous forme d\'ecimale.

\vskip\parskip
\begin{enumerate}
\item Dresser l'arbre pond\'er\'e associ\'e \`a cette situation.

\item On choisit en sortie d'usine, au hasard, une boite produite. D\'eterminer les probabilit\'es des \'ev\`enemnts suivants :

\vskip\parskip
\begin{enumerate}
\item ``{\sl la boite contient des chocolats noir et nature}''

\item ``{\sl la boite contient des chocolats noir ou nature}''
\end{enumerate}

\item L'entreprise fixe les prix des bo\^ites de la mani\`ere suivante :
\vskip\parskip
\begin{itemize}
\item le prix de base d'une bo\^ite de chocolat est de $9$\,\euro ;

\item si le chocolat utilis\'e est le chocolat noir alors le prix est major\'e de $4$\,\euro ;

\item si le chocolat utilis\'e est le chocolat blanc alors le prix est major\'e de $2$\,\euro ;

\item si les chocolats sont fourr\'es au caramel, le prix de la bo\^ite augmente de $2$\,\euro.
\end{itemize}

La variable al\'eatoire $\mathcal{X}$ associe \`a bo\^ite produit par l'usine son prix de ventre.

\vskip\parskip
\begin{enumerate}
\item Dresser le tableau repr\'esentant la loi de probabilit\'e associ\'ee \`a la variable al\'eatoire $\mathcal{X}$.

\item D\'eterminer l'esp\'erance de la variable al\'eatoire $\mathcal{X}$ arrondi au dixi\`eme pr\`es.
\end{enumerate}
\end{enumerate}


\end{multicols*}


\end{document}